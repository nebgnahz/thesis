\subsection{JetStream++}
\label{sub:jetstream++}

We modified the open source version of
JetStream\footnote{\url{https://github.com/princeton-sns/jetstream/}, \newline
  commit bf0931b2d74d20fdf891669188feb84c96AF84.} in order to use our profile as
its manual policy. Because JetStream doesn't support simultaneous degradation
across multiple operators, we implemented a simple \texttt{VideoSource} operator
that understands how to change image resolutions, frame rate, and video encoding
quantization. At runtime, \texttt{VideoSource} queries congestion policy manager
and adjusts three dimensions simultaneously. This operator is then exposed to
the Python-implemented control plane. We call this modified version
JetStream++.\footnote{\url{https://github.com/awstream/jetstream-clone/pull/1}}

JetStream's code base is modular and extensible: the modifications include 53
lines for the header file, 171 lines for implementation, 75 lines for unit test,
and 49 lines of python as the application. While extending JetStream with our
profile is not challenging, JetStream++ performs degradation in a single
operator and loses the composability. We could modify JetStream to support
degradation across multiple operators, but that would require substantial
changes to JetStream. Using JetStream++ with our profile, the comparison is
enough to illustrate the difference between \sysname{}'s and JetStream's
runtime.

%%% Local Variables:
%%% mode: latex
%%% TeX-master: "../network"
%%% End:

%% LocalWords: Mbps analytics runtime JetStream OpenCV YOLO pre GStreamer appsrc
%% LocalWords: appsink zerolatency quantization dataset SVM geo topk VideoSource
%% LocalWords: JetStream's composability TK TCP UDP HLS FFmpeg bitrates nginx
%% LocalWords: packetization TPUT topk NodeJS metadata timestamp Mbps Kbps
%% LocalWords: aws PD's TK's hls js VoD awstream tk fdf feb
