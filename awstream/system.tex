\section{\awstream{} Design}
\label{sec:system}

To address the issues with manual policies or application-specific
optimizations, \awstream{} structures adaptation as a set of approximate,
modular, and extensible specifications (\autoref{sec:structure-adapt}). The
well-defined structure allows us to build a generic profiling tool that learns
an accurate relationship---we call it the profile---between bandwidth
consumption and application accuracy (\autoref{sec:automatic-profiling}). The
profile then guides the runtime to react with precision: achieving low latency
and high accuracy when facing insufficient bandwidth
(\autoref{sec:runtime}). \autoref{fig:overview} shows the high-level overview of
\awstream{}.

\begin{figure}
  \centering
  \includegraphics[width=0.8\linewidth]{figures/system.pdf}
  \caption{\todo{update section number}. \awstream{}'s phases: development,
    profiling, and runtime. \awstream{} also manages wide-area deployment.}
  \label{fig:overview}
\end{figure}

\input{awstream/api}
\input{awstream/profiling}
\input{awstream/runtime}

%%% Local Variables:
%%% mode: latex
%%% TeX-master: "../network"
%%% End:
