\section{Related Work}
\label{sec:related-work}

\paraf{JetStream.} JetStream is the first to use degradation to reduce latency
for wide-area streaming analytics. Compared to JetStream, \sysname{} makes five
major contributions: $(1)$~a novel API design to specify degradation in a simple
and composable way; $(2)$~automatic offline profiling to search for
Pareto-optimal configurations; $(3)$~online profiling to address model drift;
$(4)$~an improved runtime system achieving sub-second latency (comparison in
\autoref{sec:runtime-adaptation}); $(5)$~support for different resource
allocation policies for multiple applications.

\para{Stream Processing Systems.} Early streaming
databases~\cite{abadi2005design, chandrasekaran2003telegraphcq} have explored
the use of dataflow models with specialized operators for stream
processing. Recent research projects and open-source
systems~\cite{akidau2013millwheel, toshniwal2014storm, sanjeev2015twitter,
  zaharia2013discretized, carbone2015apache} primarily focus on fault-tolerance
in the context of a single cluster. When facing back pressure,
Storm~\cite{toshniwal2014storm}, Spark Streaming~\cite{zaharia2013discretized}
and Heron~\cite{sanjeev2015twitter} throttle data ingestion; Apache
Flink~\cite{carbone2015apache} uses edge-by-edge back-pressure techniques
similar to TCP flow control; Faucet~\cite{lattuada2016faucet} leaves the flow
control logic up to developers.  While our work has a large debt to prior
streaming work, \sysname{} targets at the wide area and explicitly explores the
trade-off between data fidelity and freshness.

\para{Approximate Analytics.} The idea of degrading computation fidelity for
responsiveness is also explored elsewhere, such as SQL
queries~\cite{hellerstein1997online, agarwal2013blinkdb,
  ananthanarayanan2014grass}, real-time processing~\cite{farrell2016meantime},
and video processing within large clusters~\cite{zhang2017live}. The trade-off
between available resource and application accuracy is a common theme among all
these systems. \sysname{} targets at wide-area streaming analytics, an emerging
application domain especially with the advent of IoT\@.

\para{WAN-Aware Systems.} Geo-distributed systems need to cope with high latency
and limited bandwidth. While some systems assume the network can prioritize a
small amount of critical data under congestion~\cite{cho2012surviving}, most
systems reduce data sizes in the first place to avoid congestion,
e.g.\,LBFS~\cite{muthitacharoen2001low}. Over the past two years, we have seen a
plethora of geo-distributed analytical frameworks~\cite{vulimiri2015wananlytics,
  vulimiri2015global, pu2015low, kloudas2015pixida, viswanathan2016clarinet}
that incorporate heterogeneous wide-area bandwidth into query optimization to
minimize data movement. While effective in speeding up analytics, these systems
focus on batch tasks such as MapReduce jobs or SQL queries. Such tasks are often
executed infrequently and without real time constrain. In contrast, \sysname{}
processes streams continuously in real time.

%% - Pixida~\cite{kloudas2015pixida} minimizes data movement across inter-DC
%% links by solving a traffic minimization problem in data analytics.

%% - Iridium~\cite{pu2015low} optimizes data and task placement jointly to
%% achieve an overall low latency goal.

%% - Clarinet~\cite{viswanathan2016clarinet} incorporate bandwidth information
%% into query optimizer to reduce data transfer.

%% WheelFS~\cite{stribling2009flexible}

%% Geode~\cite{vulimiri2015global} develops input data movement and join
%% algorithm selection strategies to minimize WAN bandwidth usage.

%% WANalytics~\cite{vulimiri2015wananlytics} focuses on batch analytics.

%% SWAG~\cite{hung2015scheduling} coordinates compute task scheduling across
%% DCs.

%% \cite{heintz2015towards} discusses the complex trade-offs involved in
%% wide-area analytics.

\para{(Adaptive) Video Streaming.} Multimedia streaming protocols,
e.g.\,RTP~\cite{schulzrinne2006rtp}, aim to be fast instead of reliable. While
they can achieve low latency, their accuracy can be poor under congestion.
Recent work has moved towards HTTP-based protocols and focused on designing
adaptation strategy to improve QoE, as in research~\cite{mao2017neural,
  sun2016cs2p, yin2015control} and industry (HLS~\cite{pantos2016http},
DASH~\cite{michalos2012dynamic, sodagar2011mpeg}). These adaptation strategies
are often pull-based: client keeps checking the index file for changes. And
clients have to wait for the next chunk (typically 2-10 seconds). In addition,
as we have shown in \autoref{sec:runtime-adaptation}, their adaptation is a poor
match for analytics that rely on image details (e.g.\,6\% accuracy for PD). In
contrast, \sysname{} learns an adaptation strategy for each application (also
not limited to video analytics); and the runtime is optimized for low latency.

\para{QoS.} Most QoS work~\cite{ferrari1990scheme, shenker1994integrated,
  shenker1995fundamental} in the 1990s focuses on network-layer solutions that
are not widely deployable. \sysname{} adopts an end-host application-layer
approach ready today for WAN. For other application-layer
approaches~\cite{vandalore2001survey}, \sysname{}'s API can incorporate their
techniques, such as encoding the number of layers as a knob to realize the
layered approach in Rejaie et al~\cite{rejaie2000layered}. Fundamentally,
\sysname{} does not provide hard QoS guarantees; instead \sysname{} maximizes
achievable accuracy (application performance) and minimizes latency (system
performance) with respect to bandwidth constraints: a multidimensional
optimization.

%% TCP, for example, adapts to available bandwidth by controlling the congestion
%% window with AIMD~\cite{jacobson1988congestion}.  Previous work has proposed
%% modifications to TCP for specific contexts. For streaming over TCP, Goel et
%% al.~\cite{goel2008low} proposes adaptive buffer-size tuning. For the cloud,
%% Adaptive TCP~\cite{wu2013adaptive} proposes to modify the congestion control
%% behavior based on flow size for the cloud.

% \para{Scheduling and allocation:} Resource allocations in the cloud is how to
% efficiently allocate tasks.  In the wide area, we face fundamental limit that
% therefore degradation is more effective. For multiple tasks, Existing
% allocations are for resources without considering application trade-offs.
% MediaNet~\cite{hicks2003user} uses both local and online global resource
% scheduling to improve user performance and network utilization, and adapts
% without requiring underlying support for resource
% reservations. VideoStorm~\cite{zhang2017live} primarily focuses on cluster
% resource allocation among video queries. For wide-area, the resource
% allocation is limited. We do not control the capacity; but still we can
% allocate the available bandwidth.

% Performance modeling: CherryPick~\cite{alipourfard2017cherrypick},
% Ernest~\cite{venkataraman2016ernest}.
%% Dolly~\cite{ananthanarayanan2013effective}.
%% \para{Program Synthesis:}

%%% Local Variables:
%%% mode: latex
%%% TeX-master: "../awstream"
%%% End:

%% LocalWords: JetStream analytics composable runtime Borealis TelegraphCQ
%% LocalWords: dataflow MillWheel Flink TCP BlinkDB VideoStorm IoT geo LBFS
%% LocalWords: WANalytics Pixida MapReduce RTP HLS QoE VoD QoS deployable Rejaie
%% LocalWords: et al
