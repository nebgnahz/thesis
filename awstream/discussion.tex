\section{Discussion and Future work}
\label{sec:discussion}

\paraf{Reducing Developer Effort.} While \sysname{} simplifies developing
adaptive applications, there are still application-specific parts required for
developers: wrapping appropriate \maybe{} calls, providing training data, and
implementing accuracy functions. Because \sysname{}'s API is extensible, we plan
to build libraries for common degradation operations and accuracy functions,
similar to machine learning libraries.

\para{Fault-tolerance and Recovery.} \sysname{} tolerates bandwidth variation
but not network partition or host failure. Although servers within data centers
can handle faults in existing systems, such as Spark
Streaming~\cite{zaharia2013discretized}, it is difficult to make edge clients
failure-oblivious.  We leave failure detection and recovery as a future work.

\para{Profile Modeling.} \sysname{} currently performs an exhaustive search when
profiling, and only use parallelism and sampling to speed up. We plan to improve
our profiler with statistical methods such as Bayesian
Optimization~\cite{snoek2012practical, alipourfard2017cherrypick} that can model
black box functions $B(c)$ and $A(c)$.

% \para{Expressiveness}: Our \maybe{} APIs allow an easy integration with
% existing stream processing systems. While it follows the operator model,
% combined with other operators, this is expressive enough. We've presented
% three applications in this paper; and we are implement more application using
% this framework to understand the expressiveness better.

% \para{Context detection.} \sysname{} is currently limited to one profile: the
% offline profiling generates the default profile and the online profiling
% updates the single profile continuously.  Real-world applications may produce
% data with a multi-modal distribution, where the model changes upon context
% changes, such as indoor versus outdoor. Therefore, one possible optimization
% to \sysname{} is to allow multiple profiles for one application, detect
% context changes, and use the profile that best matches the current data
% distribution.  Switching contexts could reduce, or even eliminate, the
% overhead of online profiling.

\para{Predicting Bandwidth Changes.} \sysname{} currently does not predict
future bandwidth. While reacting to bandwidth changes is enough to achieve
sub-second latency, we plan to improve our runtime further (e.g.\,removing
latency spikes) with techniques that can predict future resources, such as model
predictive control (MPC)~\cite{camacho2013model, yin2015control}.

%%% Local Variables:
%%% mode: latex
%%% TeX-master: "../awstream"
%%% End:

%% LocalWords: CherryPick runtime MPC QoE profiler