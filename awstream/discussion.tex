\section{Discussion}
\label{sec:discussion}

\paraf{Reducing Developer Effort.} While \awstream{} simplifies developing
adaptive applications, there are still application-specific parts required for
developers: wrapping appropriate \maybe{} calls, providing training data, and
implementing accuracy functions. Because \awstream{}'s API is extensible, we plan
to build libraries for common degradation operations and accuracy functions,
similar to machine learning libraries.

\para{Fault-tolerance and Recovery.} \awstream{} tolerates bandwidth variation
but not network partition or host failure. Although servers within data centers
can handle faults in existing systems, such as Spark
Streaming~\cite{zaharia2013discretized}, it is difficult to make edge clients
failure-oblivious.  We leave failure detection and recovery as a future work.

\para{Profile Modeling.} \awstream{} currently performs an exhaustive search when
profiling. While parallelism and sampling are effective, profiling complexity
grows exponentially with the number of knobs. Inspired by recent success of
using Bayesian Optimization~\cite{snoek2012practical, alipourfard2017cherrypick,
  solnik2017bayesian} to model black-box functions, we are currently exploring
multi-objective Bayesian Optimization~\cite{hernandez2016predictive} that can
find \textit{near-optimal} configurations without exhaustive search.

% \para{Expressiveness}: Our \maybe{} APIs allow an easy integration with
% existing stream processing systems. While it follows the operator model,
% combined with other operators, this is expressive enough. We've presented
% three applications in this paper; and we are implement more application using
% this framework to understand the expressiveness better.

\para{Context detection.} \awstream{} is currently limited to one profile: the
offline profiling generates the default profile and the online profiling
updates the single profile continuously.  Real-world applications may produce
data with a multi-modal distribution, where the model changes upon context
changes, such as indoor versus outdoor. Therefore, one possible optimization
to \awstream{} is to allow multiple profiles for one application, detect
context changes, and use the profile that best matches the current data
distribution.  Switching contexts could reduce, or even eliminate, the
overhead of online profiling.

\para{Bandwidth Estimation and Prediction.} Accurately estimating and predicting
available bandwidth in wide area remains a challenge~\cite{huang2012confused,
  zou2015can}. \awstream{} uses network throughput and behaves cautiously to
avoid building up queues: congestion is detected at both sender/receiver; data
rate only increases after probing.  Recent research on adaptive video streaming
explores model predictive control (MPC)~\cite{yin2015control, sun2016cs2p} and
neural network~\cite{mao2017neural}. We plan to explore these techniques next.

%%% Local Variables:
%%% mode: latex
%%% TeX-master: "../network"
%%% End:

%% LocalWords: CherryPick runtime MPC QoE profiler
