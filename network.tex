\documentclass[thesis.tex]{subfiles}

\begin{document}

\chapter{Network Resource Adaptation}
\label{cha:netw-reso-adapt}

In this chapter, we focus on swarm applications adapting to network
resources. Specifically, we target at swarm applications that rely on wide-area
streaming and perform analytics elsewhere, e.g., the cloud. These streaming
applications face the challenge of scarce and variable wide-area network (WAN)
bandwidth. Many applications today are built directly with TCP or UDP, and they
suffer from increased latency or degraded accuracy. State-of-the-art approaches
that adapt to network changes require developer writing sub-optimal manual
policies or are limited to application-specific optimizations.

We present \awstream{}, a stream processing system that uses adaptation to
simultaneously achieve low latency and high accuracy in the wide area, requiring
minimal developer efforts. To realize this, \awstream{} follows our proposed
methodology: $(i)$ it integrates application adaptation as a first-class
programming abstraction in the stream processing model; $(ii)$ with a
combination of offline and online profiling, it automatically learns an accurate
profile that models accuracy and bandwidth trade-off; and $(iii)$ at runtime, it
carefully adjusts the application data rate to match the available bandwidth
while maximizing the achievable accuracy. We evaluate \awstream{} with three
real-world applications: augmented reality, pedestrian detection, and monitoring
log analysis. Our experiments show that \awstream{} achieves sub-second latency
with only nominal accuracy drop (2-6\%).

\documentclass[thesis.tex]{subfiles}

\begin{document}

\chapter{Introduction}

Over the past two decades, we have seen a growing number of networked sensors
and actuators installed in our connected world. These sensors and actuators
offer an unprecedented ability to monitor and act. Because of the enormous
potentials in solving societal-scale problems, this trend has gained significant
attraction, as demonstrated by many parallel efforts: Internet of Things
(IoT)~\cite{atzori2010internet}, Internet of Everything
(IoE)~\cite{bradley2013internet}, Industry 4.0~\cite{lasi2014industry}, The
Industrial Internet~\cite{eigner2018industrial}, TSensors (Trillion
Sensors)~\cite{bogue2014towards}, Machine to Machine
(M2M)~\cite{anton2014machine}, Smarter Planet~\cite{palmisano2008smarter},
etc. In this thesis, we refer to this trend as the swarm because it well
characterizes where the potentials lie: the number and the scale of
interconnected devices.

Swarm applications generate, transport, distill, and process large streams of
data across the wide area. For example, large cities such as London and Beijing
have deployed millions of cameras for surveillance and traffic
control~\cite{skynet, london.surveillance}. Buildings are increasingly equipped
with a wide variety of sensors to improve energy efficiency and occupant
comfort~\cite{dawson2010smap, krioukov2012building}. Interactive applications
that involve humans in the loop require a low latency response to engage the
users.

Riding on the popularity of the cloud infrastructure, swarm applications have
adopted the cloud as a universal computation resource and a storage
backend~\cite{carriots, grovestreams, sami, xively, gupta2014bolt,
  zachariah1001internet}. While this trend is understandable given the economic
benefits and simplified management of the cloud, it is not the best long term
approach. We will discuss issues related with privacy, security, availability,
latency, bandwidth, etc.

To compensate the cloud, new infrastructure focusing edge computing appear, such
as the fog~\cite{bonomi2012fog, bar2013fog}, cloudlet~\cite{ha2014towards,
  satyanarayanan2009case, chen2018application}, and swarmbox. While reducing the
distance, these further increase the heterogeneous landscape: swarm components
are of various types, requiring interfacing and interoperability across multiple
platforms and models of computation.

We recognize the challenges in developing swarm applications: constrained
resource and heterogeneous environment. We argue that the key to unfold the
potential of swarm applications is to allow applications to adapt to environment
changes. Manual adaptation or exhaustive explore the design space are not
feasible given the scale and large problem space.

To reduce developer efforts, we propose a three-stage development framework:
API, profiling and runtime use. In this thesis, we propose to provide adaptation
as a core abstraction in application development. In this way, we can build
tools to automatically learn the effect of adaptation and apply adaptation
strategies accordingly. We demonstrate this methodology with two systems
focusing on network resources and compute resources for swarm applications.

\vspace{1em}

\noindent\textbf{Thesis Statement}: \textit{Providing adaptation as a programming
  abstraction allows for resilient swarm applications with less developer
  effort.}

\vspace{1em}

\section{Challenges with Developing Swarm Applications}
\label{sec:chall-with-exist}

We identify the following challenges that are unique to swarm applications:

\para{Limited Resource.} Swarm systems rely on vast numbers of heterogeneous
sensors that are generating massive amounts of data.

\para{Ad-hoc Development.} Developers strive to make the application working due
to many moving pieces.

\para{Huge Design Space.} Many devices. Choices of sensors, algorithms, data
quality.

\para{Heterogeneous Platforms.} Ranging from cheap low power platforms to
powerful workstation.

These challenges are hard to overcome.

\section{Adaptation as a First-class Citizen}
\label{sec:adaptation}

\para{API.} To encode adaption and allow tools to be built in a systematic way.
Such API can be reused both for development and for execution: with a effective
runtime.

\para{Automatic Profiling and Optimization.} Free developer from manual tuning
knobs or struggle to explore a huge parameter space.

Take network resource as an example, when facing situations where the bandwidth
is not sufficient, applications deployed today either choose a conservative
setting (e.g.\,only delivering 360p videos) or leave their fate to the
underlying transport layer: (1) in the case of TCP, the sender will be blocked
and data are backlogged, leading to severe delay; (2) in the case of UDP,
uncontrolled packet loss occurs, leading to application performance
drop. Instead of ``suffering'' from a degraded network, applications can act
proactively by adjusting their behavior: reducing the data rate to ensure that
important data are delivered in time.

This network-adaptation profile remains the same across devices.

When we apply the same principle to compute resource, in Chapter 3, we focus on
performance modeling and improve the efficiency. Unique challenges:

\section{Summary of Results}
\label{sec:summary-results-1}

\begin{enumerate}
\item We propose to introduces new programming abstractions by which a developer
  expresses \emph{what} adaptations are available. Importantly, developers do
  not have to specify exactly when and how different adaptations are to be used
  which is instead left to the underlying framework.

\item Rather than rely on manual policies, we can build tools that automatically
  \emph{learns} adaptation policy that are Pareto optimal. This learning process
  can be both offline and online, exploiting parallelism, sampling or
  statistical approaches to efficiently explore the configuration space.

\item We also demonstrate how such strategy can be employed in running systems
  and adapt to resource changes. For example, \sysname{} matches the streaming
  data rate to the measured available bandwidth. Upon encountering network
  congestion, it increases the degradation level to reduce the data rate, such
  that no persistent queue builds up. To recover, it progressively decreases the
  degradation level after probing for more available bandwidth. Our experiments
  show that \sysname{} achieves sub-second latency with only nominal accuracy
  drop (2-6\%).
\end{enumerate}

\section{Thesis Organization}
\label{sec:thesis-organization}

The remainder of this thesis is organized as follows:

\begin{itemize}[topsep=5pt]
\item \autoref{cha:background} covers the background for swarm applications. I
  first summarize the landscape of the emerging swarm applications and show that
  they have fundamental differences from previous related concepts.  Many swarm
  applications are constructed using a cloud-centric approach. I then argue
  against it by discussing the pitfalls including security, privacy,
  scalability, latency, etc. A new tier of computing infrastructure, the edge,
  arises to accompany the cloud. While it reduces network latency and provides
  more resource, the edge has its own challenges, such as increased
  heterogeneity. The swarm landscape and the argument against the cloud are
  based on joint work with Nitesh Mor, John Kolb, Douglas S. Chan, Nikhil Goyal,
  Ken Lutz, Eric Allman, John Wawrzynek, Edward A. Lee, and John
  Kubiatowicz~\cite{zhang2015cloud}.
\item In \autoref{cha:netw-reso-adapt}, I present adapting swarm applications to
  network resources. Many swarm applications that transport large streams of
  data across the wide area faces challenges with the scarce and variable
  bandwidth. This chapter focus on \awstream{} that integrates application
  adaptation as a first-class programming abstraction and automatically learns
  an accurate profile that models accuracy and bandwidth trade-off. Using the
  profile to guide application adaptation at runtime, we demonstrate that
  \awstream{} achieves sub-second latency with only nominal accuracy drop
  (2-6\%).  The chapter is based on joint work with Xin Jin, Sylvia Ratnasamy,
  John Wawrzynek, and Edward A. Lee~\cite{zhang2018awstream}.
\item In \autoref{cha:comp-reso-adapt}, I present adapting swarm applications to
  compute resources. Due to the heterogeneous capabilities of end devices and
  variable network/serving latency, it is challenging to provide a consistent
  bounded response times for swarm applications. I propose to build a
  performance model that characterizes accuracy and processing time trade-off to
  guide the execution. This chapter focuses on efficient profiling: using
  Bayesian Optimization (BO) to address the large parameter space and profile
  transfer to address heterogeneous capabilities of different devices.
\item Chapter 5 discusses related research and industrial efforts.
\item Finally, I conclude this thesis and identify important research directions
  for future work.
\end{itemize}

The research presented in this thesis is supported in part by Berkeley
Ubiquitous SwarmLab~\cite{swarmlab} and the TerraSwarm Research
Center~\cite{terraswarm}, one of six centers supported by the STARnet phase of
the Focus Center Research Program (FCRP) a Semiconductor Research Corporation
program sponsored by MARCO and DARPA.



\end{document}

%%% Local Variables:
%%% mode: latex
%%% TeX-master: t
%%% End:

\newpage
\section{Applications and Challenges}
\label{sec:motivation}

We focus on computation-heavy ML applications that is beyond the capability of
local devices. Offer some concrete numbers here. Mention that many approaches in
application domains are improving accuracy at a cost of increased
computation. And the spectrum of accuracy-cost is common in these applications.

Also describe application benchmark data set here.

For Face, we use FDDB dataset.

\subsection{Challenges}
\label{sec:challenges}

Motivated by the above applications, we outline the key challenges of exploiting
accuracy-cost trade-off for prediction serving and describe how \sysname{}
addresses these challenges.

\subsubsection*{Heterogeneous Environment}

% \begin{figure}[t]
%   \centering
%   \includegraphics[width=\columnwidth]{figures/redundancy.pdf}
%   \caption{Redundant requests in \sysname{}}
%   \label{fig:redundant}
% \end{figure}

\begin{table*}
  \centering
  \begin{tabular}{c c c c c c}
    \toprule
    Task          & RPi              & Mac             & Swarmbox        & Workstation    & GPU \\
    \midrule
    Encode (JPEG) & 19.6 $\pm$ 2.6   & 4.3 $\pm$ 1.0   & 5.2 $\pm$ 0.9   & 1.3 $\pm$ 0.3  & -   \\
    Decode (JPEG) & 5.4 $\pm$ 0.9    & 1.0 $\pm$ 0.3   & 0.9 $\pm$ 0.2   & 0.5 $\pm$ 0.3  & -   \\
    Viola Jones   & 343.4 $\pm$ 69.4 & 40.0 $\pm$ 10.7 & 48.2 $\pm$ 10.6 & 26.4 $\pm$ 5.7 &     \\
    \bottomrule
  \end{tabular}
  \caption{Performance of analytics operations (serialization, deserialization,
    network transmit, server processing).}
  \label{tab:perf-motiv}
\end{table*}

Our target application environment consists of machines with large range of
computing resources. $(i)$ End-devices, like mobile phones or IoT platforms, are
significantly limited in their computing power. Performing ML inference often
take seconds to complete. $(ii)$ Edge and Cloud. Both the edge and the cloud
suffers from variable latency, unstable connection, and service contention to
provide consistent response times, especially for 99\% requests.

There is a dizzying array of platforms ranging from \$5 Raspberry Pi to \$1000+
GPU-powered workstation~\cite{zhang2015cloud}.  Even in the cloud, there are
various VM options in the cloud: companies rent VMs based on budget or because
of a lack of expertise.

\textbf{Solution:} Because of the such heterogeneity, we hypothesis an ensemble
of available resources (\autoref{fig:dr}) can overcome the shortcomings of
individual platforms and offer end-users with bounded response times, similar to
prior solutions in both cloud-offloading (Tango~\cite{gordon2015accelerating})
and straggler mitigation in the cloud
(Dolly~\cite{ananthanarayanan2013effective}).

\subsubsection*{Network Variations}

\begin{figure}[t]
  \begin{subfigure}[t]{0.49\columnwidth}
    \centering
    \includegraphics[width=\textwidth]{figures/fcc_latency.pdf}
    \caption{Network latency variation.}
    \label{fig:fcc-latency}
  \end{subfigure}
  \hfill
  \begin{subfigure}[t]{0.49\columnwidth}
    \centering
    \includegraphics[width=\textwidth]{figures/tf_latency.pdf}
    \caption{Service latency variation.}
    \label{fig:tf-latency}
  \end{subfigure}
  \caption{Network latency increases during downstream and upstream speed tests
    (left). Service time increases during load increase (right).}
\end{figure}

Using FCC broadband measurements, we validate the large variation in wide area
network~(\autoref{fig:fcc-latency}). The median network delay increases from
\SI{22}{\ms} to \SI{80}{\ms} under downstream load and \SI{272}{\ms} under
upstream load.

\textbf{Solution:} Client side, redundant as described above. Server side, time
synchronization and update SLO; and early rejection.

\subsubsection*{Service/Workload Variation}

\noindent We measure TensorFlow serving's performance with different level of load and
validate the large variation in prediction serving
systems~(\autoref{fig:tf-latency}). With modest 1K load, the p99.9 latency
increases from \SI{3.5}{\ms} to \SI{22.5}{\ms}. With 5K load, even the median
latency increases to \SI{21.5}{\ms}: a 22.4$\times$ increase from \SI{0.96}{\ms}
with no load.

\textbf{Solution:} SLO-aware scheduling and the ability to reject requests.

\subsubsection*{Complex Performance Model}

\begin{figure}[t]
  \centering
  \includegraphics[width=.9\columnwidth]{figures/tradeoff-cnn.pdf}
  \caption{Accuracy-cost trade-off for different CNNs.}
  \label{fig:tradeoff-cnn}
\end{figure}

\begin{figure}[t]
  \centering
  \includegraphics[width=.9\columnwidth]{figures/exhaustive-face.pdf}
  \caption{Complex performance model: spanning multiple dimensions and
    exhibiting non-linear relationship.}
  \label{fig:complex-perf-model}
\end{figure}

For many ML inference task, there exist more than one algorithm, or tunable
parameters for each algorithm with different accuracy and processing times.  We
can speed up computation by providing a less accurate response. This would allow
some computation tractable on end devices and handling more requests on the
edge/cloud.

For example, accuracy-cost trade-offs for object detection using convolutional
neural network (CNN)~\cite{huang2016speed}. \autoref{fig:tradeoff-cnn} shows one
such benchmark~\cite{cnn.benchmarks}.

Many algorithms have large number of knobs to tune that will affect accuracy and
processing cost. We use Viola-Jones (VJ) cascade face
detector~\cite{viola2001rapid} as an example. \autoref{fig:complex-perf-model}
shows the large parameter space with respect to three parameters:
\texttt{min\_size}, \texttt{min\_neighbors}, and \texttt{scale}.

ML algorithms have many tunable parameters. For many algorithms, processing
times and the accuracy may exhibit \textit{non-linear} behavior with respect to
the parameters.

\textbf{Solution:} Bayesian Optimization.

% \begin{table}
%   \small
%   \centering
%   \begin{tabular}{c c c c}
%       \toprule
%       Algorithm & Mobile & Server & GPU \\
%       \midrule
%       VJ Face & $2263.71$ & $197.77 \pm 10.56$ & $23.44 \pm 5.23$ \\
%       HOG + SVM & 1987.9 & 59.7 & 20 \\
%       CNN & $800$ & $300$ & $40$ \\
%       \bottomrule
%     \end{tabular}
%    \caption{Processing Times for Example Model Serving on different
%      platforms.\protect\footnotemark}
%    \label{tab:times}
% \end{table}

% \footnotetext{Mobile is Android Nexus 7; Server is Intel Core XXX; GPU is GTX
%   970. Data is averaged over 100 frames with resolution $640 \times 480$.}

%%% Local Variables:
%%% mode: latex
%%% TeX-master: "../serving"
%%% End:
\section{\awstream{} Design}
\label{sec:system}

To address the issues with manual policies or application-specific
optimizations, \awstream{} structures adaptation as a set of approximate,
modular, and extensible specifications (\autoref{sec:structure-adapt}). The
well-defined structure allows us to build a generic profiling tool that learns
an accurate relationship---we call it the profile---between bandwidth
consumption and application accuracy (\autoref{sec:automatic-profiling}). The
profile then guides the runtime to react with precision: achieving low latency
and high accuracy when facing insufficient bandwidth
(\autoref{sec:runtime}). \autoref{fig:overview} shows the high-level overview of
\awstream{}.

\begin{figure}
  \centering
  \includegraphics[width=0.8\linewidth]{figures/system.pdf}
  \caption{\todo{update section number}. \awstream{}'s phases: development,
    profiling, and runtime. \awstream{} also manages wide-area deployment.}
  \label{fig:overview}
\end{figure}

\input{awstream/api}
\input{awstream/profiling}
\input{awstream/runtime}

%%% Local Variables:
%%% mode: latex
%%% TeX-master: "../network"
%%% End:

\newpage

\section{Implementation}
\label{sec:implementation}

We implement BO-based performance modeling with Spearmint package. Our runtime
client and server communicate through an RPC interface. Currently, we have
implemented face detection, pedestrian detection, and object detection.

\begin{table}
  \footnotesize
  \centering
  \begin{tabular}{c c c c}
    \toprule
    Application & Algorithms & Dataset \\
    \midrule
    Face & Viola-Jones, LBP & FDDB \\
    \midrule
    Object & ResNet, Inception, AlexNet & ImageNet \\
    \bottomrule
  \end{tabular}
  \caption{\sysname{} Applications}
  \label{tab:apps}
\end{table}

\newpage

Details about how we implement the applications.

\newpage

%%% Local Variables:
%%% mode: latex
%%% TeX-master: "../serving"
%%% End:

\section{Evaluation}
\label{sec:evaluation}

\begin{itemize}[itemsep=0pt, topsep=1pt]
\item[\autoref{sec:perf-modeling}] Compared with naive approaches (random and
  gradient-based search), \sysname{} finds a Pareto front that is closer to the
  optimal (\autoref{fig:bo}).
\item[\autoref{sec:runtime}] For all applications, \sysname{} achieves bounded
  service latency across various network conditions (\autoref{fig:end-to-end}).
\end{itemize}

\subsection{Performance Modeling}
\label{sec:perf-modeling}

This section shows that BO can effectively explore the large design space and
come up with a better Pareto-optimal parameter set. We compare \sysname{} with
two baseline profilers: $(i)$ greedy approach as described in
VideoStorm~\cite{zhang2017live}; $(ii)$ random sampling.

Following prior work~\cite{zhang2017live}, several tricks can enhance this
search process. To avoid starting with an expensive configuration and exploring
its neighbors, (which are also likely to be expensive, thus wasting CPU), we
pick k random configurations and start from the one with the highest X(c). We
found that using even k = 3 can successfully avoid starting in an expensive part
of the search space. Second, we cache intermediate results in the query’s DAG
and reuse them in evaluating configura- tions with overlapping knob values. For
each dimen- sion, we could fix the other dimensions and search for the cheapest
configuration possible. This could lead to suboptimal decisions if for example,
because of bad ap- plication configuration a dimension is not fully explored or
there are local minima in the problem space.


Greedy converts MOO into SOO with a parameter $X = \beta$ $A - \beta T$. It
starts with a random configuration $c$ and then picks a neighbor configuration
(by changing the value of a random dimension). If the new configuration $c'$
returns a higher $X$, it updates and iterate form $c'$ again. Otherwise, it
picks a different neighbor $c''$ by changing another dimension. It starts with
three random starting point; it also evaluate a number of random $\beta$.

\autoref{fig:bo} shows that with the same budget (the number of parameters to
evaluate), our profiler can find parameters with better trade-offs between
application accuracy and processing times.

\begin{figure}
  \centering
  \includegraphics[width=0.95\columnwidth]{figures/profiling.pdf}
  \caption{Using Face as an example, BO evaluates 50 configurations and
    recommends 29 configurations as the Pareto-optimal boundary (the blue
    line). Greedy and Random find sub-optimal Pareto configurations with a
    budget of 80 evaluations (the yellow line in each figure).}
  \label{fig:bo}
\end{figure}

\subsection{Profile Transfer.}

Profile transfer without re-running the entire BO. The ``Pareto-optimal'' is
horizontally stretched/compressed.

\begin{figure}
  \centering
  \includegraphics[width=0.95\linewidth]{figures/serving-cross-platform.pdf}
  \caption{(Left) Empirically, processing times follows a linear
    approximation. (Right) Stretched/compressed profile. See paper for
    details.}
\end{figure}

%%% Local Variables:
%%% mode: latex
%%% TeX-master: "../compute"
%%% End:

\input{awstream/hls}
\subsection{JetStream++}
\label{sub:jetstream++}

We modified the open source version of
JetStream\footnote{\url{https://github.com/princeton-sns/jetstream/}, \newline
  commit bf0931b2d74d20fdf891669188feb84c96AF84.} in order to use our profile as
its manual policy. Because JetStream doesn't support simultaneous degradation
across multiple operators, we implemented a simple \texttt{VideoSource} operator
that understands how to change image resolutions, frame rate, and video encoding
quantization. At runtime, \texttt{VideoSource} queries congestion policy manager
and adjusts three dimensions simultaneously. This operator is then exposed to
the Python-implemented control plane. We call this modified version
JetStream++.\footnote{\url{https://github.com/awstream/jetstream-clone/pull/1}}

JetStream's code base is modular and extensible: the modifications include 53
lines for the header file, 171 lines for implementation, 75 lines for unit test,
and 49 lines of python as the application. While extending JetStream with our
profile is not challenging, JetStream++ performs degradation in a single
operator and loses the composability. We could modify JetStream to support
degradation across multiple operators, but that would require substantial
changes to JetStream. Using JetStream++ with our profile, the comparison is
enough to illustrate the difference between \sysname{}'s and JetStream's
runtime.

%%% Local Variables:
%%% mode: latex
%%% TeX-master: "../network"
%%% End:

%% LocalWords: Mbps analytics runtime JetStream OpenCV YOLO pre GStreamer appsrc
%% LocalWords: appsink zerolatency quantization dataset SVM geo topk VideoSource
%% LocalWords: JetStream's composability TK TCP UDP HLS FFmpeg bitrates nginx
%% LocalWords: packetization TPUT topk NodeJS metadata timestamp Mbps Kbps
%% LocalWords: aws PD's TK's hls js VoD awstream tk fdf feb

\input{awstream/discussion}
\documentclass[thesis.tex]{subfiles}

\begin{document}

\chapter{Conclusion and Future Work}
\label{cha:concl-future-work}

This thesis claims that \textit{systematic adaptation and quantitative profiling
  are the key to a resilient swarm}. In this thesis, I have studied swarm
applications and demonstrated how to allow adaptation for network and compute
resources. This chapter summarizes the contributions of this thesis and
discusses valuable directions for future work.

\section{Conclusions and Contributions}
\label{sec:contributions}

\paraf{Reviewing the Emerging Swarm.} This thesis provides an overview of
emerging swarm and the architecture design trends for swarm applications. We
discuss issues with existing approaches that directly connect devices to the
cloud. We also analyze the challenges swarm applications face, with a focus on
constrained resources and heterogeneous environment.

\para{Systematic and Quantitative Adaptation.} The core of this thesis is a
systematic and quantitative approach for developing adaptive swarm
applications. Instead of relying on manual policies, developers use well-defined
programming abstractions to express adaptation options. We then use a
data-driven profiling to automatically learn the profile---a set of
Pareto-optimal configurations---that characterizes resource demands and
application performance.

\para{Design, Implementation, and Evaluation of \awstream{}.} Swarm applications
that communicate across the wide area need to handle the scarce and variable WAN
bandwidth. We present a complete design and implementation of the framework
\awstream{} for network resource adaptation. We have developed several swarm
applications: pedestrian detection, augmented reality, and monitoring log
analysis. Our experiments show that all applications can achieve sub-second
latency with nominal accuracy drops.

\para{Improving Profiling Efficiency.} In \awstream{}, we demonstrated how
parallelism and sampling techniques can improve profiling efficiency. For
network resource adaptation, we have improved the efficiency further to address
two challenges: Bayesian Optimization to address the large parameter space;
profile transfer to address device heterogeneity.

\section{Future Directions}
\label{sec:future-directions}

In previous chapters, we have discussed some potential improvements to our
existing work. In this section, we discuss future directions in a broader
context.

The core challenge for the swarm comes from the scale of inter-connected
devices. As the number continues to grow, we expect to see more diverse swarm
applications. While we have used several applications in this thesis as case
studies, we need more applications to evaluate how general the proposed APIs
are, how effective the profiling techniques are, and how responsive the runtime
system is.

With our framework, developers need to encode adaptation with our APIs, provide
training data, and implement accuracy functions. There could be better tooling
and library of reusable functions (such as F1 score for object detection) to
further simplify developer effort, similar to machine learning libraries.

Instead of adapting to resources, swarm applications should be able to
dynamically recruit resources, such as sensors, actuators, data, and computing
infrastructure. What are the necessary technologies to realize this recruitment?
Discovery, management. One promising work is the mutable accessors discussed by
Brooks et al.~\cite{brooks2018component}. An mutable accessor is an abstract
interface specification for candidate accessors. It reifies a concrete accessor
downloaded from the Internet or retrieved through a discovery mechanism. This
reification serves as recruiting resources for the swarm applications.

\end{document}

%%% Local Variables:
%%% mode: latex
%%% TeX-master: t
%%% End:


\end{document}

%%% Local Variables:
%%% mode: latex
%%% TeX-master: t
%%% End:
