\documentclass[thesis.tex]{subfiles}

\begin{document}

\chapter{Introduction}

Over the past two decades, we have seen a growing number of networked sensors
and actuators installed in our connected world. These sensors and actuators
offer an unprecedented ability to monitor and act. Because of the enormous
potentials in solving societal-scale problems, this trend has gained significant
attraction, as demonstrated by its many names: Internet of Things
(IoT)~\cite{atzori2010internet}, Internet of Everything
(IoE)~\cite{bradley2013internet}, Industry 4.0~\cite{lasi2014industry}, The
Industrial Internet~\cite{eigner2018industrial}, TSensors (Trillion
Sensors)~\cite{bogue2014towards}, Machine to Machine
(M2M)~\cite{anton2014machine}, Smarter Planet~\cite{palmisano2008smarter}, etc.
In this thesis, we use the term ``swarm''---as called by Jan Rabaey in a keynote
talk at ASPDAC 2008~\cite{rabaey2008brand}---to refer to the vast collection of
networked sensors and actuators.

Swarm applications transport, distill, and process large streams of data across
the wide area in real time. For example, large cities such as London and Beijing
have deployed millions of cameras for surveillance and traffic
control~\cite{skynet, london.surveillance}. Buildings are increasingly equipped
with a wide variety of sensors to improve energy efficiency and occupant
comfort~\cite{krioukov2012building}.

Existing approach of directly connecting to the cloud and issues.

Fundamental difference and challenges by the swarm. Swarm systems rely on vast
numbers of heterogeneous sensors that are generating massive amounts of
data. (2) Swarms and swarmlets that dynamically recruit resources will compete
for those resources. (3) Heterogeneous: swarm components are of various types,
requiring interfacing and interoperability across multiple platforms and models
of computation.

This thesis argues that by providing adaptation as a core abstraction in
application development, we can allow swarm applications with better
performance. This thesis offers two systems with complete design, implementation
and evaluation regarding network and compute resource,
respectively.

\vspace{1em}

\noindent\textbf{Thesis Statement}: \textit{Intelligence is the ability to adapt to
  changes.}

\vspace{1em}

\section{Challenges with Existing Approaches}
\label{sec:chall-with-exist}

\para{Limited Resource.}

\para{Ad-hoc Development.}

\para{Huge Design Space.}

\para{Heterogeneous Platforms.}

\section{Adaptation as a First-class Citizen}
\label{sec:adaptation}

When facing situations where the bandwidth is not sufficient, applications
deployed today either choose a conservative setting (e.g.\,only delivering 360p
videos) or leave their fate to the underlying transport layer: (1) in the case
of TCP, the sender will be blocked and data are backlogged, leading to severe
delay; (2) in the case of UDP, uncontrolled packet loss occurs, leading to
application performance drop. Instead of ``suffering'' from a degraded network,
applications can act proactively by adjusting their behavior: reducing the data
rate to ensure that important data are delivered in time.

\para{API.}

\para{Automatic Profiling and Optimization.}

\para{Runtime Adaptation.}

\section{Summary of Results}
\label{sec:summary-results-1}

\begin{enumerate}
\item \sysname{} introduces new programming abstractions by which a developer
  expresses \emph{what} degradation functions can be used by the framework.
  Importantly, developers do not have to specify exactly when and how different
  degradation functions are to be used which is instead left to the \sysname{}
  framework.

\item Rather than rely on manual policies, \sysname{} automatically
  \emph{learns} a Pareto-optimal policy or strategy for when and how to invoke
  different degradation functions.  For this, we design a methodology that uses
  a combination of offline and online training to build an accurate model of the
  relationship between an application's accuracy and its bandwidth consumption
  under different combinations of degradation functions. Our solution exploits
  parallelism and sampling to efficiently explore the configuration space and
  learn an optimal strategy.

\item \sysname{}'s final contribution is the design and implementation of a
  runtime system that continually measures and adapts to network conditions.
  \sysname{} matches the streaming data rate to the measured available
  bandwidth, and achieves high accuracy by using the learned Pareto-optimal
  configurations.  Upon encountering network congestion, our adaptation
  algorithm increases the degradation level to reduce the data rate, such that
  no persistent queue builds up. To recover, it progressively decreases the
  degradation level after probing for more available bandwidth.
\end{enumerate}

\section{Thesis Organization}
\label{sec:thesis-organization}

The remainder of this thesis is organized as follows:

\begin{itemize}
\item In Chapter 2, I first describe the landscape development for swarm
  applications.
\item In Chapter 3, I present the adaptation on network resources.
\item In Chapter 4, I present the adaptation on compute resources.
\item Chapter 5 discusses related research and industrial efforts related to
  adaptation.
\item Finally, I conclude this thesis and identify important research directions
  for future work.
\end{itemize}

\end{document}

%%% Local Variables:
%%% mode: latex
%%% TeX-master: t
%%% End:
