\documentclass[thesis.tex]{subfiles}

\begin{document}

\chapter{Introduction}

Over the past two decades, we have seen a growing number of networked sensors
and actuators installed in our connected world. These sensors and actuators
offer an unprecedented ability to monitor and act. Because of the enormous
potential in solving societal-scale problems, this trend has gained significant
attraction, as demonstrated by many parallel efforts: Internet of Things
(IoT)~\cite{atzori2010internet}, Internet of Everything
(IoE)~\cite{bradley2013internet}, Industry 4.0~\cite{lasi2014industry}, The
Industrial Internet~\cite{eigner2018industrial}, TSensors (Trillion
Sensors)~\cite{bogue2014towards}, Machine to Machine
(M2M)~\cite{anton2014machine}, Smarter Planet~\cite{palmisano2008smarter},
etc. In this thesis, we refer to this trend as \textit{the swarm} because it
well characterizes where the potential lies: the number and the scale of
interconnected devices.

Swarm applications generate, transport, distill, and process large streams of
data across the wide area, often in real time. For example, large cities such as
London and Beijing have deployed millions of cameras for surveillance and
traffic control~\cite{skynet, london.surveillance}. Buildings are increasingly
equipped with a wide variety of sensors to improve energy efficiency and
occupant comfort~\cite{dawson2010smap, krioukov2012building}.

With billions, potentially trillions~\cite{middleton2013forecast}, of devices
interconnected, the swarm will become an enormous challenge to the underlying
communication, computation, and storage infrastructure. For example, many swarm
applications have adopted the cloud as a universal computation resource and a
storage backend~\cite{carriots, sami, gupta2014bolt, zachariah1001internet}
because of the unlimited-resource illusion, economic benefits, and simplified
management. However, transmitting data from sensors to the cloud requires
traversing the wide-area network (WAN), which has scarce and variable
bandwidth. Edge infrastructure---the fog~\cite{bonomi2012fog, bar2013fog},
cloudlet~\cite{ha2014towards, satyanarayanan2009case, chen2018application}, and
swarmbox---can accompany the cloud by its reduced communication distances and
extra resources. However, edge devices are extremely heterogeneous: some have
limited processing capabilities while others can be overloaded by service
requests.

The success of the swarm relies on addressing the gap between the growing
application demand and scarce/variable resources. Facing insufficient resources,
swarm applications that do not adapt will suffer. Take network resources as an
example, when there is not enough bandwidth, an application may incur large
latency if it uses TCP that causes backlogged data, or devastated data quality
if it uses UDP that causes uncontrollable packet loss. These are extreme design
points in the trade-off space between data fidelity and data freshness
(\autoref{fig:intro}). In order to be resilient to resource availability and
changes, swarm applications can adapt its behaviors and hence its resource
demand, just as video streaming dynamically adjusts its bitrate to the available
bandwidth~\cite{michalos2012dynamic}.

While enabling adaptation is easy, it is challenging to come up with accurate
adaptation policies that governs the application behavior in an optimal
way. Developers often quickly come up with some heuristics. Although such manual
policies are already a step forward compared to non-adaptive applications, they
lack quantitative measurements to back up. Accurate adaptation policies often
require extensive domain expertise or considerable effort. In addition, at a
swarm scale, each application has its own trade-off space and it is not viable
to study each individual application extensively.

In this thesis, we propose to adapt swarm applications, in a systematic and
quantitative approach. By ``systematic'', I mean programming abstractions that
allow developers to embed adaptation options; by ``quantitative'', I mean an
automatic profiling tool that learns the adaptation strategy in a data-driven
approach.

\vspace{1em}

\noindent\textbf{Thesis Statement:} \textit{Systematic adaptation and
  quantitative profiling are the key to a resilient swarm.}

\vspace{1em}

In the remainder of this chapter, I first summarize the challenges with
developing swarm applications and elaborate our systematic and quantitative
adaptation. I then highlight our key results and layout the roadmap for the rest
of the thesis.

\section{Challenges with Developing Swarm Applications}
\label{sec:chall-with-exist}

We identify the following challenges that we need to address for a prosperity of
the swarm,

\begin{itemize}[topsep=5pt, itemsep=2pt]

\item \textbf{Constrained Resources.} The swarm, with its scale, creates a large
application demand that the infrastructure will fail to meet. In terms of
network resources, the WAN bandwidth is unable to keep up with the pace of data
generated by all different types of sensors. In terms of compute resources, end
devices, especially low-power microcontrollers, cannot fulfill many
computation-heavy tasks, e.g., machine learning (ML) inference.

\item \textbf{Heterogeneous Devices.} The swarm has a wide spectrum of devices with
heterogeneous computing capabilities, ranging from powerful computing units with
specialized resources (such as GPU/TPU) to low-power
microcontrollers. Applications that work on one platform will end up with
completely different performances on another platform.

\end{itemize}

A key approach to address the above mentioned issues is to employ adaptation. If
we can adapt the application behavior to match the available and varying
resource, swarm applications would become resilient. However, enabling
adaptation has the following challenges,

\begin{itemize}[topsep=5pt, itemsep=2pt]

\item \textbf{Large Design Space.} Swarm applications often have multiple
  adjustable stages or use a complex algorithm with many tunable
  parameters. While these adjustable stages or tunable parameters provide
  flexibility in constructing swarm applications, when developing adaptation
  strategies, they form a large combinatorial design space.

\item \textbf{Ad-hoc Development and Manual Efforts.} Developers often come up
  with manual policies based on heuristics. These policies are not accurate and
  lead to sub-optimal performance.

\end{itemize}

\section{Systematic and Quantitative Adaptation}
\label{sec:adaptation}

We propose to incorporate adaptation as a first-class programming abstraction
(systematic), automatically learn the adaptation strategy, the profile, with a
data-driven approach (quantitative). In this way, swarm applications can use the
profile to guide its adaptation at runtime. This approach has three stages,

\para{Programming Abstractions.} We propose to introduce programming
abstractions that are declarative instead of procedural. Developers do not need
to express exactly \emph{when} and \emph{how} the application will
adapt. Instead, they only specify \emph{what} adaptations are available. In this
thesis, we demonstrate two such programming abstractions: \maybe{} for stream
processing in network resource adaptation and macros for algorithm parameters in
compute resource adaptation.

\para{Data-driven Automatic Profiling.} Based on the available adaptation
options specified by developers, we can build tools that automatically
\emph{learns} Pareto-optimal adaptation policies. To handle the large design
space, we can use system techniques or statistical techniques. In this thesis,
we demonstrate how parallelism speeds up profiling for stream processing in
network resource adaptation and how statistical approaches, specifically
Bayesian Optimization, learns near-optimal sets with substantially few number of
samples.

\para{Runtime Adaptation.} Once the profile is learned, it is used in running
systems to guide the adaptation: matching application demand to available
resources. For example, \sysname{} matches the streaming data rate to the
measured available bandwidth. Upon encountering network congestion, it looks up
the profile for a configuration that maximizes the achievable application
accuracy while satisfying the network bandwidth constrain.

\section{Summary of Results}
\label{sec:summary-results-1}

We have demonstrated the systemic and quantitative approach towards adaptation
for network resources and compute resources. In both cases, we show our
programming abstractions and demonstrate that the data-driven approach generates
accurate profiles. For network resources, we use parallelism and sampling
techniques to speed up the profiling by up to 29$\times$ and 8.7$\times$
respectively. For compute resources, we have significantly reduced the number of
profiling points, by more than 50$\times$.

For \awstream{}, we have built three real-world applications and evaluated it
extensively by comparing with several baselines.  Our evaluation shows that
\awstream{} significantly outperforms non-adaptive applications: achieving a
40--100$\times$ reduction in packet delivery times relative to applications
built over TCP, or an over 45--88\% improvement in data fidelity (application
accuracy) relative to applications built over UDP. We also compare \awstream{}
to JetStream~\cite{rabkin2014aggregation}, a state-of-the-art system for
building adaptive streaming analytics that is based on manual policies. Our
results show that besides the benefit of generating optimal policies
\textit{automatically}, \awstream{} achieves a 15-20$\times$ reduction in
latency and 1-5\% improvement in accuracy simultaneously relative to JetStream.

\section{Thesis Organization}
\label{sec:thesis-organization}

The remainder of this thesis is organized as follows:

\begin{itemize}[topsep=5pt]
\item \autoref{cha:background} covers the background for swarm applications. I
  first summarize the landscape of the emerging swarm applications and show that
  they have fundamental differences from previous related concepts.  Many swarm
  applications are constructed using a cloud-centric approach. I then argue
  against it by discussing the pitfalls including security, privacy,
  scalability, latency, etc. A new tier of computing infrastructure, the edge,
  arises to accompany the cloud. While it reduces network latency and provides
  more resource, the edge has its own challenges, such as increased
  heterogeneity. The swarm landscape and the argument against the cloud are
  based on joint work with Nitesh Mor, John Kolb, Douglas S. Chan, Nikhil Goyal,
  Ken Lutz, Eric Allman, John Wawrzynek, Edward A. Lee, and John
  Kubiatowicz~\cite{zhang2015cloud}.
\item In \autoref{cha:netw-reso-adapt}, I present adapting swarm applications to
  network resources. Many swarm applications that transport large streams of
  data across the wide area faces challenges with the scarce and variable
  bandwidth. This chapter focus on \awstream{} that integrates application
  adaptation as a first-class programming abstraction and automatically learns
  an accurate profile that models accuracy and bandwidth trade-off. Using the
  profile to guide application adaptation at runtime, we demonstrate that
  \awstream{} achieves sub-second latency with only nominal accuracy drop
  (2-6\%).  The chapter is based on joint work with Xin Jin, Sylvia Ratnasamy,
  John Wawrzynek, and Edward A. Lee~\cite{zhang2018awstream}.
\item In \autoref{cha:comp-reso-adapt}, I present adapting swarm applications to
  compute resources. Due to the heterogeneous capabilities of end devices and
  variable network/serving latency, it is challenging to provide a consistent
  bounded response times for swarm applications. I propose to build a
  performance model that characterizes accuracy and processing time trade-off to
  guide the execution. This chapter focuses on efficient profiling: using
  Bayesian Optimization (BO) to address the large parameter space and profile
  transfer to address heterogeneous capabilities of different devices.
\item Chapter 5 discusses related research and industrial efforts.
\item Finally, I conclude this thesis and identify important research directions
  for future work.
\end{itemize}

The research presented in this thesis is supported in part by Berkeley
Ubiquitous SwarmLab~\cite{swarmlab} and the TerraSwarm Research
Center~\cite{terraswarm}, one of six centers supported by the STARnet phase of
the Focus Center Research Program (FCRP) a Semiconductor Research Corporation
program sponsored by MARCO and DARPA.



\end{document}

%%% Local Variables:
%%% mode: latex
%%% TeX-master: t
%%% End:
