\documentclass[thesis.tex]{subfiles}

\begin{document}

\chapter{Introduction}

Over the past two decades, we have seen a growing number of networked sensors
and actuators installed in our connected world. These sensors and actuators
offer an unprecedented ability to monitor and act. Because of the enormous
potential in solving societal-scale problems, this trend has gained significant
attention, as demonstrated by many parallel efforts: Internet of Things
(IoT)~\cite{atzori2010internet}, Internet of Everything
(IoE)~\cite{bradley2013internet}, Industry 4.0~\cite{lasi2014industry}, The
Industrial Internet~\cite{eigner2018industrial}, TSensors (Trillion
Sensors)~\cite{bogue2014towards}, Machine to Machine
(M2M)~\cite{anton2014machine}, Smarter Planet~\cite{palmisano2008smarter},
etc. In this thesis, we refer to this trend as \textit{the swarm} because it
well characterizes where the potential lies: the number and the scale of
interconnected devices.

Swarm applications generate, transport, distill, and process large streams of
data across the wide area, often in real time. For example, large cities such as
London and Beijing have deployed millions of cameras for surveillance and
traffic control~\cite{skynet, london.surveillance}. Buildings are increasingly
equipped with a wide variety of sensors to improve energy efficiency and
occupant comfort~\cite{dawson2010smap, krioukov2012building}.

With billions, potentially trillions~\cite{middleton2013forecast}, of devices
interconnected, the swarm will become an enormous challenge to the underlying
communication, computation, and storage infrastructure. For example, many swarm
applications have adopted the cloud as a universal computation resource and
storage backend~\cite{carriots, sami, gupta2014bolt, zachariah1001internet}
because of its elasticity, economic benefits, and simplified
management. However, transmitting data from sensors to the cloud requires
traversing the wide-area network (WAN), which has scarce and variable
bandwidth. Edge infrastructure---the fog~\cite{bonomi2012fog, bar2013fog},
cloudlet~\cite{ha2014towards, satyanarayanan2009case, chen2018application}, and
swarmbox---can accompany the cloud as it reduces communication distances and
provides extra resources. However, edge devices are extremely heterogeneous:
some have limited processing capabilities while others can be overloaded by
service requests.

The success of the swarm relies on addressing the gap between the growing
application demand and scarce/variable resources. Facing insufficient resources,
swarm applications that do not adapt will suffer. Take network resources as an
example; when there is not enough bandwidth, an application may incur large
latency that causes backlogged data if it uses TCP, or devastated data quality
that has uncontrollable packet loss if it uses UDP. These are extreme design
points in the trade-off space between data fidelity and data freshness (as we
will show in \autoref{fig:intro}). In order to be resilient to resource
availability and changes, swarm applications can adapt their behavior and change
their resource demand, just as video streaming dynamically adjusts its bitrate
to the available bandwidth~\cite{michalos2012dynamic}.

While enabling adaptation is easy, it is challenging to come up with accurate
adaptation policies that govern the application behavior in an optimal
way. Developers often quickly develop manual policies based on heuristics. While
such manual policies are already a step forward compared to non-adaptive
applications, they lack quantitative measurements to back up the
decisions. Accurate adaptation policies often require extensive domain expertise
or considerable effort. In addition, each application has its own trade-off. At
a swarm scale, it is not viable to study each individual application
extensively.

In this thesis, I propose to adapt swarm applications in a systematic and
quantitative way. By ``systematic'', I mean programming abstractions that
allow developers to embed adaptation options; by ``quantitative'', I am refering
to a data-driven automatic profiling tool that learns the adaptation strategy.

\vspace{1em}

\noindent\textbf{Thesis Statement:} \textit{Systematic adaptation and
  quantitative profiling are the key to a resilient swarm.}

\vspace{1em}

In the remainder of this chapter, I first summarize the challenges with
developing swarm applications and elaborate our systematic and quantitative
adaptation. I then highlight our key results and lay out the roadmap for the rest
of the thesis.

\section{Challenges with Developing Swarm Applications}
\label{sec:chall-with-exist}

We identify the following challenges that we need to address for a prosperity of
the swarm,

\begin{itemize}[topsep=5pt, itemsep=2pt, leftmargin=10pt]

\item \textbf{Constrained Resources.} The swarm, with its scale, creates a large
  application demand that the infrastructure will fail to meet. In terms of
  network resources, the WAN bandwidth is unable to keep up with the pace of
  data generated by all different types of sensors. In terms of compute
  resources, end devices, especially low-power microcontrollers, cannot fulfill
  many computation-heavy tasks, e.g., machine learning (ML) inference.

\item \textbf{Heterogeneous Devices.} The swarm has a wide spectrum of devices
  with heterogeneous computing capabilities, ranging from powerful computing
  units with specialized resources (such as GPU/TPU) to low-power
  microcontrollers. Applications that work on one platform will end up with a
  completely different performance on another platform.

\end{itemize}

A key approach to address the above challenges is to employ adaptation, i.e.,
adapting the application behavior to match the available and varying
resource. However, to support adaptation, we need to address the following
issues:

\begin{itemize}[topsep=5pt, itemsep=2pt, leftmargin=10pt]

\item \textbf{Large Design Space.} Swarm applications often have multiple
  adjustable stages or use a complex algorithm with many tunable
  parameters. While these adjustable stages or tunable parameters provide
  flexibility in constructing swarm applications, when developing adaptation
  strategies, they form a large combinatorial design space.

\item \textbf{Ad-hoc Development and Manual Efforts.} Existing approaches that
  allow adaptation require developers writing manual policies, often based on
  heuristics rather than quantitative analysis. These policies are not accurate,
  prone to human errors, and they lead to sub-optimal performance.

\end{itemize}

\section{Systematic and Quantitative Adaptation}
\label{sec:adaptation}

We propose to incorporate adaptation as first-class programming abstractions
(systematic) and automatically learn the adaptation strategy, the profile, with
a data-driven approach (quantitative). In this way, swarm applications can use
the profile to guide its adaptation at runtime. This approach has three stages,

\para{Programming Abstractions.} We propose to introduce programming
abstractions that are declarative instead of procedural. Developers do not need
to express exactly \emph{when} and \emph{how} the application will
adapt. Instead, they only specify \emph{what} adaptations are available. In this
thesis, we demonstrate two such programming abstractions: \maybe{} for stream
processing in network resource adaptation and macro annotations for algorithm
parameters in compute resource adaptation.

\para{Data-driven Automatic Profiling.} Based on the available adaptation
options specified by developers, we can build tools that automatically
\emph{learn} Pareto-optimal adaptation policies. To handle the large design
space, we use both system techniques and statistical techniques. In this thesis,
we demonstrate how parallelism and sampling speed up profiling for stream
processing in network resource adaptation and how Bayesian Optimization learns
near-optimal sets with substantially fewer number of samples.

\para{Runtime Adaptation.} Once the profile is learned, it is used at
\emph{runtime} to guide application adaptation, matching application demand to
available resources. For example, \awstream{} matches the streaming data rate to
the estimated available bandwidth. By avoiding congestion with a Pareto-optimal
configuration from the profile, \awstream{} achieves sub-second latency while
maximizing the application accuracy.

\section{Summary of Results and Contributions}
\label{sec:contributions}

In this thesis, we demonstrate a systemic and quantitative approach towards
adaptation to network resources and compute resources. This thesis makes the
following contributions:

\paraf{Reviewing the Emerging Swarm.} This thesis provides an overview of
emerging swarm and the architecture design trends for swarm applications. We
discuss issues with existing approaches that directly connect devices to the
cloud. We also analyze the challenges swarm applications face, with a focus on
constrained resources and heterogeneous environment.

\para{Systematic and Quantitative Adaptation.} The core of this thesis is a
systematic and quantitative approach for developing adaptive swarm
applications. Instead of relying on manual policies, developers use well-defined
programming abstractions to express adaptation options. We then use a
data-driven profiling to automatically learn the profile---a set of
Pareto-optimal configurations---that characterizes resource demands and
application performance.

\para{Design, Implementation, and Evaluation of \awstream{}.} Swarm applications
that communicate across the wide area need to handle the scarce and variable WAN
bandwidth. We present a complete design and implementation of the framework
\awstream{} for network resource adaptation. We have developed several swarm
applications: pedestrian detection, augmented reality, and monitoring log
analysis. Compared with non-adaptive applications, \awstream{} achieves a
40--100$\times$ reduction in latency relative to applications using TCP, and a
45--88\% improvement in application accuracy relative to applications using
UDP. Compared with manual policies using a state-of-the-art system
JetStream~\cite{rabkin2014aggregation}, \awstream{} achieves a 15-20$\times$
reduction in latency and 1-5\% improvement in accuracy simultaneously.

\para{Improving Profiling Efficiency.} In \awstream{}, we demonstrated how
parallelism and sampling techniques can speed up the profiling by up to
29$\times$ and 8.7$\times$ respectively. For compute resource adaptation, we
have improved the efficiency further to address two challenges: Bayesian
Optimization to address the large parameter space; profile transfer to address
device heterogeneity. We show that BO reduces the number of profiling points by
more than 50$\times$.

\section{Thesis Organization}
\label{sec:thesis-organization}

The remainder of this thesis is organized as follows:

\begin{itemize}[topsep=5pt, leftmargin=15pt]
\item \autoref{cha:background} covers the background for swarm applications. I
  first summarize the landscape of the emerging swarm applications and show that
  they have fundamental differences from previous related concepts.  Many swarm
  applications are constructed using a cloud-centric approach. I then argue
  against it by discussing the pitfalls, including security, privacy,
  scalability, latency, etc. A new tier of computing infrastructure, the edge,
  arises to accompany the cloud. While it reduces network latency and provides
  more resources, the edge has its own challenges, such as increased
  heterogeneity. The swarm landscape and the argument against the cloud are
  based on joint work with Nitesh Mor, John Kolb, Douglas S. Chan, Nikhil Goyal,
  Ken Lutz, Eric Allman, John Wawrzynek, Edward A. Lee, and John
  Kubiatowicz~\cite{zhang2015cloud}.
\item In \autoref{cha:netw-reso-adapt}, I present adapting swarm applications to
  network resources. Many swarm applications that transport large streams of
  data across a wide area faces challenges with the scarce and variable
  bandwidth. This chapter focus on \awstream{} that integrates application
  adaptation as a first-class programming abstraction and automatically learns
  an accurate profile that models accuracy and bandwidth trade-off. Using the
  profile to guide application adaptation at runtime, we demonstrate that
  \awstream{} achieves sub-second latency with only nominal accuracy drop
  (2-6\%).  The chapter is based on joint work with Xin Jin, Sylvia Ratnasamy,
  John Wawrzynek, and Edward A. Lee~\cite{zhang2018awstream}.
\item In \autoref{cha:comp-reso-adapt}, I present adapting swarm applications to
  compute resources. Due to the heterogeneous capabilities of end devices and
  variable network/serving latency, it is challenging to provide consistent
  bounded response times for swarm applications. I propose to build a
  performance model that characterizes accuracy and processing time trade-off to
  guide the execution. This chapter focuses on efficient profiling using
  Bayesian Optimization (BO) to address the large parameter space and profile
  transfer to address heterogeneous capabilities of different devices.
\item \autoref{cha:related-work} surveys related research and industrial
  efforts.
\item Finally, in \autoref{cha:concl-future-work} I conclude this thesis and
  identify important research directions for future work.
\end{itemize}

The research presented in this thesis is supported in part by
\href{https://swarmlab.berkeley.edu/}{Berkeley Ubiquitous SwarmLab}; by the
\href{https://terraswarm.org/}{TerraSwarm Research Center}, one of six centers
supported by the STARnet phase of the Focus Center Research Program (FCRP) a
Semiconductor Research Corporation program sponsored by MARCO and DARPA; by
\href{https://www.icyphy.org/}{iCyPhy}, the Berkeley Industrial Cyberphysical
Systems Research Center, supported by Denso, Ford, Siemens, and Toyota.

\end{document}

%%% Local Variables:
%%% mode: latex
%%% TeX-master: t
%%% End:
