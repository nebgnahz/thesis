\documentclass[thesis.tex]{subfiles}

\begin{document}

\chapter{Introduction}

Over the past two decades, we have seen a growing number of networked sensors
and actuators installed in our connected world. These sensors and actuators
offer an unprecedented ability to monitor and act. Because of the enormous
potentials in solving societal-scale problems, this trend has gained significant
attraction, as demonstrated by many parallel efforts: Internet of Things
(IoT)~\cite{atzori2010internet}, Internet of Everything
(IoE)~\cite{bradley2013internet}, Industry 4.0~\cite{lasi2014industry}, The
Industrial Internet~\cite{eigner2018industrial}, TSensors (Trillion
Sensors)~\cite{bogue2014towards}, Machine to Machine
(M2M)~\cite{anton2014machine}, Smarter Planet~\cite{palmisano2008smarter},
etc. In this thesis, we refer to this trend as \textit{the swarm} because it
well characterizes where the potentials lie: the number and the scale of
interconnected devices.

Swarm applications generate, transport, distill, and process large streams of
data across the wide area, often in real time. For example, large cities such as
London and Beijing have deployed millions of cameras for surveillance and
traffic control~\cite{skynet, london.surveillance}. Buildings are increasingly
equipped with a wide variety of sensors to improve energy efficiency and
occupant comfort~\cite{dawson2010smap, krioukov2012building}.

Riding on the popularity of the cloud infrastructure, swarm applications have
adopted the cloud as a universal computation resource and a storage
backend~\cite{carriots, grovestreams, sami, xively, gupta2014bolt,
  zachariah1001internet}. This trend, while understandable given the economic
benefits and simplified management of the cloud, is not the best long term
approach. I will summarize issues privacy, security, availability, quality of
service, etc.

To compensate the cloud, a new infrastructure, the edge, has appeared, such as
the fog~\cite{bonomi2012fog, bar2013fog}, cloudlet~\cite{ha2014towards,
  satyanarayanan2009case, chen2018application}, and swarmbox. While the edge
reduces communication distances, it introduces more levels of heterogeneous.

When developing swarm applications, one must address these challenges with
limited and varying resources and the heterogeneity. For example, the wide-area
network (WAN) bandwidth is scarce and variable. Take network resource as an
example, when facing situations where the bandwidth is not sufficient,
applications deployed today either choose a conservative setting (e.g.\,only
delivering 360p videos) or leave their fate to the underlying transport layer:
(1) in the case of TCP, the sender will be blocked and data are backlogged,
leading to severe delay; (2) in the case of UDP, uncontrolled packet loss
occurs, leading to application performance drop. Instead of ``suffering'' from a
degraded network, applications can act proactively by adjusting their behavior:
reducing the data rate to ensure that important data are delivered in time.

We argue that the key to unfold the potential of swarm applications is to allow
applications to adapt to environment changes. And given the scale of swarm
development, manual adaptation is not feasible.

In this thesis, we propose to adapt swarm applications, in systematic and
quantitative approach. The key is to programming abstraction that developers can
use to embed adaptation and then build tools that automatically adapt. to
provide adaptation as a core abstraction in application development. In this
way, we can build tools to automatically learn the effect of adaptation and
apply adaptation strategies according to resource availability and environment
change.

\vspace{1em}

\noindent\textbf{Thesis Statement}: \textit{Providing adaptation as a programming
  abstraction allows for resilient swarm applications with less developer
  effort.}

\vspace{1em}

\section{Challenges with Developing Swarm Applications}
\label{sec:chall-with-exist}

We identify the following challenges that are unique to swarm applications:

\para{Constrained Resource.} Swarm systems rely on vast numbers of
heterogeneous sensors that are generating massive amounts of data.

\para{Heterogeneous Devices.} The swarm has a wide spectrum of devices with
heterogeneous computing capabilities.

\para{Ad-hoc Development.} Developers strive to make the application working due
to many moving pieces.

\para{Large Design Space.} Many devices. Algorithm tuning. Choices of sensors,
algorithms, data quality.

\section{Systematic and Quantitative Adaptation}
\label{sec:adaptation}

\para{Programming Abstractions.} We propose to introduces new programming
abstractions by which a developer expresses \emph{what} adaptations are
available. Importantly, developers do not have to specify exactly when and how
different adaptations are to be used which is instead left to the underlying
framework.

\para{Data-driven Automatic Profiling.} Rather than rely on manual policies, we
can build tools that automatically \emph{learns} adaptation policy that are
Pareto optimal. This learning process can be both offline and online, exploiting
parallelism, sampling or statistical approaches to efficiently explore the
configuration space. This network-adaptation profile remains the same across
devices.When we apply the same principle to compute resource, in Chapter 3, we
focus on performance modeling and improve the efficiency. Unique challenges:

We also demonstrate how such strategy can be employed in running systems and
adapt to resource changes. For example, \sysname{} matches the streaming data
rate to the measured available bandwidth. Upon encountering network congestion,
it increases the degradation level to reduce the data rate, such that no
persistent queue builds up. To recover, it progressively decreases the
degradation level after probing for more available bandwidth. Our experiments
show that \sysname{} achieves sub-second latency with only nominal accuracy drop
(2-6\%).

\section{Summary of Results}
\label{sec:summary-results-1}

We implement \awstream{} and use it to prototype three streaming applications:
augmented reality (AR), pedestrian detection (PD), and distributed Top-K
(TK). We use real-world data to profile these applications and evaluate their
runtime performance on a geo-distributed public cloud.  We show that
\awstream{}'s data-driven approach generates accurate profiles and that our
parallelism and sampling techniques can speed up profiling by up to 29$\times$
and 8.7$\times$ respectively.

With the combination of \awstream{}'s ability to learn better policies and its
well-designed runtime, our evaluation shows that \awstream{} significantly
outperforms non-adaptive applications: achieving a 40--100$\times$ reduction in
packet delivery times relative to applications built over TCP, or an over
45--88\% improvement in data fidelity (application accuracy) relative to
applications built over UDP. We also compare \awstream{} to
JetStream~\cite{rabkin2014aggregation}, a state-of-the-art system for building
adaptive streaming analytics that is based on manual policies. Our results show
that besides the benefit of generating optimal policies \textit{automatically},
\awstream{} achieves a 15-20$\times$ reduction in latency and 1-5\% improvement
in accuracy simultaneously relative to JetStream.

\section{Thesis Organization}
\label{sec:thesis-organization}

The remainder of this thesis is organized as follows:

\begin{itemize}[topsep=5pt]
\item \autoref{cha:background} covers the background for swarm applications. I
  first summarize the landscape of the emerging swarm applications and show that
  they have fundamental differences from previous related concepts.  Many swarm
  applications are constructed using a cloud-centric approach. I then argue
  against it by discussing the pitfalls including security, privacy,
  scalability, latency, etc. A new tier of computing infrastructure, the edge,
  arises to accompany the cloud. While it reduces network latency and provides
  more resource, the edge has its own challenges, such as increased
  heterogeneity. The swarm landscape and the argument against the cloud are
  based on joint work with Nitesh Mor, John Kolb, Douglas S. Chan, Nikhil Goyal,
  Ken Lutz, Eric Allman, John Wawrzynek, Edward A. Lee, and John
  Kubiatowicz~\cite{zhang2015cloud}.
\item In \autoref{cha:netw-reso-adapt}, I present adapting swarm applications to
  network resources. Many swarm applications that transport large streams of
  data across the wide area faces challenges with the scarce and variable
  bandwidth. This chapter focus on \awstream{} that integrates application
  adaptation as a first-class programming abstraction and automatically learns
  an accurate profile that models accuracy and bandwidth trade-off. Using the
  profile to guide application adaptation at runtime, we demonstrate that
  \awstream{} achieves sub-second latency with only nominal accuracy drop
  (2-6\%).  The chapter is based on joint work with Xin Jin, Sylvia Ratnasamy,
  John Wawrzynek, and Edward A. Lee~\cite{zhang2018awstream}.
\item In \autoref{cha:comp-reso-adapt}, I present adapting swarm applications to
  compute resources. Due to the heterogeneous capabilities of end devices and
  variable network/serving latency, it is challenging to provide a consistent
  bounded response times for swarm applications. I propose to build a
  performance model that characterizes accuracy and processing time trade-off to
  guide the execution. This chapter focuses on efficient profiling: using
  Bayesian Optimization (BO) to address the large parameter space and profile
  transfer to address heterogeneous capabilities of different devices.
\item Chapter 5 discusses related research and industrial efforts.
\item Finally, I conclude this thesis and identify important research directions
  for future work.
\end{itemize}

The research presented in this thesis is supported in part by Berkeley
Ubiquitous SwarmLab~\cite{swarmlab} and the TerraSwarm Research
Center~\cite{terraswarm}, one of six centers supported by the STARnet phase of
the Focus Center Research Program (FCRP) a Semiconductor Research Corporation
program sponsored by MARCO and DARPA.



\end{document}

%%% Local Variables:
%%% mode: latex
%%% TeX-master: t
%%% End:
