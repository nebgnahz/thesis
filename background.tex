\documentclass[thesis.tex]{subfiles}

\begin{document}

\chapter{Background and Motivation}
\label{cha:background}

In this chapter, we first show the trends of swarm applications and how they
differ from previous topics, such as wireless sensor networks, cyber-physical
systems and ubiquitous computing. By analyzing the fundamental properties of
swarm applications, we summarize the core challenges with developing swarm
applications, such as constrained resources and heterogeneous environment.

One key potential with swarm applications is due to the scale facilitated by the
inter-connectivity, especially as the cloud becomes mature and is used as the
universal computation and storage backend. Many applications are constructed by
directly connecting devices to the cloud. This trend, although understandable,
is not the best long term approach. We discuss the pitfalls with this
cloud-centric approach and argue that ``the cloud is not enough.''

To accompany the cloud, a new tier of computing infrastructure, the edge,
arises. Due to its close proximity to end devices, the edge is effective to
reduce network latency and provide more resource guarantees to end
devices. However, unlike the cloud, the edge is resource constrained and it is
unclear what capabilities we can expect from the edge. As a result, swarm
applications relying on the edge need to take the heterogeneous landscape into
account.

\section{The Emerging Swarm}
\label{sec:emerging-swarm}

Smart devices are everywhere: \href{http://ilumi.co/}{light bulb},
\href{https://nest.com/}{thermostat}, \href{http://lunasleep.com/}{mattress
  cover}, \href{https://www.indiegogo.com/projects/smart-diapers}{e-diaper},
\href{https://www.fitbit.com/}{fitness tracker},
\href{https://www.fitbit.com/aria}{smart scale},
\href{https://www.myvessyl.com/}{smart cup},
\href{https://www.kickstarter.com/projects/1816678675/smartplate-instantly-track-and-analyze-everything}{smart
  plate},
\href{http://www.amazon.com/HAPILABS-102-HAPIfork-Bluetooth-Enabled-Smart/dp/B00FRPCPEC}{smart
  fork}, \href{http://electroluxdesignlab.com/en/submission/smart-knife/}{smart
  knife},
\href{http://www.clickandgrow.com/pages/what-is-click-grow}{flowerpot},
\href{http://www.williams-sonoma.com/products/breville-die-cast-2-slice-stainless-steel-smart-toaster/}{toaster},
\href{https://garageio.com/}{garage door},
\href{http://www2.withings.com/us/en/products/baby/smart-baby-monitor}{baby
  monitor},
\href{https://www.indiegogo.com/projects/smartmat-the-world-s-first-intelligent-yoga-mat}{yoga
  mat},
\href{http://usnews.rankingsandreviews.com/cars-trucks/best-cars-blog/2013/02/2015_GM_Vehicles_Will_Get_Wi-Fi_Internet_Access/}{sport-utility
  vehicle}, and
\href{http://www.samsung.com/us/appliances/refrigerators/RF28HMELBSR/AA}{refrigerator}.\footnote{Courtesy
  of Prof.\,Randy Katz.} Capable of computation and communication, they monitor
and interact with the physical world.  The growth of number of devices Analysts
predicted 26 billion devices by 2020~\cite{middleton2013forecast} and trillion
by 2022.
 
% At Person-level, Quantified Self (QS). Home/Office classic demo: turn on the
% lights when I get to the office air-conditioning, HVAC (Building Robotics, Comfy
% from SDB). Industrial (closer to the Wireless Sensor Networks in 2000s)
% agriculture structure monitoring environment monitoring (SFpark, Beijing
% Pollution).

We refer to the collection of sensors and actuators installed in our environment
as the ``Swarm'', a term coined by Jan Rabaey during a keynote talk at ASPDAC
2008~\cite{rabaey2008brand}. This term well characterizes where the potentials
lie: it is not in the individual components, but rather the scale and the number
of interconnected devices: a shift from ``Moore's Law''\footnote{The number of
  transistors on a chip doubles every year while the costs are halved.} to
``Metcalfe's Law.''\footnote{The effect of a telecommunications network is
  proportional to the square of the number of connected users of the system,
  i.e., $n^2$.}.

At Berkeley, the Qualcomm Ubiquitous Swarm Lab~\cite{swarmlab} and Terraswarm
Research Center~\cite{terraswarm} were launched to address the huge potentials
and challenges for Swarm systems in December 2011 and January 2013,
respectively. Outside of Berkeley, this trend is also gaining attraction,
although the names can be different: Internet of Things
(IoT)~\cite{atzori2010internet}, Internet of Everything
(IoE)~\cite{bradley2013internet}, Industry 4.0~\cite{lasi2014industry}, The
Industrial Internet~\cite{eigner2018industrial}, Trillion Sensors
(TSensors)~\cite{bogue2014towards}, Machine to Machine
(M2M)~\cite{anton2014machine}, Smarter Planet~\cite{palmisano2008smarter},
etc. Because of the widespread use of IoT outside of Berkeley, we will use IoT
and swarm interchangeably throughout this thesis. Quoting Lee, the term ``IoT''
includes the technical solution ``Internet technology'' in the problem statement
``connected things''~\cite{lee2016iot}.

\subsection{Swarm Applications}
\label{sec:swarm-applications}

Instead of grouping applications based on their domains---transportation and
logistics domain, healthcare domain, smart environment (home, office, plant)
domain, and personal/social domain~\cite{atzori2010internet}, we characterize
the applications based on the usage pattern, hence application
requirements. Swarm applications fall into two general categories,
\todo{Expand the following two items.}

\begin{itemize}[topsep=5pt]

\item \textbf{Ambient Data Collection and Analytics.} These applications involve
  sensors installed in buildings~\cite{dawson2010smap},
  homes~\cite{hnat2011hitchhiker}, cities~\cite{sfpark}, and on humans
  themselves\footnote{Often referred to as Quantified Self.}~\cite{fitbit,
    swan2013quantified}.  Normally, data is not immediately inspected and
  collected data is later used for analytics~\cite{kolter2011redd}.  The trend
  of data collection is constantly growing and many researchers predict a new
  big-data problem~\cite{diaz2012big, zaslavsky2013sensing}.  Many of these
  applications have serious privacy implications, such as personal health,
  operational security, etc.

\item \textbf{Real-time Applications with Low-latency Requirement.}~These
  applications include reactive environments with humans in the
  loop~\cite{cooperstock1997reactive}.
%% http://www.nngroup.com/articles/response-times-3-important-limits/
An upper latency limit to avoid notice by human participants is about 100
ms~\cite{nielsen1994usability}.  In autonomous systems where humans are not
involved (such as robots taking actions based on sensors), tight control over
latency is important for deterministic
applications~\cite{eidson2012distributed}.  Tight latency requirements are often
incompatible with the unpredictable performance of cloud-based analytics or
controllers.

\end{itemize}

\subsection{Swarm Platforms}
\label{sec:swarm-platforms}

At the same time, we have seen a dizzying array of embedded platforms, from
powerful computing units to low-power microcontrollers (see
\autoref{tab:embedded}). None of these platforms \emph{must} connect with the
cloud; in fact the smallest devices require \emph{gateway} devices to even
communicate with cloud applications. Below are three categories of embedded
platforms:

\para{Smartphones.}~Many companies (like Fitbit~\cite{fitbit} or
Automatic~\cite{automatic}) already use smartphones as gateways to connect
low-power devices to the network.  Researchers have explored how smartphones can
be used for IoT including reusing discarded smartphones~\cite{challen2014mote},
writing new operating systems~\cite{janos} and developing novel
applications~\cite{hong2014smartphone}.

\para{Mini PC.}~Ranging from the powerful Mac Mini and Intel Next Unit of
Computing (NUC) to inexpensive Raspberry Pi, BeagleBone Black, these devices
typically run various versions of Linux to simplify application deployment.
Many companies~\cite{ninja, smartthings, wink} adopt these mini PCs as their
gateway devices.

\para{Microcontroller Platforms.}~Examples include Arduino~\cite{arduino},
mbed~\cite{mbed}, and Spark~\cite{spark}. This is an emerging category,
providing new open platforms on crowdfunding websites~\cite{kickstarter}, good
ecosystems featuring great support and libraries (e.g., Adafruit Online
Tutorials~\cite{adafruit}), and novel applications/products~\cite{iotlist}.

\begin{table}
  \centering
  \begin{tabular}{c c c c}
    \toprule
    Device & CPU Speed & Memory & Price \\
    \midrule
    Intel NUC & 1.3 GHz & 16 GB & \texttildelow\$300 \\
    Typical Phones & 2 GHz & 2 GB & \texttildelow\$300 \\
    Discarded Phones\tablefootnote{This data is from \cite{challen2014mote}, where the
    original authors noted ``Customer buyback price quoted by Sprint for a smartphone in good condition.''} & 1 GHz & 512 MB & \texttildelow\$22 \\
    BeagleBone Black & 1 GHz & 512 MB & \$55 \\
    Raspberry Pi & 900 MHz & 512 MB & \$35 \\
    Arduino Uno & 16 MHz & 2 KB & \texttildelow\$22 \\
    %% http://arduino.cc/en/Products.Compare
    mbed NXP LPC1768 & 96 MHz & 32 KB & \$10 \\
    \bottomrule
  \end{tabular}
  \vspace*{-0.075in}
  \caption{The world of IoT includes a wide spectrum of computing platforms
    (price as of 2015).}
  \vspace*{-0.1in}
  \label{tab:embedded}
\end{table}

%% CSV Data:
%% Intel NUC:
%% http://www.intel.com/content/dam/www/public/us/en/documents/product-briefs/nuc-kit-d54250wyk-product-brief.pdf

% Platform, CPU, Memory, Price
% Intel NUC, 1.3 GHz*4, 16 GB,
% Apple A8, 1.4 GHz*2, 1 G, $600
% Nexus 6, 2.7 GHz*4, 3G, $600
% Raspberry Pi, 900MHz*4, 1G, $35
% BeagleBone, 720MHz, 256MB,
% Arduino, 16MHz, 8KB, $75
% mbed NXP LPC11U24, 48MHz, 8KB, $10
% mbed NXP LPC1768, 96MHz, 32KB, $10

\subsection{Swarm and Related Concepts}
\label{sec:swarm-relat-conc}

At first glance, the Swarm may seem similar to other concepts such as wireless
sensor network (WSN), cyber-physical systems (CPS), and ubiquitous computing
(Ubicomp). However, as we will show with side-by-side comparisons, the Swarm has
its fundamental differences and new challenges to address.

\subsubsection{Swarm and WSN}
\label{sec:swarm-wsn}

In the 1990s, ``Smart Dust''~\cite{kahn1999next} envisions wireless sensor nodes
with a volume of one cubic millimeter that can sense, communicate and
compute. These devices can then be spread through our environment and enhance
our interaction with the physical world. When answering the question
\question{what happens if sensors become tiny, wireless, and self-contained,}
researchers come up with the vision of a group of spatially dispersed and
dedicated wireless sensors to monitor and record the physical conditions of the
environment.

WSN offers the potential to dramatically advance scientific fields or extracting
engineering insights by enabling dense temporal and spatial measurements. The
literature is rich and interested readers are welcome to read
further~\cite{akyildiz2002wireless, zhao2009wireless}. We list a few notable
representative deployments as follows,

\begin{itemize}[itemsep=5pt]
\item \textbf{Redwoods}~\cite{tolle2005macroscope}. Tolle et al.\,deployed 80
  nodes at redwoods that recorded 44 days of air temperature, relative humidity,
  and photosynthetically active solar radiation data.
\item \textbf{GGB}~\cite{kim2007health}. Kim et al.\,deployed 59 nodes over the
  span and the tower, collecting ambient vibrations in two directions
  synchronously at 1KHz rate at Golden Gate Bridge for structural health
  monitoring
\item \textbf{ZebraNet}~\cite{zhang2005habitat}. It tracks animal migration by
  placing sensor nodes into the zebra's collar and the nodes travel with the
  animals.
\end{itemize}

While WSNs have great potential for many applications in a wide range of
scenarios, the swarm is one step further, as we illustrate below with their
differences:

\para{WSN is application-specific while the Swarm can re-purpose itself}.
Envision a futuristic “tale of two smart cities” with safe, efficient, and
comfortable transportation and communication during the best of times, and
secure, quick, and adapt- able emergency response during the worst of
times. Smart Cities use the TerraSwarm infrastructure to aggregate information
from multiple sources, and use this information to (for example) automatically
reroute traffic and identify health and safety threats, such as those created by
an earthquake or a terrorist attack. TerraSwarm applications identify
individuals who can benefit from information that has been gathered, and notify
them using local resources such as cell phones, nearby displays, or audio
systems. These sys- tems can be used to form response teams and implement a
range of rescue and security operations.

\para{WSN runs on a dedicated set of resources while the Swarm can dynamically
  recruit devices}. Sensor node failure due to environmental conditions, such as
underwater. While open architectures with dynamically recruit-able resources can
open up significant security and privacy risks, they can also make systems more
efficient (through sharing of resources), more resilient (through dynamic
reconfiguration leveraging redundant re- sources), and more capable, enabling
applications we have not yet invented or that cannot yet be realized.

\para{WSN nodes are significantly less heterogeneous than Swarm devices}.

\para{WSNs are mostly stand-alone while the swarm interacts with the cloud and
  others closely}. The swarm differs as it coexists and closely interacts with
the two other essential components of the global information-technology platform
that have emerged over the past decades: the cloud, offering unbounded
computational, communication and storage capacity; and the mobiles, which are
the preferred means of access to and creation of information today.

In summary, while the swarm finds its origin in the WSN concept, the swarm
represents a much broader vision, potentially connecting trillions of sensory
and actuating devices that are heterogeneous and worldwide into a single
platform abstraction.

\subsubsection{Swarm and CPS}
\label{sec:swarm-cps}

What is CPS? Cyber-physical systems (CPS) are physical and engineered systems
whose operations are monitored, coordinated, controlled and integrated by a
computing and communication core~\cite{rajkumar2010cyber}. For CPS, ``cyber''
and ``physical'' parts are tightly coupled in a feedback relation, continuously
affecting one another. This property of CPSs renders the separate understanding
of the physical and the cyber insufficient, and requires CPS research to focus
on the intersection of the two rather than the union by developing formalisms
that require a complete understanding of the cyber-physical interaction [83,
78]. Advancements in this domain have recently enabled disruptive applications,
most notably in industrial automation, manufactur- ing, transportation, wearable
technologies, and energy systems. Multiple research directions have emerged to
realize systems sharing the theme of connecting billions of devices to humans
and their environment, realizing large-scale ``smart'' systems. One such
direction that has been receiving significant attention from researchers and
investors is the Internet-of-Things (IoT).

CPS and swarm. The swarm can be viewed as a realization of CPSs, which
emphasizes the concept of intensively networked components. IoT envisions
leveraging Internet technology to connect and unprecedented number of devices,
yielding a ``swarm'' of heterogeneous sensors and actuators that can interact
with the physical environment and can be used to enable dynamic decision making
in many different domains [79, 80]. Of course, the resulting device swarms are
envisioned to be ``smart,'' continuously learning from behavioral patterns of
humans and other devices, then autonomously adapting to changes at run time.

Swarm challenges in addition to CPS. This ability builds upon the implicit
assumption that systems will be able to make real-time decisions on streaming
data. While the meaningful interpretation of this specification is exceedingly
context-dependent, for the purposes of the IoT, the implied requirement is the
ability of a system to contin- ually process streams of data in order to extract
actionable information.

\subsubsection{Swarm and Ubicomp}
\label{sec:swarm-wsn}

What is Ubicomp? Ubiquitous computing (or ``ubicomp'') is a concept in software
engineering and computer science where computing is made to appear anytime and
everywhere.

Swarm and Ubicomp.

Swarm challenges.

%%% Local Variables:
%%% mode: latex
%%% TeX-master: "../background"
%%% End:

\section{Massive Adoption of Cloud Platforms}
\label{sec:cloud}

Cloud Computing~\cite{armbrust2010view} delivers computing services over the
Internet, offering computing as a utility. The term ``cloud'' often refers to
the hardware and systems in the datacenters that provide those computing
services. A cloud can be made public such that customers use it in a
pay-as-you-go manner, such as Amazon Web Services, Google AppEngine, and
Microsoft Azure. In this way, the cloud offers a number of advantages over
traditional IT approaches.

\begin{enumerate}
\item \textbf{Cost.} The cloud eliminates up-front capital expenses and resource
  management. Cloud users can focus on their application logic without hiring IT
  experts to manage on-site datacenters. These time-consuming IT chores include
  provisioning, physical configuration, hardware upgrades, software patching,
  etc.

\item \textbf{Elasticity.} The cloud offers an illusion of infinite resources on
  demand. It allows cloud users to start small and increase their resources with
  the demand growth. Cloud users do not need to plan far ahead for provisioning
  and can request the right amount of resources when they are needed. They can
  also request specialized hardware resources, such as FPGA, GPU, and TPU.

\item \textbf{Utilization.} The pay-as-you-go model incentives users to release
  the resources that are no longer needed. As a result, these resources can be
  reused. Taking advantage of statistical multiplexing, the cloud can greatly
  improve resource utilization.

\item \textbf{Availability.} The cloud facilitates data availability and
  sharing. All end users can access the services ``anytime, anywhere'', making
  collaboration seamless. In addition, the cloud keeps the data safe and durable
  (with backups, disaster recovery).

\end{enumerate}

Because of these benefits, cloud computing has shaped the software industry and
made the development and deployment of web services easier than ever. It has
become the \textit{de facto} central piece around which modern web applications
are constructed. The trend of migrating computation, storage and services to the
cloud is evident. Taking Amazon S3 as an example, over two trillion objects were
reported stored in the system back as of April 2013~\cite{barr2013amazon}.

%%% Local Variables:
%%% mode: latex
%%% TeX-master: "../background"
%%% End:

\section{The Cloud is Not Enough}
\label{sec:cloud-not-enough}

At first glance, directly using the cloud seems to be a natural architecture for
swarm applications. The industry can benefit hugely from the economic model of
the cloud. With little investment in infrastructure, even novice users can start
collecting sensor data and streaming it back to the cloud. However, several
significant problems with this approach are revealed on closer inspection,
including issues with privacy, security, scalability, latency, bandwidth and
availability.  While these problems are not new to typical web applications,
they are exacerbated in the swarm space because of fundamental differences
between swarm and web services. We summarize these concerns in
\autoref{tab:cloud-pitfalls} and list the reasoning as follows:

\begin{table}
  \centering
  \begin{tabular}{c c c}
    \toprule
    & Web/IT & Swarm/IoT \\
    \midrule
    Privacy \& Security & Open for access & Sensitive data \\
    Scalability & Power law & Billion devices \\
    Interaction Model & Human & Machine \\
    Latency & Variable & Bounded  \\
    Bandwidth & Downstream & Upstream   \\
    Availability (QoS) & No guarantee & Requirement  \\
    Durability Management & Cloud controls & Users control \\
    \bottomrule
  \end{tabular}
  \caption{Pitfalls with Today's Approach to directly using the cloud.}
  \label{tab:cloud-pitfalls}
\end{table}

\begin{enumerate}

\item \textbf{Privacy and Security.} Sensors implanted in our surrounding
  environment collect extremely sensitive information.  In a talk given by
  Wadlow~\cite{wadlow}, he described the IoT as ``hundreds of computers that are
  aware of me, can talk about me, and are out of my control.''  This is a strong
  call for intrinsic security and privacy.  This need is echoed in critical
  posts and talks---``Internet of Crappy Things''~\cite{alex2015internet}, ``The
  Internet of Fails''~\cite{stanislav2014the}, etc.  The FTC's Technical
  Report~\cite{ftc2015internet} also emphasizes security in the IoT spaces.  The
  list in \autoref{chap:iot-hacks} demonstrates the severity of the security
  landscape. On the other hand, as a centralized resource out of users' control,
  the cloud presents an ever-present opportunity to violate privacy.  Today,
  privacy has become a luxury~\cite{angwin2014has}, a situation that will be
  exacerbated in the IoT.

\item \textbf{Scalability.} By 2020, Cisco estimates 50
  billion~\cite{evans2011internet} devices will be connected to the cloud, while
  Gartner estimates 26 billion~\cite{middleton2013forecast}. Scalability in the
  IoT spaces will be more challenging than web-scale or Internet-scale
  applications; the amount of data generated easily exceeds the reported
  trillion objects in Amazon S3~\cite{barr2013amazon}. The bisection bandwidth
  requirements for a centralized cloud solution are staggering, especially since
  most data acquired by IoT devices can or should be processed locally and
  discarded.

\item \textbf{Modeling: Peripheral devices are physical.}  Both sensors and
  actuators are physically present devices in our environment.  Although sensor
  data can be collected and replicated (similar to virtualizing
  sensors~\cite{yuriyama2010sensor}), the data is still generated from the edge
  of the network.  Moreover, actuators cannot be virtualized and oftentimes the
  actuations cannot be rolled back~\cite{dawson2013boss}. This is significantly
  different from the model of web services today.

\begin{figure}
  \centering
  \includegraphics[width=0.7\textwidth]{figures/cloud-view.pdf}
  \caption{Although applications usually view the cloud as the center of
    all connected devices (\textit{upper diagram}), in reality the cloud
    is usually on the edge of the Internet backbone, just like other
    devices (\textit{lower diagram}).}
  \label{fig:network}
\end{figure}

\item \textbf{Latency: The cloud model differs from reality.}  Application
  developers view the cloud as a component that interconnects the smart devices.
  However, from a network point of view, the cloud is on the edge of the network
  (see \autoref{fig:network}).  Even simple IoT applications, such as those that
  turn on a fan in response to a rise in local temperature, will experience
  unpredictable latencies from sensing, wireless transmission, gateway
  processing, Internet Service Provider (ISP), Internet, and cloud processing.

%% - sensor latency: 10-100 microseconds
%% - wireless latency: 676 µs (BLE, Overview and Evaluation of Bluetooth Low
%% Energy: An Emerging Low-Power Wireless Technology). Normally it's ~100 ms.
%% May need some data from http://iplane.cs.washington.edu/data/data.html

\item \textbf{Bandwidth: Upstream traffic dominates.}  Shipping data to the
  cloud incurs a significant amount of upstream traffic.  Typical broadband
  networks have more downstream bandwidth than upstream bandwidth.  IoT
  applications, however, generate data at the edges of the network, a pattern
  that will easily saturate the upstream link bandwidth---especially at scale.
  For example, a single Dropcam requires ``a high speed internet connection with
  at least 0.5 Mbps'' to use its service~\cite{dropcam}.  Even simple sensors,
  such as energy meters, can benefit from a higher sampling rate (the motivation
  of 1 kHz energy data with ground-truth from the Ubicomplab at the University
  of Washington~\cite{gupta2015household} and 15 kHz sampling of energy from MIT
  REDD Dataset~\cite{kolter2011redd}).

\item \textbf{Quality of Service (QoS) Guarantees.} Web users tolerate
  variable latency and occasional loss of web services.  In contrast, the
  temporary unavailability of sensors or actuators within IoT applications will
  directly impact the physical world.  While significant engineering effort has
  been put into improving the availability and latency profile of the cloud
  (allowing Service Level Agreements), such efforts are stymied by operator
  error, software bugs, DDoS attacks, and normal packet-to-packet variations
  from wide-area routing. Further, the Internet connection to people's homes is
  far from perfect.  Over 10\% of home networks in the developed world see
  connectivity interruptions of more than ten minutes more frequently than once
  every 10 days~\cite{grover2013peeking}; this situation is worse in developing
  countries.

\item \textbf{Durability Management.} Some sensor data is ephemeral: while other
  data should be durable against global disasters.  For ephemeral data, there is
  no effective way of verifying the data has been completely destroyed because
  the cloud is out of the user's control. For durable data, regardless of the
  promised guarantees~\cite{s3durability}, the reliability of cloud storage
  remains a major concern and there is active research in this
  direction~\cite{bessani2013depsky}.  Moreover, whatever durability is achieved
  by the cloud, it is typically done so without concern for application-specific
  privacy or export rules.  Note that control over durability is closely related
  to control in general: making sure that users retain control over their data
  rather than providers.

\end{enumerate}

%%% Local Variables:
%%% mode: latex
%%% TeX-master: "../background"
%%% End:

\section{Edge Computing}
\label{sec:edge-computing}

The edge represents a new tier of infrastructure envisioned to address some of
the pitfalls with the cloud for IoT applications. Cisco named it \textit{the
  fog} because the fog is a cloud close to the ground~\cite{bonomi2012fog}. CMU
chose \textit{cloudlet} to indicate this small-scale cloud
datacenter~\cite{ha2014towards, satyanarayanan2009case}. At Berkeley, we use
\textit{swarmbox} to name the hardware platform that accompanies swarm
devices. Both Smartphones and Mini PCs mentioned
in~\autoref{sec:swarm-platforms} can act as edge computing platforms. Other edge
computing infrastructure may be provided by carriers, hosted in central
offices\footnote{Often, a central office is defined as a building used to house
  the inside plant equipment of potentially several telephone exchanges, each
  serving a certain geographical area.} or cell towers~\cite{att2017edge}.

One common form of the edge computers are gateways. Many gateways provide
application-specific connectivity for IoT devices to interact with the Internet,
bridging low-power communication protocols such as BLE/802.15.4 to IP. Even when
devices can utilize more standard communication protocols such as WiFi, the
gateways still exist as downloadable applications that run on cellphones or
computers, providing custom services or user interfaces.

\begin{figure}
  \centering
  \includegraphics[width=0.85\textwidth]{figures/background.pdf}
  \caption{The characteristics of the mobile, the edge and the cloud.}
  \label{fig:edge}
\end{figure}

\autoref{fig:edge} illustrates the characteristics of the three-tier
architecture. The edge is much more powerful than end devices but less so
compared with the highly-centralized cloud. The edge serves a local area, such
as homes, buildings, or a city. As a result, it has moderate workload. The main
benefit of edge computers comes from the prime location: it sits in the middle
between IoT/mobile devices and the cloud. Applications built with the edge can
reduce network latency, keep data/information local, and tolerate cloud service
outage.

Researcher have begun to explore the benefit of edge platforms. For example, Kim
proposes an approach that leverages the edge computers as locally centralized
points for authentication and authorization to address IoT
security~\cite{kim2017securing}. Zhuo demonstrates how the edge infrastructure
can support computation offloading to achieve low
latency~\cite{chen2018application}. Mor et al.\,proposes a data-centric design
that focuses around the distribution, preservation, and protection of
information to address data privacy, scalability, durability,
etc~\cite{mor2016toward}.

While an open edge platform can realize these
benefit~\cite{zachariah1001internet}, companies tend to provide their own
gateways, such as Ninja Sphere~\cite{ninja}, SmartThings Hub~\cite{smartthings},
Wink Hub~\cite{wink}. The fact that custom gateways are an integral part of
swarm applications leads directly to ``stovepipe'' solutions or
balkanization. Data and services from one company cannot be shared or utilized
by devices from another company: connection protocols, data formats, and
security mechanisms (when present) are proprietary and often undocumented.

The other challenge that comes with the edge infrastructure is their
heterogeneous capabilities. The cloud, with its elasticity, offers the illusion
of infinite compute resources. The edge, on the other hand, can be limited. and
there are practical issues such as the budget. As we mentioned in
\autoref{tab:embedded}, the swarm has a wide spectrum of devices.

the edge is resource constrained and it is unclear what capabilities we can
expect from the edge. As a result, swarm applications relying on the edge need
to take the heterogeneous landscape into account.

% Researchers adopt a similar approach (e.g., HomeHub for the
% LoT~\cite{brush2013lab}).

%%% Local Variables:
%%% mode: latex
%%% TeX-master: "../background"
%%% End:


\end{document}

%%% Local Variables:
%%% mode: latex
%%% TeX-master: t
%%% End:
