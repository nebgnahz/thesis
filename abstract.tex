\begin{abstract}

  The growing number of Internet-connected devices (sensors and actuators) over
  the wide-area are challenging how we construct analytical
  applications. Traditional approaches separate data collection, transportation
  and distillation as individual one-shot tasks, often performed offline. They
  are a poor match to the need of acting on data in real time. Recently, the
  recognition of timely decision-making has spawned many stream processing
  systems for ``big data.''  However, these systems are primarily tailored
  towards the infrastructure within a single cluster. In a cluster, while
  resource allocation is challenging, there are usually enough resources and the
  allocation is a management problem for maximal utilization. In contrast, the
  wide-area faces resource scarcity and variability, making it not possible to
  guarantee enough allocation for applications. Instead, applications have to
  adapt their behaviors to match the available resources.

  While developers can design each individual application to match a specific
  resource configuration, the ad-hoc solution does not generalize across
  applications and different data distributions. Besides, optimizations created
  at design time often don't take the dynamics of the environment into
  consideration and will behave sub-optimally at runtime.

  Recognizing the need for an adaptive behavior across diverse wide-area
  streaming applications, this manuscript proposes a system-level approach that
  separates the application logic from adaptation mechanisms. To achieve this
  goal, I propose a three-stage design framework: (i) a high-level programming
  abstraction that allows developers to express adaptation options; (ii) an
  offline profiling tool that learns the resource demand and the impact on
  application utilities for a specific adaptation strategy---generating an
  application profile; (iii) a runtime system responsive to environment changes,
  maximizing application utility according to the learned profile.

  %% data acquisition devices running on battery have a limited energy budget;
  %% data transportation links such as wireless channels or the wide-area have
  %% limited capacity.

  The resource and adaptation above are intentionally not specific as it
  provides a general framework applicable to network resources, computing
  resources, as well as storage resources. In this proposal, I focus on the
  adaptation with regards to network resources in the wide-area. Adapting to the
  heterogeneous computing infrastructures is planned as future work. It will be
  another part of the final thesis. There is an ongoing research effort (led by
  my colleagues and I had participated) on storage resources; the final thesis
  will also briefly discuss the design and some preliminary results.

  The bulk body of this proposal focuses on network resources. Specifically, I
  present \sysname{}, a stream processing system for the wide area where the
  network capacity is scarce and variable. The key observation is the explicit
  trade-off between application accuracy and bandwidth demand. The system design
  follows the three-stage framework above: degradation APIs, offline profiling
  and a runtime system.

  Using \sysname{}, I have built three real-world applications: a pedestrian
  detection surveillance application, augmented reality for mobile devices and a
  distributed top-k. At places where traditional non-adaptive approaches would
  lead to either significant application accuracy drop or long tail latency,
  \sysname{} gracefully adapts to the network changes, maintaining the balance
  between application utility and system performance.

\end{abstract}

%%% Local Variables:
%%% mode: latex
%%% TeX-master: "thesis"
%%% End:
