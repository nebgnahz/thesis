\begin{abstract}

  The swarm refers to the vast collection of networked sensors and actuators
  installed in our connected world. Many swarm applications generate, transport,
  distill, and process large streams of data across a wide area in real
  time. The increasing volume of data generated at the edge challenges the
  existing approaches of directly connecting devices to the cloud.

  This thesis begins with an overview of emerging swarm and the architecture
  design trends for swarm applications. The swarm faces challenges from the
  scarce and variable WAN bandwidth and the heterogeneous compute environment
  (from low-power microcontrollers to powerful compute units). When network
  resources or compute resources are insufficient, non-adaptive applications
  will suffer from increased latency or degraded accuracy. Existing approaches
  that support adaptation either require extensive developer effort to write
  manual policies or are limited to application-specific solutions.

  This thesis proposes a systematic and quantitative approach to build adaptive
  swarm applications. The solution has three stages: (1) a set of programming
  abstractions that allow developers to express adaptation options; (2) an
  automatic profiling tool that learns an accurate profile to characterize
  resource demands and application performance; (3) a runtime system responsive
  to environment changes, maximizing application performance guided by the
  learned profile.

  We evaluate our methodology with adaptations to network resources and compute
  resources. For network resources, we present a complete design and
  implementation of the framework \awstream{} and several swarm applications:
  pedestrian detection, augmented reality, and monitoring log analysis. Our
  experiments show that all applications can achieve sub-second latency with
  nominal accuracy drops. For compute resources, we focus on improving the
  profiling efficiency---using Bayesian Optimization to address the large
  parameter space and profile transfer to address device heterogeneity.

\end{abstract}

%%% Local Variables:
%%% mode: latex
%%% TeX-master: "thesis"
%%% End:
