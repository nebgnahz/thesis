\begin{abstract}

  The swarm refers to the vast collection of networked sensors and actuators
  installed in our connected world. Many swarm applications transport, distill,
  and process large streams of data across the wide area in real time. The
  increasing volume of data generated at the edge challenges the existing
  approaches of directly connecting devices to the cloud.

  In this thesis, I will first present architecture design trends and challenges
  for developing swarm applications. Specifically, I will focus on the
  challenges from the scarce and variable WAN bandwidth and the heterogeneous
  compute environment (from low-power microcontrollers to powerful compute
  units). When network resources or compute resources are insufficient,
  non-adaptive applications will suffer from increased latency or degraded
  accuracy. Existing approaches that supports adaptation require extensive
  developer efforts to write manual policies or are limited to
  application-specific solutions.

  I propose a systematic and quantitative approach to build adaptive swarm
  applications. The solution includes three stages: (1) a set of programming
  abstractions that allow developers to express adaptation options; (2) an
  automatic profiling tool that learns an accurate profile to characterize
  resource demands and application performance; (3) a runtime system responsive
  to environment changes, maximizing application performance guided by the
  learned profile.

  We evaluate our methodology with adaptations to network resources and compute
  resources. For network resources, I present the complete design and
  implementation of the framework \awstream{} and several swarm applications:
  pedestrian detection, augmented reality, and monitoring log analysis. Our
  experiments show that all applications can achieve low latency responses with
  nominal accuracy drops. For compute resources, I focus on improving the
  profiling efficiency---using Bayesian Optimization to address the large
  parameter space and profile transfer to address device heterogeneity.

\end{abstract}

%%% Local Variables:
%%% mode: latex
%%% TeX-master: "thesis"
%%% End:
