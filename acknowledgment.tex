\documentclass[thesis.tex]{subfiles}

\begin{document}

\begin{acknowledgements}

  First and foremost, I would like to thank my advisor Prof.\,Edward Lee for his
  guidance and support in the past six years. Like most Ph.D. journeys, there
  were multiple occasions that I felt desperately frustrated---when I was stuck
  in a problem or could not see the value of my work. At those times, all I need
  was to have a meeting with Edward. He has a kind of magic that sweeps
  frustration away. With his breath of knowledge and ability to connect insights
  from diverse topics, I always left his office feeling inspired and
  encouraged. I have learned so much from Edward, including building
  proof-of-concepts (he still writes a lot of code), giving credit to people,
  valuing collaboration, and communicating ideas effectively. I consider it my
  great privilege to be his Ph.D. student.

  During my Ph.D., I am mostly involved with the TerraSwarm project, which
  allowed me to collaborate with many outstanding researchers and
  professors. Prof.\,Bj\"orn Hartmann mentored me in my first project, from
  which I learned many practical research skills: how to brainstorm novel ideas,
  set a research agenda, write a paper, design graphics\footnote{The figure
    showing an augmented version of Google Glass is used on the web page of
    \href{https://eecs.berkeley.edu/research}{Research at Berkeley EECS}.},
  revise a paper, and give a conference talk. Later, with Prof.\,John
  Kubiatowicz and many other folks in the Global Data Plane project, we wrote a
  position paper with a fun title: ``The Cloud is Not Enough.''  Prof\,John
  Wawrzynek offered invaluable feedback in the last few years of my Ph.D. Our
  discussion began with a grand SwarmOS vision and evolved into a specific
  direction on adaptation, i.e., this thesis.

  I would like to thank Prof.\,Syliva Ratnasamy and Prof.\,John Chuang for being
  on my thesis committee and offering their feedback. Sylvia taught me to ask
  fundamental questions about the swarm and how it would change the status
  quo. John, with his experience in many novel applications, offered
  perspectives from a broader context.

  Research without thorough evaluations will not make convincing arguments. I
  attribute my evaluation skills to two masters---Prof.\,Xin Jin and Dr.\,David
  Mellis. When working on AWStream, Xin taught me how to design evaluation
  goals, run experiments, and present the results in compelling figures. With
  David, we organized a one-day workshop to evaluate a user interface design for
  makers to use machine learning algorithms. Words fail me when describing how
  much I learned from them.

  I must thank my MSRA mentor Prof.\,Xiaofan Fred Jiang, without whom I may not
  have pursued a Ph.D. or come to Berkeley. I would also like to Prof.\,Lin
  Zhang for his guidance in my undergraduate thesis. These early experiences
  help build a solid foundation for my research in the swarm.

  I have enjoyed my stay in the DOP center. The friendly atmosphere and the
  inspiring conversations with colleagues and friends have helped me march on in
  the grinding Ph.D. experience. I am luck to have shared 545K Cory with Eunsuk
  Kang, Ilge Akkaya, Eleftherios Matsikoudis and 545N Cory with Mehrdad
  Niknami. We would chat about anything when it comes to a research break. I
  came to Berkeley around the same time with Hokeun Kim, Marten Lohstroh,
  Antonio Iannopollo, and Ankush Desai. I feel sincerely grateful to have the
  chance to grow together with you towards independent researchers. I also feel
  fortunate to have met and learned from other past and current members of the
  Ptolemy group---Patricia Derler, Chris Shaver (Yvan Vivid), Michael Zimmer,
  Christos Stergiou, Dai Bui, Ben Lickly, Matt Weber, and Gil Lederman.

  When it comes to logistics, Mary Stewart and Christopher Brooks are
  undoubtedly the best. Mary deals with the physical side and Christopher
  manages the cyber side. I am extremely impressed by and would like to learn
  Christopher's management skills, be it project, software, or life. His email
  ``New Year's Cleanup'' in 2013 was an eye-opener to me about tidying up
  digital assets. Thank you for keeping my graduate life easy off the boring
  office chores and logistics.

  A great part of Edward's research group is that there are always visiting
  scholars. Over the past few years, I have met researchers with
  shockingly-broad backgrounds---Shuhei Emoto, Marjan Sirjani, Joseph Ng,
  Chadlia Jerad Ep Ben Haj Hmida, Moez Ben Haj Hmida, Maryam Bagheri, Victor
  Nouvellet, Atul Sandur, Andr\'es Goen, Ravi Akella, and many others.

  I would also like to thank other students, researchers and collaborators I
  have worked with closely---Yu-Hsiang Sean Chen, Claire Tuna, Achal Dave for
  the Google Glass project and Nitesh Mor, Jack Kolb, Eric Allman for Global
  Data Plane.

  Graduate life would have become monotonous without friends to take my mind off
  the stress. Thank you---Kaifei Chen, Qifan Pu, Xiang Gao, Peihan Miao, Yuting
  Wei, Chaoran Guo, Qian Zhong, Tianshi Wang, Zhuo Chen---for many memorable
  moments.

  Berkeley and nearby cities, Emeryville, Oakland, Albany, El Cerrito, are by
  far the best place I have lived. Despite living on a budget as a poor graduate
  student, I still felt great happiness, because of many local stores: Monterey
  Market, Berkeley Bowl, Cheese Board, Acme Bread, Blue Bottle Coffee, Sweet
  Maria's, Yaoya-san, The Local Butcher Shop and many more that I refrained from
  listing due to space constrains.

  Last but not least, I am grateful to my family: my parents and my
  sister. While my physical distance is moving further away from home---Zhouzhi
  in the same town, Xi'an in the same city, Beijing in the same country, now
  Berkeley, on the same planet---you have always been there with your unwavering
  support and unconditional love. I would especially like to thank my girlfriend
  Limin Chen. Our love has endured the test of the time, time zone difference,
  and long distance. Thank you for your love, company, and encouragement in my
  Ph.D. journey.

\end{acknowledgements}

\end{document}

%%% Local Variables:
%%% mode: latex
%%% TeX-master: t
%%% End:
