\section{The Cloud is Not Enough}
\label{sec:cloud-not-enough}

The IoT industry has benefited hugely from the economic model of the cloud.
With little investment in infrastructure, even novice users can start collecting
sensor data and streaming it back to the cloud~\cite{armbrust2010view}.  Several
IoT cloud platforms~\cite{carriots, grovestreams, xively, sami} have gone
further by offering easy-to-use APIs, data processing, visualization, and sample
code for various hardware platforms.

On the ``thing'' side, hardware platforms such as Arduino~\cite{arduino},
Raspberry Pi~\cite{rpi}, and BeagleBone Black~\cite{bbb} have allowed easy and
cheap prototyping for customized IoT applications.  Companies~\cite{ninja,
  smartthings, wink} in this space offer complete solutions including hardware
gateways, smartphone applications, web portals and cloud storage.

With this blizzard of activity, the current trend seems to be that all
peripherals, including sensors and actuators, communicate directly with the
cloud and interact with each other through web services~\cite{lee2014swarm}.  At
first glance, this seems to be a natural architecture for IoT applications.
However, several significant problems with this approach are revealed on closer
inspection: for instance issues with privacy, security, scalability, latency,
bandwidth and availability.  While these problems are not new to typical web
applications, they are exacerbated in the IoT space because of fundamental
differences between IoT and web services. The reasoning is as follows:

\noindent\textbf{1.~Privacy and Security.} Sensors
implanted in our surrounding environment collect extremely sensitive
information.  In a recent talk given by Wadlow~\cite{wadlow}, he described the
IoT as ``hundreds of computers that are aware of me, can talk about me, and are
out of my control.''  This is a strong call for intrinsic security and privacy.
This need is echoed in critical posts and talks (``Internet of Crappy
Things''~\cite{alex2015internet}, ``The Internet of
Fails''~\cite{stanislav2014the}).  The FTC's Technical
Report~\cite{ftc2015internet} also emphasizes security in the IoT spaces.  As a
centralized resource out of users' control, the cloud presents an ever-present
opportunity to violate privacy.  Today, privacy has become a
luxury~\cite{angwin2014has}, a situation that will be exacerbated in the IoT.

\noindent\textbf{2.~Scalability.} By 2020, Cisco estimates 50
billion~\cite{evans2011internet} devices will be connected to the cloud,
while Gartner estimates 26 billion~\cite{middleton2013forecast}. Scalability in
the IoT spaces will be more challenging than web-scale or Internet-scale
applications; the amount of data generated easily exceeds the reported trillion
objects in Amazon S3~\cite{barr2013amazon}. The bisection bandwidth requirements
for a centralized cloud solution are staggering, especially since most
data acquired by IoT devices can or should be processed locally and discarded.

\noindent\textbf{3.~Modeling: Peripheral devices are physical.}  Both sensors
and actuators are physically present devices in our environment.  Although
sensor data can be collected and replicated (similar to virtualizing
sensors~\cite{yuriyama2010sensor}), the data is still generated from the edge of
the network.  Moreover, actuators cannot be virtualized and oftentimes the
actuations cannot be rolled back.  This is significantly different from the
model of web services today.

\begin{figure}
  \centering
  \includegraphics[width=0.8\columnwidth]{figures/cloud-view.pdf}
  \caption{Although applications usually view the cloud as the center of
    all connected devices (\textit{upper diagram}), in reality the cloud
    is usually on the edge of the Internet backbone, just like other
    devices (\textit{lower diagram}).}
  \label{fig:network}
\end{figure}

\noindent\textbf{4.~Latency: The cloud model differs from reality.}  Application
developers view the cloud as a component that interconnects the smart
devices.  However, from a network point of view, the cloud is on
the edge of the network (see \autoref{fig:network}).  Even
simple IoT applications, such as those that turn on a fan in response to a
rise in local temperature, will experience unpredictable latencies from
sensing, wireless transmission, gateway processing, Internet Service Provider
(ISP), Internet, and cloud processing.

%% - sensor latency: 10-100 microseconds
%% - wireless latency: 676 µs (BLE, Overview and Evaluation of Bluetooth Low
%% Energy: An Emerging Low-Power Wireless Technology). Normally it's ~100 ms.
%% May need some data from http://iplane.cs.washington.edu/data/data.html

\noindent\textbf{5.~Bandwidth: Upstream traffic dominates.}  Shipping data to the
cloud incurs a significant amount of upstream traffic.  Typical broadband
networks have more downstream bandwidth than upstream bandwidth.  IoT
applications, however, generate data at the edges of the network, a pattern that
will easily saturate the upstream link bandwidth---especially at scale.  For
example, a single Dropcam requires ``a high speed internet connection with at
least 0.5 Mbps'' to use its service~\cite{dropcam}.  Even simple sensors, such
as energy meters, can benefit from a higher sampling rate (the motivation of 1
kHz energy data with ground-truth from the Ubicomplab at the University of
Washington~\cite{gupta2015household} and 15 kHz sampling of energy from MIT REDD
Dataset~\cite{kolter2011redd}).

\noindent\textbf{6.~Quality of Service (QoS) Guarantees.} Web users
tolerate variable latency and occasional loss of web services.  In contrast, the
temporary unavailability of sensors or actuators within IoT applications will
directly impact the physical world.  While significant engineering effort has
been put into improving the availability and latency profile of the cloud
(allowing Service Level Agreements), such efforts are stymied by operator error,
software bugs, DDoS attacks, and normal packet-to-packet variations from
wide-area routing. Further, the Internet connection to people's homes is far
from perfect.  Over 10\% of home networks in the developed world see
connectivity interruptions of more than ten minutes more frequently than once
every 10 days~\cite{grover2013peeking}; this situation is worse in developing
countries.

\noindent\textbf{7. Durability Management.} Some sensor
data is ephemeral: while other data should be durable against global disasters.
For ephemeral data, there is no effective way of verifying the data has been
completely destroyed because the cloud is out of the user's control. For
durable data, regardless of the promised guarantees~\cite{s3durability}, the
reliability of cloud storage remains a major concern and there is active
research in this direction~\cite{bessani2013depsky}.  Moreover, whatever
durability is achieved by the cloud, it is typically done so without concern for
application-specific privacy or export rules.  Note that control over durability
is closely related to control in general: making sure that users retain control
over their data rather than providers.

%%% Local Variables:
%%% mode: latex
%%% TeX-master: "../intro"
%%% End:
