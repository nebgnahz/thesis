\section{Heterogeneous Embedded Platforms}
\label{sec:embedded-platforms}

At the same time, we have seen a dizzying array of embedded platforms, from
powerful computing units to low-power microcontrollers (see
\autoref{tab:embedded}). None of these platforms \emph{must} connect with the
cloud; in fact the smallest devices require \emph{gateway} devices to even
communicate with cloud applications. Below are three categories of embedded
platforms:

\begin{enumerate}
\item \textbf{Smartphones:}~Many companies (like Fitbit~\cite{fitbit} or
  Automatic~\cite{automatic}) already use smartphones as gateways to connect
  low-power devices to the network.  Researchers have explored how smartphones
  can be used for IoT including reusing discarded
  smartphones~\cite{challen2014mote}, writing new operating systems~\cite{janos}
  and developing novel applications~\cite{hong2014smartphone}.

\item \textbf{Mini PC:}~Ranging from the powerful Mac Mini and Intel Next Unit
  of Computing (NUC) to inexpensive Raspberry Pi, BeagleBone Black, these
  devices typically run various versions of Linux to simplify application
  deployment.  Many companies~\cite{ninja, smartthings, wink} adopt these mini
  PCs as their gateway devices.

\item \textbf{Microcontroller platforms:}~Examples include
  Arduino~\cite{arduino}, mbed~\cite{mbed}, and Spark~\cite{spark}. This is an
  emerging category, providing new open platforms on crowdfunding
  websites~\cite{kickstarter}, good ecosystems featuring great support and
  libraries (e.g., Adafruit Online Tutorials~\cite{adafruit}), and novel
  applications/products~\cite{iotlist}.

\end{enumerate}

\begin{table}
  \centering
  \begin{tabular}{c c c c}
    \toprule
    Device & CPU Speed & Memory & Price \\
    \midrule
    Intel NUC & 1.3 GHz & 16 GB & \texttildelow\$300 \\
    \hline
    Typical Phones & 2 GHz & 2 GB & \texttildelow\$300 \\
    \hline
    Discarded Phones\tablefootnote{This data is from \cite{challen2014mote}, where the
    original authors noted ``Customer buyback price quoted by Sprint for a smartphone in good condition.''} & 1 GHz & 512 MB & \texttildelow\$22 \\
    \hline
    BeagleBone Black & 1 GHz & 512 MB & \$55 \\
    \hline
    Raspberry Pi & 900 MHz & 512 MB & \$35 \\
    \hline
    Arduino Uno & 16 MHz & 2 KB & \texttildelow\$22 \\
    %% http://arduino.cc/en/Products.Compare
    \hline
    mbed NXP LPC1768 & 96 MHz & 32 KB & \$10 \\
    \bottomrule
  \end{tabular}
  \vspace*{-0.075in}
  \caption{The world of IoT includes a wide spectrum of computing platforms
    (price as of 2015).}
  \vspace*{-0.1in}
  \label{tab:embedded}
\end{table}

%% CSV Data:
%% Intel NUC:
%% http://www.intel.com/content/dam/www/public/us/en/documents/product-briefs/nuc-kit-d54250wyk-product-brief.pdf

% Platform, CPU, Memory, Price
% Intel NUC, 1.3 GHz*4, 16 GB,
% Apple A8, 1.4 GHz*2, 1 G, $600
% Nexus 6, 2.7 GHz*4, 3G, $600
% Raspberry Pi, 900MHz*4, 1G, $35
% BeagleBone, 720MHz, 256MB,
% Arduino, 16MHz, 8KB, $75
% mbed NXP LPC11U24, 48MHz, 8KB, $10
% mbed NXP LPC1768, 96MHz, 32KB, $10

%%% Local Variables:
%%% mode: latex
%%% TeX-master: "../background"
%%% End:
