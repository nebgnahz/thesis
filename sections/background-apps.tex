
\section{Swarm Applications}
\label{sec:swarm-applications}

Many of today's IoT solutions arise by connecting embedded platforms to the
cloud.  For example, Bolt~\cite{gupta2014bolt} provides data management for the
Lab of Things (LoT)~\cite{brush2013lab} and uses Amazon S3 or Azure for data
storage\footnote{This approach is for ``efficiently sharing data across
  homes''~\cite{gupta2014bolt}.}.  Such direct connections often require an
application gateway~\cite{zachariah1001internet} to support low-power wireless
communications such as Z-Wave or Bluetooth Low Energy (BLE).  Even when devices
can utilize more standard communication protocols such as WiFi, the gateways
still exist as downloadable applications that run on cellphones or computers.
Companies tend to provide their own gateways (such as Ninja Sphere~\cite{ninja},
SmartThings Hub~\cite{smartthings}, Wink Hub~\cite{wink}); and researchers adopt
a similar approach (e.g., HomeHub for the LoT~\cite{brush2013lab}).

The fact that custom gateways are an integral part of IoT applications leads
directly to ``stovepipe'' solutions or balkanization. Data and services from one
company cannot be shared or utilized by devices from another company: connection
protocols, data formats, and security mechanisms (when present) are proprietary
and often undocumented.

To date, IoT applications seem to fall into two general categories:

\para{Ambient Data Collection and Analytics.}~These applications involve sensors
installed in buildings~\cite{dawson2010smap}, homes~\cite{hnat2011hitchhiker},
cities~\cite{sfpark}, and on humans themselves\footnote{Often referred to as
  Quantified Self.}~\cite{fitbit, swan2013quantified}.  Normally, data is not
immediately inspected and collected data is later used for
analytics~\cite{kolter2011redd}.  The trend of data collection is constantly
growing and many researchers predict a new big-data problem~\cite{diaz2012big,
  zaslavsky2013sensing}.  Many of these applications have serious privacy
implications (for personal health, operational security, etc.).

\para{Real-time Applications with Low-latency Requirement.}~These applications
include reactive environments with humans in the
loop~\cite{cooperstock1997reactive}.
%% http://www.nngroup.com/articles/response-times-3-important-limits/
An upper latency limit to avoid notice by human participants is about
100 ms~\cite{nielsen1994usability}.  In autonomous systems where humans
are not involved (such as robots taking actions based on sensors), tight
control over latency is important for deterministic
applications~\cite{eidson2012distributed}.  Tight
latency requirements are often incompatible with the unpredictable
performance of cloud-based analytics or controllers.

%%% Local Variables:
%%% mode: latex
%%% TeX-master: "../background"
%%% End:
