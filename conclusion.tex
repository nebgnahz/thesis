\documentclass[thesis.tex]{subfiles}

\begin{document}

\chapter{Conclusion and Future Directions}
\label{cha:concl-future-work}

This thesis claims that systematic adaptation and quantitative profiling are the
key to a resilient swarm. This chapter summarizes our work and discusses
valuable directions for future work.

\section{Thesis Summary}
\label{sec:thesis-summary}

This thesis began with an overview of the emerging swarm. As the swarm grows at
a staggering rate, it faces challenges the underlying communication and
computation infrastructure: the scarce and variable WAN bandwidth and the
heterogeneous compute environment. Existing approaches with manual policies or
application-specific policies do not work for the swarm scale.

I propose to adapt swarm application in a systematic and quantitative way.  The
proposed approach consists of three parts: well-defined programming
abstractions, data-driven profiling, and runtime adaptation. Instead of
specifying policies manually, developers use our APIs to express what adaptation
options are available. From these options, a profiling tool automatically learns
a Pareto-optimal profile that characterizes resource demands and application
performance. At runtime, applications adapt their behaviors based on the learned
profile.

For network resources, such as the scarce and variable WAN bandwidth, swarm
applications explore the trade-off between application accuracy and data
sizes. Developers use \maybe{} APIs to encode adaptation options. Our profiling
tool uses parallelism and sampling techniques to efficiently learn the
Pareto-optimal profile. We implemented three applications and compared against
several baselines. Our evaluation shows that \awstream{} significantly
outperforms non-adaptive applications and applications with manual policies.

For compute resource, such as the heterogeneous devices across the swarm, the
edge and the cloud, swarm applications explore the trade-off between application
accuracy and processing times. We propose to use macro annotations to encode
algorithm parameters. The large parameter space and heterogeneous environment
require us to profile efficiently. We use Bayesian Optimization, and profile
transfer.

\section{Future Directions}
\label{sec:future-directions}

In previous chapters, we have discussed some potential improvements to our
existing work. In this section, we discuss future directions in a broader
context.

The core challenge for the swarm comes from the scale of inter-connected
devices. As the number continues to grow, we expect to see more diverse swarm
applications. While we have used several applications in this thesis as case
studies, we need more applications to evaluate how general the proposed APIs
are, how effective the profiling techniques are, and how responsive the runtime
system is.

With our framework, developers need to encode adaptation with our APIs, provide
training data, and implement accuracy functions. There could be better tooling
and library of reusable functions (such as F1 score for object detection) to
further simplify developer effort, similar to machine learning libraries.

Instead of adapting to resources, swarm applications should be able to
dynamically recruit resources, such as sensors, actuators, data, and computing
infrastructure. What are the necessary technologies to realize this recruitment?
Discovery, management. One promising work is the mutable accessors discussed by
Brooks et al.~\cite{brooks2018component}. An mutable accessor is an abstract
interface specification for candidate accessors. It reifies a concrete accessor
downloaded from the Internet or retrieved through a discovery mechanism. This
reification serves as recruiting resources for the swarm applications.

\end{document}

%%% Local Variables:
%%% mode: latex
%%% TeX-master: t
%%% End:
