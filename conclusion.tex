\documentclass[thesis.tex]{subfiles}

\begin{document}

\chapter{Conclusion and Future Directions}
\label{cha:concl-future-work}

This thesis claims that systematic adaptation and quantitative profiling are the
key to a resilient swarm. This chapter summarizes our work, discusses future
directions, and concludes this thesis.

\section{Thesis Summary}
\label{sec:thesis-summary-1}

This thesis began with an overview of the emerging swarm. As the swarm grows at
a staggering rate, it faces challenges from the underlying communication and
computation infrastructure: the scarce and variable WAN bandwidth and the
heterogeneous compute environment. Existing approaches with manual policies or
application-specific policies do not work for the swarm scale.

I propose to adapt swarm applications in a systematic and quantitative way. The
proposed methodology consists of three parts: $(i)$ a set of well-defined
programming abstractions that allow developers to express what adaptation
options are available; $(ii)$ a data-driven profiling tool that automatically
learns a Pareto-optimal profile; $(iii)$ runtime adaptation that maximizes
application performance while satisfying resource constrains. We apply our
methodology to network resources and compute resources.

For network resources, swarm applications need to address the scarce and
variable WAN bandwidth by exploring the trade-off between application accuracy
and data size demand. In \autoref{cha:netw-reso-adapt}, we have presented our
design, implementation, and evaluation of \awstream{}. Compared with
non-adaptive applications, \awstream{} achieves a 40--100$\times$ reduction in
latency relative to applications using TCP, and a 45--88\% improvement in
application accuracy relative to applications using UDP. Compared with manual
policies using a state-of-the-art system JetStream~\cite{rabkin2014aggregation},
\awstream{} achieves a 15-20$\times$ reduction in latency and 1-5\% improvement
in accuracy simultaneously.

For compute resources, swarm applications need to address the heterogeneous
compute environment by exploring the trade-off between application accuracy and
processing times. In \autoref{cha:comp-reso-adapt}, we have focused on improving
profiling efficiency. First, we use Bayesian Optimization to avoid an exhaustive
exploration of the large parameter space. BO significantly reduces the number of
samples needed, by more than 50$\times$ compared to an exhaustive
profiling. With the same search budget, BO finds better Pareto-optimal
configurations than previous proposed search methods. Second, to profile for new
devices, we propose to transfer an existing profile by updating the processing
times of Pareto-optimal configurations. The profile transfer eliminates the need
for get all training data, evaluate the algorithm for all data, and run a BO
engine.

In summary, this thesis presents a systematic and quantitative approach for
adaptation.

\section{Future Directions}
\label{sec:future-directions}

Before concluding this thesis, I discuss three directions that I find valuable
as future work. Earlier in \autoref{sec:awstream-discussion} and
\autoref{sec:brt-discussion}, we have discussed potential improvements that are
near term and specific to our current design and implementation. This section
focuses on future directions that are in a longer term and for a broader
context.

\para{More Diverse Applications and Generalization.} The core challenge for the
swarm comes from the scale of interconnected devices. As the number of devices
continues to grow, we expect to see more diverse swarm applications. While we
have used several applications in this thesis as case studies, we need more
applications to evaluate how general the proposed APIs are, how effective the
profiling techniques are, and how responsive the runtime system is. Some
enhancements may be easily incorporated into our existing system. For example,
because our APIs are extensible, we can build a library of reusable functions
for accuracy measurements to reduce developer effort. Other enhancements may
require a redesign of APIs, profiling tools, and/or the runtime.

\para{Adaptation and Resource Allocation.} In addition to adapting to resources,
swarm applications should be able to dynamically \emph{recruit} resources, such
as sensors, actuators, data, and computing infrastructure. We have envisioned
SwarmOS that controls and allocates resources for different swarm
applications. However, some swarm resources are beyond the control of individual
entities (such as WAN bandwidth); some swarm resources are not as elastic as the
cloud (such as end devices that need to be purchased and installed). In these
cases, swarm applications need to adapt to such resources. One interesting
direction is to use a combination of adaptation and resource allocation. In
\autoref{sec:resource-allocation}, we explored allocating available bandwidth
among multiple applications and how different allocation schemes affect
application behaviors. We believe this is an important future direction for the
SwarmOS vision.

\para{Adaptation and Mutation.} The adaptation we have explored thus far is
limited in what can be changed: applications only change the algorithms and
parameters that a developer has specified as adaptation options. An extreme case
for adaptation is self-modifying code, which is notoriously difficult to
understand and manage. A slightly more flexible yet still disciplined approach
is mutable accessors as discussed by Brooks et al.~\cite{brooks2018component}. A
mutable accessor is an abstract interface specification for candidate
accessors. It reifies a concrete accessor downloaded from the Internet or
retrieved through a discovery mechanism. In summary, along the spectrum of
adaptation, we expect future work will extend our profiling methodology to
applications that change their structure and mutate their implementation.

\section{Reflections and Concluding Remarks}
\label{sec:conclusions}

In 2013, Gartner~\cite{middleton2013forecast} positioned Internet of Things at
the top of the hype cycle, i.e., peak of inflated expectations. In 2014--2015,
the market is all about smart gadgets from crowdfunding
projects~\cite{kickstarter} and startups~\cite{fitbit}; large companies also
began to build IoT platforms~\cite{sami, awsiot}. Despite this blizzard of
activities, IoT applications were not as disruptive as predicted. Many smart
devices around this time provided questionable value, such as a smart bottle
opener that messages your friends when you open a beer, a clothes peg that
notifies you when your washing is dry, a smart egg tray that gives you a count
of remaining eggs, etc~\cite{hartmann2016societal}. Many products and companies
then failed. A survey from Cisco estimated the failure rate to be around
75\%~\cite{cisco2017journey}. In 2016--2017, public attention has transitioned
to AI and blockchain. Interestingly, IoT still marches towards massive scale and
actually benefits from the AI wave. With AI tackling speech recognition and
object detection, it has become increasingly common to control smart devices
through conversation with Alexa~\cite{alexa} and guard home security through
person detection with security cameras~\cite{dropcam}. Looking forward, there
will be more research and industrial effort. The swarm vision will be slowly but
steadily realized.

Developers face many challenges to get a swarm application running, even in
normal conditions. It is often beyond their plans to address abnormal cases when
the application needs to adapt. We have observed anecdotal cases when developers
use manual and ad-hoc approaches for adaptation as quick fixes. The difficulties
of developing swarm applications come from many factors. Due to missing
platforms, developers have to dive into low-level, hardware-specific application
design. Due to the fragmented and stove-piped ecosystem, developers spend
non-trivial amount of time implementing adapters or wrappers for the
interoperability of components.  We expect to see more work on building swarm
platforms~\cite{mor2016toward, latronico2015vision} and tackling
interoperability~\cite{brooks2018component}. And mature swarm platforms will
lead to a proliferation of swarm applications. The adaptation approach described
in this thesis will then allow resilient swarm applications that can work well
even with insufficient resources.

This thesis proposes to adapt swarm applications with a systematic and
quantitative approach. However, it must be noted that not all swarm applications
necessarily need the proposed adaptation. Some swarm applications generate small
amount of data, e.g., millihertz sensors. Some swarm applications are delay
tolerant, e.g., sensor data collection. In these cases, it is fine to develop
applications as is. In addition, I would remark that our approach generalizes
beyond swarm applications. While this thesis centers around swarm applications,
which are an emerging category, we expect to apply this methodology in other
contexts where applications also need to cope with scarce and variable
resources.

\end{document}

%%% Local Variables:
%%% mode: latex
%%% TeX-master: t
%%% End:
