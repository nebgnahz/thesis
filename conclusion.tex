\documentclass[thesis.tex]{subfiles}

\begin{document}

\chapter{Conclusion and Future Work}
\label{cha:concl-future-work}

This thesis claims that \textit{systematic adaptation and quantitative profiling
  are the key to a resilient swarm}. In this thesis, I have studied swarm
applications and demonstrated how to allow adaptation for network and compute
resources. This chapter summarizes the contributions of this thesis and
discusses valuable directions for future work.

\section{Conclusions and Contributions}
\label{sec:contributions}

\paraf{Reviewing the Emerging Swarm.} This thesis provides an overview of
emerging swarm and the architecture design trends for swarm applications. We
discuss issues with existing approaches that directly connect devices to the
cloud. We also analyze the challenges swarm applications face, with a focus on
constrained resources and heterogeneous environment.

\para{Systematic and Quantitative Adaptation.} The core of this thesis is a
systematic and quantitative approach for developing adaptive swarm
applications. Instead of relying on manual policies, developers use well-defined
programming abstractions to express adaptation options. We then use a
data-driven profiling to automatically learn the profile---a set of
Pareto-optimal configurations---that characterizes resource demands and
application performance.

\para{Design, Implementation, and Evaluation of \awstream{}.} Swarm applications
that communicate across the wide area need to handle the scarce and variable WAN
bandwidth. We present a complete design and implementation of the framework
\awstream{} for network resource adaptation. We have developed several swarm
applications: pedestrian detection, augmented reality, and monitoring log
analysis. Our experiments show that all applications can achieve sub-second
latency with nominal accuracy drops.

\para{Improving Profiling Efficiency.} In \awstream{}, we demonstrated how
parallelism and sampling techniques can improve profiling efficiency. For
network resource adaptation, we have improved the efficiency further to address
two challenges: Bayesian Optimization to address the large parameter space;
profile transfer to address device heterogeneity.

\section{Future Directions}
\label{sec:future-directions}

In previous chapters, we have discussed some potential improvements to our
existing work. In this section, we discuss future directions in a broader
context.

The core challenge for the swarm comes from the scale of inter-connected
devices. As the number continues to grow, we expect to see more diverse swarm
applications. While we have used several applications in this thesis as case
studies, we need more applications to evaluate how general the proposed APIs
are, how effective the profiling techniques are, and how responsive the runtime
system is.

With our framework, developers need to encode adaptation with our APIs, provide
training data, and implement accuracy functions. There could be better tooling
and library of reusable functions (such as F1 score for object detection) to
further simplify developer effort, similar to machine learning libraries.

Instead of adapting to resources, swarm applications should be able to
dynamically recruit resources, such as sensors, actuators, data, and computing
infrastructure. What are the necessary technologies to realize this recruitment?
Discovery, management. One promising work is the mutable accessors discussed by
Brooks et al.~\cite{brooks2018component}. An mutable accessor is an abstract
interface specification for candidate accessors. It reifies a concrete accessor
downloaded from the Internet or retrieved through a discovery mechanism. This
reification serves as recruiting resources for the swarm applications.

\end{document}

%%% Local Variables:
%%% mode: latex
%%% TeX-master: t
%%% End:
