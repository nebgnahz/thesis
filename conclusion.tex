\documentclass[thesis.tex]{subfiles}

\begin{document}

\chapter{Conclusion and Future Directions}
\label{cha:concl-future-work}

This thesis claims that systematic adaptation and quantitative profiling are the
key to a resilient swarm. This chapter summarizes our work and discusses
valuable directions for future work.

\section{Thesis Summary}
\label{sec:thesis-summary}

This thesis began with an overview of the emerging swarm. As the swarm grows at
a staggering rate, it faces challenges the underlying communication and
computation infrastructure: the scarce and variable WAN bandwidth and the
heterogeneous compute environment. Existing approaches with manual policies or
application-specific policies do not work for the swarm scale.

I propose to adapt swarm application in a systematic and quantitative way.  The
proposed approach consists of three parts: well-defined programming
abstractions, data-driven profiling, and runtime adaptation. Instead of
specifying policies manually, developers use our APIs to express what adaptation
options are available. From these options, a profiling tool automatically learns
a Pareto-optimal profile that characterizes resource demands and application
performance. At runtime, applications adapt their behaviors based on the learned
profile.

For network resources, such as the scarce and variable WAN bandwidth, swarm
applications explore the trade-off between application accuracy and data
sizes. Developers use \maybe{} APIs to encode adaptation options. Our profiling
tool uses parallelism and sampling techniques to efficiently learn the
Pareto-optimal profile. We implemented three applications and compared against
several baselines. Our evaluation shows that \awstream{} significantly
outperforms non-adaptive applications and applications with manual policies.

For compute resource, such as the heterogeneous devices across the swarm, the
edge and the cloud, swarm applications explore the trade-off between application
accuracy and processing times. We propose to use macro annotations to encode
algorithm parameters. The large parameter space and heterogeneous environment
require us to profile efficiently. We use Bayesian Optimization, and profile
transfer.

\section{Future Directions}
\label{sec:future-directions}

In \autoref{sec:awstream-discussion} and \autoref{brt-discussion}, we have
discussed potential improvements that are near term and specific to our current
design and implementation. Before we concluded this thesis, we discuss future
directions in a longer term and for a broader context.

\para{More Diverse Applications and Generalization.} The core challenge for the
swarm comes from the scale of inter-connected devices. As the number of devices
continues to grow, we expect to see more diverse swarm applications. While we
have used several applications in this thesis as case studies, we need more
applications to evaluate how general the proposed APIs are, how effective the
profiling techniques are, and how responsive the runtime system is. Some
enhancements may be easy to incorporate into our existing system. For example,
because our APIs are extensible, we can build a library of reusable functions
for adaptation options and accuracy measurements to reduce developer
efforts. Other enhancements may require a redesign of APIs, profiling tools,
and/or the runtime.

\para{Adaptation and Resource Allocation.} Instead of adapting to resources,
swarm applications can dynamically \emph{recruit} resources, such as sensors,
actuators, data, and computing infrastructure. We have envisioned SwarmOS that
controls and allocates resources for different swarm applications. However, some
swarm resources are beyond the control of individual entities (such as WAN
bandwidth that is); some swarm resources are not as elastic as the cloud (such
as end devices that need to be purchased and installed). In these cases, swarm
applications need to adapt to such resources. One interesting direction is to
use a combination of adaptation and resource allocation. In
\autoref{sec:resource-allocation}, we explored allocating available bandwidth
among multiple applications and how different allocation schemes affect
application behaviors. We believe this is an important future direction for the
SwarmOS vision.

\para{Adaptation and Interoperability.} The success of the swarm relies on an
open architecture. The propose adaptation techniques require collaborations
across different platforms. It will be hard with stovepipe solutions. Brooks et
al.~\cite{brooks2018component} proposes a component-based software architecture
named ``Accessors'' that are proxies for services and things. Instantiate
accessors on another host and an mutable accessor is an abstract interface
specification for candidate accessors. It reifies a concrete accessor downloaded
from the Internet or retrieved through a discovery mechanism.

\end{document}

%%% Local Variables:
%%% mode: latex
%%% TeX-master: t
%%% End:
