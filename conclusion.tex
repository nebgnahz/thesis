\documentclass[thesis.tex]{subfiles}

\begin{document}

\chapter{Conclusion and Future Work}
\label{cha:concl-future-work}

\todo{rewrite.}

The swarm has huge potentials but also challenges: \textbf{constrained
  resources} and \textbf{heterogeneous devices.}.

Adaptation addresses these challenges. But it is non-trivial to come up with
good adaptation strategies. The problem has a large design space. Manual
policies are suboptimal. Accurate policies require extensive studies.

We propose to support adaptation in programming abstraction and quantitatively
learns the Pareto-optimal profile.

Network Resource Adaptation. \awstream{} addresses scarce and variable WAN
bandwidth. It is tradeoff between application accuracy and data size demand.

Compute Resource Adaptation. Uses BO to address large parameter space and
profile transfer for heterogeneous devices. Exploiting the tradeoff between
application accuracy and processing times.

Overall, a systematic and quantitative approach for adaptation.

\section{Future Directions}
\label{sec:future-directions}

Reducing Developer Effort. While \awstream{} simplifies developing adaptive
applications, there are still application-specific parts required for
developers: wrapping appropriate \maybe{} calls, providing training data, and
implementing accuracy functions. Because \awstream{}'s API is extensible, we
plan to build libraries for common degradation operations and accuracy
functions, similar to machine learning libraries.

For \awstream{}, adaptation more proactively if we can predict bandwidth
changes. Recent research on adaptive video streaming explores model predictive
control (MPC)~\cite{yin2015control, sun2016cs2p} and neural
network~\cite{mao2017neural}. We plan to explore these techniques next.

Build server scheduler to use the profile for compute resources.

Discuss accessors, especially mutable accessors.

\end{document}

%%% Local Variables:
%%% mode: latex
%%% TeX-master: t
%%% End:
