\usepackage{times}

\usepackage[T1]{fontenc}
\usepackage[explicit]{titlesec}
\usepackage[font=small,labelfont=bf]{caption}
\usepackage[utf8]{inputenc}
\usepackage{amsmath}
\usepackage{amssymb}
\usepackage{balance}
\usepackage{booktabs} % For formal tables
\usepackage{enumitem}
\usepackage{hyperref}
\usepackage{listings}
\usepackage{microtype}
\usepackage{multirow}
\usepackage{siunitx}
\usepackage{subcaption}
\usepackage{subfiles}
\usepackage{tikz}
\usepackage{varwidth}
\usepackage{xcolor}

\pdfinfo{
  /Title      (U.C. Berkeley Dissertation)
  /Author     (Ben Zhang)
  /Keywords   ()
}

%%%%%%%%%%%%%%%%%%%%%%%%%%%%%%%%%%%%%%%%%%%%%%%%%%%%%%%%%%%%%%%
%% More defs
%%%%%%%%%%%%%%%%%%%%%%%%%%%%%%%%%%%%%%%%%%%%%%%%%%%%%%%%%%%%%%%
% SINGLE SPACE
\def\ssp{\def\baselinestretch{1.0}\large\normalsize}

% This gives you control over how far down in the hierarchy the
% table of contents will print. I use 2.
\setcounter{tocdepth}{1}
\setcounter{secnumdepth}{3}
\setlength{\parindent}{5ex}

\newcommand\degrees{\ensuremath{^\circ}}
\newcommand{\tab}{\hspace{5mm}}
\newcommand{\blankpage}{\clearpage ~ \newpage}

\pagestyle{headings}

\frenchspacing

%%%%%%%%%%%%%%%%%%%%%%%%%%%%%%%%%%%%%%%%%%%%%%%%%%%%%%%%%%%%%%%
%% Custom Styling
%%%%%%%%%%%%%%%%%%%%%%%%%%%%%%%%%%%%%%%%%%%%%%%%%%%%%%%%%%%%%%%

\definecolor[named]{ACMBlue}{cmyk}{1,0.1,0,0.1}
\definecolor[named]{ACMYellow}{cmyk}{0,0.16,1,0}
\definecolor[named]{ACMOrange}{cmyk}{0,0.42,1,0.01}
\definecolor[named]{ACMRed}{cmyk}{0,0.90,0.86,0}
\definecolor[named]{ACMLightBlue}{cmyk}{0.49,0.01,0,0}
\definecolor[named]{ACMGreen}{cmyk}{0.20,0,1,0.19}
\definecolor[named]{ACMPurple}{cmyk}{0.55,1,0,0.15}
\definecolor[named]{ACMDarkBlue}{cmyk}{1,0.58,0,0.21}

\hypersetup{colorlinks,
  linkcolor=ACMRed,
  citecolor=ACMPurple,
  urlcolor=ACMDarkBlue,
  filecolor=ACMDarkBlue
}

%%%%%%%%%%%%%%%%%%%%%%%%%%%%%%%%%%%%%%%%%%%%%%%%%%%%%%%%%%%%%%%
%% Paper setup
%%%%%%%%%%%%%%%%%%%%%%%%%%%%%%%%%%%%%%%%%%%%%%%%%%%%%%%%%%%%%%%

\newcommand{\sysname}{AWStream}
\newcommand{\para}[1]{\smallskip\noindent\textbf{#1}}
\newcommand{\paraf}[1]{\noindent\textbf{#1}}
\newcommand{\todo}[1]{{\color{ACMRed}\bf{TODO: #1}\normalfont}}
\newcommand{\fixme}[1]{{\color{ACMRed}\bf{FIXME: #1}\normalfont}}
\newcommand{\question}[1]{{\color{ACMRed}\footnotesize{Q: #1}\normalfont}}

\newcommand{\maybe}{\texttt{maybe}}

\def\Snospace~{\S{}}
\renewcommand*\sectionautorefname{\Snospace}
\renewcommand*\subsectionautorefname{\Snospace}
\renewcommand*\subsubsectionautorefname{\Snospace}
\renewcommand*{\equationautorefname}{Eq.}
\renewcommand*{\figureautorefname}{Fig.}

\newcommand{\specialcell}[2][c]{%
  \begin{tabular}[#1]{@{}c@{}}#2\end{tabular}}

\newcommand*\circled[1]{\tikz[baseline=(char.base)]{
    \node[shape=circle,draw,inner sep=0.5pt] (char) {#1};}}

\newcommand*\qe{$\text{Q}_\text{E}$}
\newcommand*\qc{$\text{Q}_\text{C}$}
\newcommand*\rc{$\text{R}_\text{C}$}
\newcommand*\spd{$\text{S}_\text{ProbeDone}$}

%%% Local Variables:
%%% mode: latex
%%% TeX-master: "thesis"
%%% End:
