\section{The Emerging Swarm}
\label{sec:emerging-swarm}

Smart devices are everywhere: \href{http://ilumi.co/}{light bulb},
\href{https://nest.com/}{thermostat}, \href{http://lunasleep.com/}{mattress
  cover}, \href{https://www.indiegogo.com/projects/smart-diapers}{e-diaper},
\href{https://www.fitbit.com/}{fitness tracker},
\href{https://www.fitbit.com/aria}{smart scale},
\href{https://www.myvessyl.com/}{smart cup},
\href{https://www.kickstarter.com/projects/1816678675/smartplate-instantly-track-and-analyze-everything}{smart
  plate},
\href{http://www.amazon.com/HAPILABS-102-HAPIfork-Bluetooth-Enabled-Smart/dp/B00FRPCPEC}{smart
  fork}, \href{http://electroluxdesignlab.com/en/submission/smart-knife/}{smart
  knife},
\href{http://www.clickandgrow.com/pages/what-is-click-grow}{flowerpot},
\href{http://www.williams-sonoma.com/products/breville-die-cast-2-slice-stainless-steel-smart-toaster/}{toaster},
\href{https://garageio.com/}{garage door},
\href{http://www2.withings.com/us/en/products/baby/smart-baby-monitor}{baby
  monitor},
\href{https://www.indiegogo.com/projects/smartmat-the-world-s-first-intelligent-yoga-mat}{yoga
  mat},
\href{http://usnews.rankingsandreviews.com/cars-trucks/best-cars-blog/2013/02/2015_GM_Vehicles_Will_Get_Wi-Fi_Internet_Access/}{sport-utility
  vehicle}, and
\href{http://www.samsung.com/us/appliances/refrigerators/RF28HMELBSR/AA}{refrigerator}.\footnote{Courtesy
  of Prof.\,Randy Katz.} Capable of computation and communication, they monitor
and interact with the physical world.  The growth of number of devices is
staggering: analysts predicted 26 billion devices by 2020 and trillion by
2022~\cite{middleton2013forecast}.

We refer to the collection of sensors and actuators installed in our environment
as the ``Swarm'', a term coined by Jan Rabaey during a keynote talk at ASPDAC
2008~\cite{rabaey2008brand}. This term well characterizes where the potentials
lie: it is not in the individual components, but rather the scale and the number
of interconnected devices, a shift from Moore's Law\footnote{The number of
  transistors on a chip doubles about every two years at the same cost.} to
Metcalfe's Law.\footnote{The effect of a telecommunications network is
  proportional to the square of the number of connected users of the system,
  i.e., $n^2$.}

At Berkeley, the Qualcomm Ubiquitous Swarm Lab~\cite{swarmlab} and Terraswarm
Research Center~\cite{terraswarm} were launched to address the huge potentials
and challenges for Swarm systems in December 2011 and January 2013,
respectively. Outside of Berkeley, this trend is also gaining attraction,
although the names can be different: Internet of Things
(IoT)~\cite{atzori2010internet}, Internet of Everything
(IoE)~\cite{bradley2013internet}, Industry 4.0~\cite{lasi2014industry}, The
Industrial Internet~\cite{eigner2018industrial}, Trillion Sensors
(TSensors)~\cite{bogue2014towards}, Machine to Machine
(M2M)~\cite{anton2014machine}, Smarter Planet~\cite{palmisano2008smarter}, etc.

Due to the widespread use of ``IoT'' outside of Berkeley, we use ``IoT'' and
``swarm'' interchangeably throughout this thesis. However, ``swarm'' is
preferred, because, quoting Lee~\cite{lee2016iot},

\begin{displayquote}
  The term ``IoT'' includes the technical solution ``Internet technology'' in
  the problem statement ``connected things.''
\end{displayquote}

\subsection{Swarm Applications}
\label{sec:swarm-applications}

With billions of devices interconnected, the swarm makes a large number of
applications possible. A previous survey by Atzori et al.\, groups applications
based on their domains: 1) transportation and logistics; 2) health-care domain;
3) smart environment (home, office, plant); 4) personal and
social~\cite{atzori2010internet}. We characterize the applications based on the
usage pattern. In this way, it is easier to analyze application requirements and
understand implications for system support. We characterize swarm applications
into two general categories.

\subsubsection{Ambient Data Collection and Analytics}
\label{sec:ambi-data-coll}

A wide range of swarm applications collect sensor data and monitor our
environment. They operate at vastly different scales:
cities~\cite{cheng2014aircloud, sfpark}, buildings~\cite{dawson2010smap},
homes~\cite{hnat2011hitchhiker}, and individuals (also referred to as Quantified
Self)~\cite{fitbit, swan2013quantified}. Because of the close interaction with
our environment, these applications often have a direct impact on our everyday
life and can help solve societal-scale problems. For example, in metropolitan
cities in developing countries, such as Beijing, the air quality has
deteriorated significantly. Air quality monitoring with fine granularity can
advise people with appropriate actions such as wearing masks or stay at
home~\cite{cheng2014aircloud}.

Many environmental sensors are collecting data about physical properties that
are intrinsically slow-changing, such as the occupancy, temperature, humidity,
etc. As a result, these sensors sample at low frequency (millihertz, i.e.,
slower than 1 Hz). For example, in a campus-wide building instrumentation that
aims to reduce energy usage~\cite{krioukov2012building}, there are 2,151
\href{http://www.keti.re.kr/}{KETI Motes}---463 \ce{CO2} sensors, 466 humidity
sensors, 26 illumination sensors, 440 light sensors, 290 PIR (passive infrared)
sensor, and 466 temperature sensors---configured to report data every five
seconds. For the air quality monitoring, the sensors only sample once a
minute~\cite{cheng2014aircloud}.

On the other hand, we are seeing high-frequency, high-precision sensors being
deployed at scale as technology evolves and applications' requirement
changes. For example, energy disaggregation takes a whole building's aggregated
energy signal and separates it into appliance-specific data. Recent works start
to sample at 15 kHz or higher to analyze the harmonics to identify type of
electrical circuitry in appliance~\cite{kolter2011redd} whereas a low sampling
rate can only show visual patterns and offer state
changes~\cite{hart1992nonintrusive}.

Regardless of the sampling frequency and data rate, in this category, the
collected data is not immediately inspected and often stored for future
anlaytics. Values and insights are gained by aggregated view, comparison with
historic data, and cross examination. With the increasing

Due to the nature of continuous sensing, aggregated historical data will be huge
and creating a new kind of big-data problem~\cite{diaz2012big,
  zaslavsky2013sensing}. For example, microsynchophasors, or uPMUs, monitor the
electrical grid with a network of 1000 devices; each produces 12 streams of 120
Hz high-precision values accurate to 100 ns. This amounts to 1.4 million points
per second and requires specialized storage systems for time-series
analysis~\cite{andersen2016btrdb}.

At last, many of these applications have serious privacy implications as the
collected data can be related with personal health or manufacture's operation
status.

\subsubsection{Real-time Applications with Low-Latency Requirements}
\label{sec:inter-low-latency}

Interactive applications that involve humans in the loop requires a low latency
response to engage the users. For example, to build an reactive
environment~\cite{cooperstock1997reactive} that engages the users attention and
gives gives the feeling of instantaneous response, the application only has a
budget of 100 ms~\cite{nielsen1994usability}. Wearable Cognitive
Assistance~\cite{chen2018application}.

In autonomous systems where humans are not involved (such as robots taking
actions based on sensors), tight control over latency is important for
deterministic applications~\cite{eidson2012distributed}. CPS.

Tight latency requirements with guarantee are often incompatible with the
unpredictable performance of cloud-based analytics or controllers. Timeliness is
not a factor in any widespread Internet technology, with Quality-of-Service
(QoS) features having been routinely omitted for decades.

\subsection{Heterogeneous Swarm Platforms}
\label{sec:swarm-platforms}

At the same time, we have seen a dizzying array of embedded platforms, from
powerful computing units to low-power microcontrollers (see
\autoref{tab:embedded}). None of these platforms \emph{must} connect with the
cloud; in fact the smallest devices require \emph{gateway} devices to even
communicate with cloud applications. Below are three categories of embedded
platforms:

\para{Microcontroller Platforms.}~Examples include Arduino~\cite{arduino},
mbed~\cite{mbed}, and Spark~\cite{spark}. This is an emerging category,
providing new open platforms on crowdfunding websites~\cite{kickstarter}, good
ecosystems featuring great support and libraries (e.g., Adafruit Online
Tutorials~\cite{adafruit}), and novel applications/products~\cite{iotlist}.

\para{Smartphones.}~Many companies (like Fitbit~\cite{fitbit} or
Automatic~\cite{automatic}) already use smartphones as gateways to connect
low-power devices to the network.  Researchers have explored how smartphones can
be used for IoT including reusing discarded smartphones~\cite{challen2014mote},
writing new operating systems~\cite{janos} and developing novel
applications~\cite{hong2014smartphone}. Mobile platforms.

\para{Mini PC.}~Ranging from the powerful Mac Mini and Intel Next Unit of
Computing (NUC) to inexpensive Raspberry Pi, BeagleBone Black, these devices
typically run various versions of Linux to simplify application deployment.
Many companies~\cite{ninja, smartthings, wink} adopt these mini PCs as their
gateway devices. Not mobile. Infrastructure.

\begin{table}
  \centering
  \begin{tabular}{c c c c}
    \toprule
    Device & CPU Speed & Memory & Price \\
    \midrule
    Intel NUC & 1.3 GHz & 16 GB & \texttildelow\$300 \\
    Typical Phones & 2 GHz & 2 GB & \texttildelow\$300 \\
    Discarded Phones\tablefootnote{This data is from \cite{challen2014mote}, where the
    original authors noted ``Customer buyback price quoted by Sprint for a
    smartphone in good condition.''}
           & 1 GHz & 512 MB & \texttildelow\$22 \\
    BeagleBone Black & 1 GHz & 512 MB & \$55 \\
    Raspberry Pi & 900 MHz & 512 MB & \$35 \\
    Arduino Uno & 16 MHz & 2 KB & \texttildelow\$22 \\
    %% http://arduino.cc/en/Products.Compare
    mbed NXP LPC1768 & 96 MHz & 32 KB & \$10 \\
    \bottomrule
  \end{tabular}
  \vspace*{-0.075in}
  \caption{The world of IoT includes a wide spectrum of computing platforms
    (price as of 2015).}
  \vspace*{-0.1in}
  \label{tab:embedded}
\end{table}

%% CSV Data:
%% Intel NUC:
%% http://www.intel.com/content/dam/www/public/us/en/documents/product-briefs/nuc-kit-d54250wyk-product-brief.pdf

% Platform, CPU, Memory, Price
% Intel NUC, 1.3 GHz*4, 16 GB,
% Apple A8, 1.4 GHz*2, 1 G, $600
% Nexus 6, 2.7 GHz*4, 3G, $600
% Raspberry Pi, 900MHz*4, 1G, $35
% BeagleBone, 720MHz, 256MB,
% Arduino, 16MHz, 8KB, $75
% mbed NXP LPC11U24, 48MHz, 8KB, $10
% mbed NXP LPC1768, 96MHz, 32KB, $10

\subsection{Swarm and Related Concepts}
\label{sec:swarm-relat-conc}

At first glance, the swarm may seem similar to other concepts such as wireless
sensor network (WSN), cyber-physical systems (CPS), and ubiquitous computing
(Ubicomp). However, as we will show with side-by-side comparisons, the swarm has
its fundamental differences and new challenges to address.

\subsubsection{Swarm and WSN}
\label{sec:swarm-wsn}

In the 1990s, ``Smart Dust''~\cite{kahn1999next} envisions wireless sensor nodes
with a volume of one cubic millimeter that can sense, communicate and
compute. These devices can then be spread through our environment and enhance
our interaction with the physical world. When answering the question
\question{what happens if sensors become tiny, wireless, and self-contained,}
researchers come up with the vision of a group of spatially dispersed and
dedicated wireless sensors to monitor and record the physical conditions of the
environment.

WSN offers the potential to dramatically advance scientific fields or extracting
engineering insights by enabling dense temporal and spatial measurements. The
literature is rich and interested readers are welcome to read
further~\cite{akyildiz2002wireless, zhao2009wireless}. We list a few notable
representative deployments as follows,

\begin{itemize}[itemsep=5pt]
\item \textbf{Redwoods}~\cite{tolle2005macroscope}. Tolle et al.\,deployed 80
  nodes at redwoods that recorded 44 days of air temperature, relative humidity,
  and photosynthetically active solar radiation data.
\item \textbf{GGB}~\cite{kim2007health}. Kim et al.\,deployed 59 nodes over the
  span and the tower, collecting ambient vibrations in two directions
  synchronously at 1KHz rate at Golden Gate Bridge for structural health
  monitoring
\item \textbf{ZebraNet}~\cite{zhang2005habitat}. It tracks animal migration by
  placing sensor nodes into the zebra's collar and the nodes travel with the
  animals.
\end{itemize}

While WSNs have great potential for many applications in a wide range of
scenarios, the swarm is one step further, as we illustrate below with their
differences:

\para{WSN is application-specific while the Swarm can re-purpose itself}.
Envision a futuristic “tale of two smart cities” with safe, efficient, and
comfortable transportation and communication during the best of times, and
secure, quick, and adapt- able emergency response during the worst of
times. Smart Cities use the TerraSwarm infrastructure to aggregate information
from multiple sources, and use this information to (for example) automatically
reroute traffic and identify health and safety threats, such as those created by
an earthquake or a terrorist attack. TerraSwarm applications identify
individuals who can benefit from information that has been gathered, and notify
them using local resources such as cell phones, nearby displays, or audio
systems. These sys- tems can be used to form response teams and implement a
range of rescue and security operations.

\para{WSN runs on a dedicated set of resources while the Swarm can dynamically
  recruit devices}. Sensor node failure due to environmental conditions, such as
underwater. While open architectures with dynamically recruit-able resources can
open up significant security and privacy risks, they can also make systems more
efficient (through sharing of resources), more resilient (through dynamic
reconfiguration leveraging redundant re- sources), and more capable, enabling
applications we have not yet invented or that cannot yet be realized.

\para{WSN nodes are significantly less heterogeneous than Swarm devices}.

\para{WSNs are mostly stand-alone while the swarm interacts with the cloud and
  others closely}. The swarm differs as it coexists and closely interacts with
the two other essential components of the global information-technology platform
that have emerged over the past decades: the cloud, offering unbounded
computational, communication and storage capacity; and the mobiles, which are
the preferred means of access to and creation of information today.

In summary, while the swarm finds its origin in the WSN concept, the swarm
represents a much broader vision, potentially connecting trillions of sensory
and actuating devices that are heterogeneous and worldwide into a single
platform abstraction.

\subsubsection{Swarm and CPS}
\label{sec:swarm-cps}

What is CPS? Cyber-physical systems (CPS) are physical and engineered systems
whose operations are monitored, coordinated, controlled and integrated by a
computing and communication core~\cite{rajkumar2010cyber}. For CPS, ``cyber''
and ``physical'' parts are tightly coupled in a feedback relation, continuously
affecting one another. This property of CPSs renders the separate understanding
of the physical and the cyber insufficient, and requires CPS research to focus
on the intersection of the two rather than the union by developing formalisms
that require a complete understanding of the cyber-physical interaction [83,
78]. Advancements in this domain have recently enabled disruptive applications,
most notably in industrial automation, manufactur- ing, transportation, wearable
technologies, and energy systems. Multiple research directions have emerged to
realize systems sharing the theme of connecting billions of devices to humans
and their environment, realizing large-scale ``smart'' systems. One such
direction that has been receiving significant attention from researchers and
investors is the Internet-of-Things (IoT).

CPS and swarm. The swarm can be viewed as a realization of CPSs, which
emphasizes the concept of intensively networked components. IoT envisions
leveraging Internet technology to connect and unprecedented number of devices,
yielding a ``swarm'' of heterogeneous sensors and actuators that can interact
with the physical environment and can be used to enable dynamic decision making
in many different domains [79, 80]. Of course, the resulting device swarms are
envisioned to be ``smart,'' continuously learning from behavioral patterns of
humans and other devices, then autonomously adapting to changes at run time.

Swarm challenges in addition to CPS. This ability builds upon the implicit
assumption that systems will be able to make real-time decisions on streaming
data. While the meaningful interpretation of this specification is exceedingly
context-dependent, for the purposes of the IoT, the implied requirement is the
ability of a system to contin- ually process streams of data in order to extract
actionable information.

\subsubsection{Swarm and Ubicomp}
\label{sec:swarm-wsn}

What is Ubicomp? Ubiquitous computing (or ``ubicomp'') is a concept in software
engineering and computer science where computing is made to appear anytime and
everywhere.

Swarm and Ubicomp.

Swarm challenges.

%%% Local Variables:
%%% mode: latex
%%% TeX-master: "../background"
%%% End:
