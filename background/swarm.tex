
\section{The Emerging Swarm}
\label{sec:emerging-swarm}

Rabaey~\cite{rabaey2011swarm} recognizes that,

\begin{displayquote}

  The functionality of the swarm arises from connections of devices, leading to
  a convergence between Moore’s and Metcalfe’s laws, in which scaling refers not
  any longer to the number of transistors per chip, but rather to the number of
  interconnected devices.

\end{displayquote}

\para{Swarm and Wireless Sensor Networks.} The swarm finds its origin in the
wireless sensor network concept that emerged in the mid 1990s. Yet, while the
original sensor nets were application-specific and limited in scope and reach,
the swarm represents a much broader vision, potentially connecting trillions of
sensory and actuating devices worldwide into a single platform abstraction. This
enables the true emergence of concepts such as cyber-physical and
cyber-biological systems, immersive computing, and augmented reality, and will
have a tremendous impact in domains such as advanced healthcare, improved energy
efficiency, environment-friendly living, mobility management, enhanced security,
and many others. As is shown in Fig. 1, the swarm coexists and closely interacts
with the two other essential components of the global information-technology
platform that have emerged over the past decades: the cloud, offering unbounded
computational, communication and storage capacity; and the mobiles, which are
the preferred means of access to and creation of information today. One may
argue that the swarm may ultimately subsume the mobile access layer, as
immersive computing and augmented reality environments with their much richer
set of information exchange models may totally eliminate the need for us to
carry a personal communication/computation device.

One of the primary aspects of the “swarm” is that future scaling is not in the
“number of transistors on a chip”, but in the “number of components connected in
the systems”, that is “Moore’s law meets Metcalfe’s law”. The conception,
analysis, design, deployment and management of swarm systems require
methodologies that differ substantially from our current understandings.  More
precisely, swarm systems are non-linear, non-deterministic, adaptive and complex


%%% Local Variables:
%%% mode: latex
%%% TeX-master: "../background"
%%% End:
