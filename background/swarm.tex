\section{The Emerging Swarm}
\label{sec:emerging-swarm}

Smart devices are everywhere: \href{http://ilumi.co/}{light bulb},
\href{https://nest.com/}{thermostat}, \href{http://lunasleep.com/}{mattress
  cover}, \href{https://www.indiegogo.com/projects/smart-diapers}{e-diaper},
\href{https://www.fitbit.com/}{fitness tracker},
\href{https://www.fitbit.com/aria}{smart scale},
\href{https://www.myvessyl.com/}{smart cup},
\href{https://www.kickstarter.com/projects/1816678675/smartplate-instantly-track-and-analyze-everything}{smart
  plate},
\href{http://www.amazon.com/HAPILABS-102-HAPIfork-Bluetooth-Enabled-Smart/dp/B00FRPCPEC}{smart
  fork}, \href{http://electroluxdesignlab.com/en/submission/smart-knife/}{smart
  knife},
\href{http://www.clickandgrow.com/pages/what-is-click-grow}{flowerpot},
\href{http://www.williams-sonoma.com/products/breville-die-cast-2-slice-stainless-steel-smart-toaster/}{toaster},
\href{https://garageio.com/}{garage door},
\href{http://www2.withings.com/us/en/products/baby/smart-baby-monitor}{baby
  monitor},
\href{https://www.indiegogo.com/projects/smartmat-the-world-s-first-intelligent-yoga-mat}{yoga
  mat},
\href{http://usnews.rankingsandreviews.com/cars-trucks/best-cars-blog/2013/02/2015_GM_Vehicles_Will_Get_Wi-Fi_Internet_Access/}{sport-utility
  vehicle}, and
\href{http://www.samsung.com/us/appliances/refrigerators/RF28HMELBSR/AA}{refrigerator}.\footnote{Courtesy
  of Prof.\,Randy Katz.} Capable of computation and communication, they monitor
and interact with the physical world.  The growth of number of devices is
staggering: analysts predicted 26 billion devices by 2020 and trillion by
2022~\cite{middleton2013forecast}.

We refer to the collection of sensors and actuators installed in our environment
as the ``Swarm'', a term coined by Jan Rabaey during a keynote talk at ASPDAC
2008~\cite{rabaey2008brand}. This term well characterizes where the potentials
lie: it is not in the individual components, but rather the scale and the number
of interconnected devices, a shift from Moore's Law\footnote{The number of
  transistors on a chip doubles about every two years at the same cost.} to
Metcalfe's Law.\footnote{The effect of a telecommunications network is
  proportional to the square of the number of connected users of the system,
  i.e., $n^2$.}

At Berkeley, the Qualcomm Ubiquitous Swarm Lab~\cite{swarmlab} and Terraswarm
Research Center~\cite{terraswarm} were launched to address the huge potentials
and challenges for Swarm systems in December 2011 and January 2013,
respectively. Outside of Berkeley, this trend is also gaining attraction,
although the names can be different: Internet of Things
(IoT)~\cite{atzori2010internet}, Internet of Everything
(IoE)~\cite{bradley2013internet}, Industry 4.0~\cite{lasi2014industry}, The
Industrial Internet~\cite{eigner2018industrial}, Trillion Sensors
(TSensors)~\cite{bogue2014towards}, Machine to Machine
(M2M)~\cite{anton2014machine}, Smarter Planet~\cite{palmisano2008smarter}, etc.

Due to the widespread use of ``IoT'' outside of Berkeley, we use ``IoT'' and
``swarm'' interchangeably throughout this thesis. However, ``swarm'' is
preferred, because, quoting Lee~\cite{lee2016iot},

\begin{displayquote}
  The term ``IoT'' includes the technical solution ``Internet technology'' in
  the problem statement ``connected things.''
\end{displayquote}

\subsection{Swarm Applications}
\label{sec:swarm-applications}

With billions of devices interconnected, the swarm makes a large number of
applications possible. A previous survey by Atzori et al.\, groups applications
based on their domains: 1) transportation and logistics; 2) health-care domain;
3) smart environment (home, office, plant); 4) personal and
social~\cite{atzori2010internet}. We characterize the applications based on the
usage pattern. In this way, it is easier to analyze application requirements and
understand implications for system support. We characterize swarm applications
into two general categories.\footnote{Note that these two categories are not
  exclusive to each other. Some applications may fall into both.}

\subsubsection{Ambient Data Collection and Analytics}
\label{sec:ambi-data-coll}

A wide range of swarm applications collect sensor data and monitor our
environment. They operate at vastly different scales:
cities~\cite{cheng2014aircloud, sfpark}, buildings~\cite{dawson2010smap},
homes~\cite{hnat2011hitchhiker}, and individuals (also referred to as Quantified
Self)~\cite{fitbit, swan2013quantified}. Because of the close interaction with
our environment, these applications often have a direct impact on our everyday
life and can help solve societal-scale problems. For example, in metropolitan
cities in developing countries, such as Beijing, the air quality has
deteriorated significantly. Air quality monitoring with fine granularity can
advise people with appropriate actions such as wearing masks or stay at
home~\cite{cheng2014aircloud}.

Many environmental sensors are collecting data about physical properties that
are intrinsically slow-changing, such as the occupancy, temperature, humidity,
etc. As a result, these sensors sample at low frequency (millihertz, i.e.,
slower than 1 Hz). For example, in a campus-wide building instrumentation that
aims to reduce energy usage~\cite{krioukov2012building}, there are 2,151
\href{http://www.keti.re.kr/}{KETI Motes}---463 \ce{CO2} sensors, 466 humidity
sensors, 26 illumination sensors, 440 light sensors, 290 PIR (passive infrared)
sensor, and 466 temperature sensors---configured to report data every five
seconds. For the air quality monitoring mentioned above, the sensors only sample
once a minute.

On the other hand, we are seeing high-frequency, high-precision sensors being
deployed at scale as technology evolves and applications' requirement
changes. For example, energy disaggregation takes a whole building's aggregated
energy signal and separates it into appliance-specific data. Recent works start
to sample at 15 kHz or higher to analyze the harmonics to identify type of
electrical circuitry in appliance~\cite{kolter2011redd} with a higher
accuracy. This contrasts with prior approaches using low sampling rates and only
relying on state changes~\cite{hart1992nonintrusive}.

Regardless of the sampling frequency and data rate, in this category,
applications do not inspect the data immediately and often store it for future
analytics and archival purposes. Data scientists may employ exploratory data
analysis to visualize the collected data for patterns, or compare data across
sensors and time for insights.

The aggregated historical data will keep increasing and create a new kind of
big-data problem~\cite{diaz2012big, zaslavsky2013sensing}, challenging our
existing ways of data storage and analysis. For example, microsynchophasors, or
uPMUs, monitor the electrical grid with a network of 1000 devices; each produces
12 streams of 120 Hz high-precision values accurate to 100 ns. This amounts to
1.4 million points per second and requires specialized storage systems for
time-series analysis~\cite{andersen2016btrdb}.

At last, because the collected data can be related with personal health or
manufacture's operation status, many of these applications have serious privacy
implications.

\subsubsection{Real-time Applications with Low-Latency Requirements}
\label{sec:inter-low-latency}

Interactive applications that involve humans in the loop require a low latency
response to engage the users. For example, Wearable cognitive
assistant~\cite{chen2018application} uses wearable devices, such as Google
Glass, for deep cognitive assistance (e.g., offering hints for social
interaction via real-time scene analysis). A general rule of thumb for systems
that give the feeling of instantaneous response is to bound the latency to 100
ms~\cite{miller1968response, nielsen1994usability}.

In autonomous systems where humans are not involved, a tight control over
latency is important for proper and deterministic
behavior~\cite{eidson2012distributed}, especially when it involves a group or
feedback control. Schulz et al.~\cite{schulz2017latency} summarizes several
latency critical uses, including factory automation, smart grids, intelligent
transport systems, etc.

Satisfying applications' tight latency requirements is challenging. Timing
behavior in software execution emerges from implementations rather than
specifications~\cite{lee2018real}. The goal for timing is often to speed up some
execution, but not to provide guaranteed and repeatable timing. Similarly, the
Internet provides ``best-effort'' service and fails to provide quality of
service (QoS) guarantees~\cite{shenker1995fundamental}. The situation is
becoming worse as swarm applications become more complex and generate large
amount of data.

\subsection{Heterogeneous Swarm Platforms}
\label{sec:swarm-platforms}

There has been a dizzying array of embedded platforms, from powerful computing
units to low-power microcontrollers (see \autoref{tab:embedded}).

\subsubsection{Microcontrollers}

Examples include Arduino~\cite{arduino}, mbed~\cite{mbed}, and
Spark~\cite{spark}. This emerging category continues to expand as new open
platforms appear on crowdfunding websites~\cite{kickstarter}.  With an
ecosystems featuring great support and libraries (e.g., Adafruit Online
Tutorials~\cite{adafruit}), people can easily add new sensors/actuators and
develop applications. These platforms have unleashed the creativity of millions
in building smart and interactive devices, spawning the Maker
Movement~\cite{dougherty2012maker}.

\subsubsection{Smartphones}

As a key enabler for mobile computing, smartphones have become ubiquitous. They
are equipped with an impressive set of sensors, including accelerometer,
gyroscope, proximity sensor, camera, microphone, etc. The processors and
displays on smartphones keep upgrading. In addition to use smartphones alone,
many companies (like Fitbit~\cite{fitbit} or Automatic~\cite{automatic}) use
smartphones as gateways to connect low-power devices to the Internet via
Bluetooth. Creative use of smartphones is also a hot topic in research,
including reusing discarded smartphones~\cite{challen2014mote}, writing new
operating systems~\cite{janos}, and developing novel
applications~\cite{hong2014smartphone}.

\subsubsection{Mini PCs}

Ranging from powerful Intel Next Unit of Computing (NUC) to inexpensive
Raspberry Pi, BeagleBone Black, these devices typically run various versions of
Linux to simplify application deployment. Many companies~\cite{ninja,
  smartthings, wink} adopt these mini PCs with all kinds of networking
capabilities---802.15.4, Bluetooth, WiFi, Ethernet---as their gateway devices.
In contrast to using smartphones as gateways, mini PCs serve as nearby
always-available infrastructure---the ``immobiles.''~\cite{swarmbox}.

\begin{table}
  \centering
  \begin{tabular}{c c c c}
    \toprule
    Device & CPU Speed & Memory & Price \\
    \midrule
    Intel NUC & 1.3 GHz & 16 GB & \texttildelow\$300 \\
    Typical Phones & 2 GHz & 2 GB & \texttildelow\$300 \\
    Discarded Phones\tablefootnote{This data is from \cite{challen2014mote}, where the
    original authors noted ``Customer buyback price quoted by Sprint for a
    smartphone in good condition.''}
           & 1 GHz & 512 MB & \texttildelow\$22 \\
    BeagleBone Black & 1 GHz & 512 MB & \$55 \\
    Raspberry Pi & 900 MHz & 512 MB & \$35 \\
    Arduino Uno & 16 MHz & 2 KB & \texttildelow\$22 \\
    %% http://arduino.cc/en/Products.Compare
    mbed NXP LPC1768 & 96 MHz & 32 KB & \$10 \\
    \bottomrule
  \end{tabular}
  \vspace*{-0.075in}
  \caption{The swarm includes a wide spectrum of computing platforms (price as
    of 2015).}
  \vspace*{-0.1in}
  \label{tab:embedded}
\end{table}

%% CSV Data:
%% Intel NUC:
%% http://www.intel.com/content/dam/www/public/us/en/documents/product-briefs/nuc-kit-d54250wyk-product-brief.pdf

% Platform, CPU, Memory, Price
% Intel NUC, 1.3 GHz*4, 16 GB,
% Apple A8, 1.4 GHz*2, 1 G, $600
% Nexus 6, 2.7 GHz*4, 3G, $600
% Raspberry Pi, 900MHz*4, 1G, $35
% BeagleBone, 720MHz, 256MB,
% Arduino, 16MHz, 8KB, $75
% mbed NXP LPC11U24, 48MHz, 8KB, $10
% mbed NXP LPC1768, 96MHz, 32KB, $10

\subsection{Swarm and Related Concepts}
\label{sec:swarm-relat-conc}

At first glance, the swarm may seem similar to other concepts such as wireless
sensor network (WSN), cyber-physical systems (CPS), and ubiquitous computing
(Ubicomp). These concepts all reflect a vision of how technology can affect our
lives if they become deeply connected to the physical world: through large scale
sensing, sense-compute-actuate, and pervasive computing. In this section, we
will show the difference with side-by-side comparisons. We argue that the swarm
has its fundamental differences and new challenges to address.

\subsubsection{Swarm and WSN}
\label{sec:swarm-wsn}

In the 1990s, ``Smart Dust''~\cite{kahn1999next} envisions wireless sensor nodes
with a volume of one cubic millimeter that can sense, communicate and
compute. Being so small, these devices can then be spread through our
environment and enhance our interaction with the physical world. This vision
quickly leads to deploying a group of spatially dispersed and dedicated wireless
sensors---a wireless sensor network---to monitor and record the physical
conditions of the environment

WSNs offer the potential to dramatically advance scientific fields or extracting
engineering insights with dense temporal and spatial measurements. Without
exhaustively listing the rich literature~\cite{akyildiz2002wireless,
  zhao2009wireless}, we show a few notable deployments as follows,

\begin{itemize}[itemsep=5pt]
\item \textbf{ZebraNet}~\cite{zhang2005habitat}. Zhang et al.\, deployed 7 nodes
  by placing sensor nodes into the zebra's collar at the 24,000-acre Sweetwaters
  Reserve at Kenya. The nodes traveled with zebras and measuresd GPS positions
  and sun/shade indication.
\item \textbf{Redwoods}~\cite{tolle2005macroscope}. Tolle et al.\,deployed 80
  nodes at the redwoods in Sonoma, California. The network recorded 44 days of
  air temperature, relative humidity, and photosynthetically active solar
  radiation data.
\item \textbf{GGB}~\cite{kim2007health}. Kim et al.\,deployed 59 nodes over the
  span and the tower of the Golden Gate Bridge. The network collected ambient
  vibrations in two directions synchronously at 1 kHz rate for structural health
  monitoring.
\end{itemize}

We then compared the swarm with WSNs and show that while the swarm finds its
origin in the WSN concept, the swarm represents a much broader vision,
potentially connecting trillions of sensory and actuating devices that are
heterogeneous and worldwide into a single platform abstraction.

\para{WSN is application-specific; the swarm can re-purpose itself}. Most WSNs
focus on specific use cases. Because energy is usually the primary concern in
deployed networks, the lifespan of a network benefits greatly from custom
hardware design and careful tuning of software stack. After the deployment,
sensor nodes are dedicated for the particular task. For the swarm, it is
important to be reprogrammed on the fly for other purposes. This is best
illustrated with the vision of ``a tale of two smart
cities''~\cite{lee2012terraswarm}: the swarm allows safe, efficient, and
comfortable transportation and communication during the best of times, and
secure, quick, and adaptable emergency response during the worst of times. The
swarm is more efficient through sharing of resources; it is more resilient
through dynamic recruiting; and it is more capable by enabling novel
applications beyond developers' imagination. At the same time, the open
architecture can lead to bigger security and privacy risks.

\para{WSN nodes are significantly less heterogeneous than swarm devices}. WSNs
typically are comprised of sensor nodes and gateway nodes. Sensors nodes are
often the same batch of hardware and form a star, tree or mesh network after
deployment. Gateway nodes are more powerful in processing capability, battery
life, and transmission range, acting as data aggregation point and bridge to
other networks. In comparison, each swarm node can have different types of
sensors and actuators. Their processing and networking capabilities will depend
on many factors, including tasks, prices, manufactures and hardware
availability.

\para{WSNs are mostly stand-alone while the swarm interacts with the cloud and
  others closely}. Once deployed, WSNs rarely interact with the external world,
except shipping sensor data to some central storage location via the
gateways. Therefore, when designing WSN, one has to carefully provision the
resources. The swarm, on the other hand, coexists and closely interacts with the
cloud and other essential components of the Internet. The cloud offers unbounded
computational, communication and storage capacity. By offloading tasks to the
cloud, the swarm can achieve what is beyond its own hardware capability.

\subsubsection{Swarm and CPS}
\label{sec:swarm-cps}

Cyber-physical systems (CPS), coined by Helen Gill in 2006, refers to the
integration of computation (the cyber) and physical processes (the
physical)~\cite{lee2015past}. CPS applications include industrial automation,
manufacturing, transportation, medical devices, energy systems, and distributed
robotics.

Because ``cyber'' and ``physical'' parts are tightly coupled in a feedback
relation, it is important to study the intersection, or the joint dynamics, of
the two for a complete understanding of the systems~\cite{rajkumar2010cyber}. It
is not sufficient to separately understand only one side. For many applications
only in the cyberspace, i.e., general-purpose software, timing is a performance
concern: a slower program is not an incorrect program and reducing the execution
time is considered an optimization. In CPS, because the decision made by the
cyber parts affects the physical processes and subsequent feedback reactions, a
precise control of the timing is a correctness concern. Another challenge that
is unique in the intersection of ``cyber'' and ``physical'' is the level of
concurrency. Physical processes are composed of many parallel
processes. Software, on the other hand, is often constructed as a sequence of
steps. The challenge is to bridging the inherent sequential cyber world with the
concurrent physical world.

The swarm can be viewed as a network of CPSs: leveraging Internet technology to
connect and unprecedented number of devices, yielding a ``swarm'' of
heterogeneous sensors and actuators that can interact with the physical
environment. As a result, the swarm also needs to address timing and concurrency
issues related with physical processes.

While the swarm can reuse technologies and methodologies developed from decades
of CPS research, its open architecture and the interoperability requirement with
third-party services have made the challenges harder. For example, in a
standalone CPS development, one can use specialized networks with synchronized
clocks and time-division multiple access (TDMA), including CAN busses,
FlexRay~\cite{flexray2005flexray}, and TTEthernet~\cite{steiner2008ttethernet},
to provide latency and bandwidth guarantees~\cite{lee2018real}. When this CPS
interacts with the public Internet and packets need to traverse the wide area
network, the guarantees are no longer possible.

\subsubsection{Swarm and Ubicomp}
\label{sec:swarm-wsn}

Ubiquitous computing (or ``ubicomp'') refers to the vision when computing is
made to appear anytime and everywhere, so ubiquitous that no one will notice
their presence~\cite{weiser1993ubiquitous}.

\todo{Elaborate.}

%%% Local Variables:
%%% mode: latex
%%% TeX-master: "../background"
%%% End:
