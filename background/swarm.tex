\section{The Emerging Swarm}
\label{sec:emerging-swarm}

The swarm refers to the collection of sensors and actuators installed in our
environment. Capable of computation and communication, they monitor and interact
with the physical world. The term ``swarm'' was coined by Jan Rabaey in a
keynote talk at ASPDAC 2008~\cite{rabaey2008brand}. Later the Qualcomm
Ubiquitous Swarm Lab~\cite{swarmlab} and Terraswarm Research
Center~\cite{terraswarm} were launched at Berkeley to address the huge
potentials and challenges for swarm systems in December 2011 and January 2013,
respectively. Outside of Berkeley, there are also buzz around a similar concept,
but with different names: Internet of Things (IoT)~\cite{atzori2010internet},
Internet of Everything (IoE)~\cite{bradley2013internet}, Industry
4.0~\cite{lasi2014industry}, The Industrial
Internet~\cite{eigner2018industrial}, TSensors (Trillion
Sensors)~\cite{bogue2014towards}, Machine to Machine
(M2M)~\cite{anton2014machine}, Smarter Planet~\cite{palmisano2008smarter}, etc.

Billions of connected devices. By 2022, 1 trillion networked sensors will be
embedded in the world around us, with up to 45 trillion in 20 years. Ubiquitous
connectivity between devices (the Internet) is almost a reality today.

The potentials of the swarm is best characterized by ``Metcalfe's law'', with
$n$ devices, the effect is proportional to the square of the number of connected
devices, i.e., $n^2$. This landscape change is further exacerbated by ``Moore's
Law'''s ending---approaching its ending. quoting Rabaey~\cite{rabaey2011swarm},

\begin{displayquote}

  The functionality of the swarm arises from connections of devices, leading to
  a convergence between Moore's and Metcalfe's laws, in which scaling refers not
  any longer to the number of transistors per chip, but rather to the number of
  interconnected devices.

\end{displayquote}

\subsection{Swarm and WSN}
\label{sec:swarm-wsn}

In the 1990s, ``Smart Dust''~\cite{kahn1999next} envisions wireless sensor nodes
with a volume of one cubic millimeter that can sense, communicate and
compute. These devices can then be spread through our environment and enhance
our interaction with the physical world. Wireless sensor networks (WSN) refers
to a group of spatially dispersed and dedicated sensors for monitoring and
recording the physical conditions of the environment and organizing the
collected data at a central location.

WSN is application-specific.

WSN runs on a dedicated set of resources.

WSN research focuses on energy.

While this approach provides performance guarantees and reliability,
application-specific and limited in scope and reach. In contrast, the swarm
represents a much broader vision, potentially connecting trillions of sensory
and actuating devices worldwide into a single platform abstraction.

While this approach provides performance guarantees and reliability, it prevents
economies of scale, and, more importantly, it prevents the explosion of
possibilities that results from sharing data and devices across applications.

While open architectures with dynamically recruitable resources can open up
signifi- cant security and privacy risks, they can also make systems more
efficient (through sharing of resources), more resilient (through dynamic
reconfiguration leveraging redundant re- sources), and more capable, enabling
applications we have not yet invented or that cannot yet be realized.

The swarm differs as it coexists and closely interacts with the two other
essential components of the global information-technology platform that have
emerged over the past decades: the cloud, offering unbounded computational,
communication and storage capacity; and the mobiles, which are the preferred
means of access to and creation of information today. (see Fig. 1).

The swarm finds its origin in the wireless sensor network concept that emerged
in the mid 1990s. Yet, while the original sensor nets were application-specific
and limited in scope and reach, the swarm represents a much broader vision,
potentially connecting trillions of sensory and actuating devices worldwide into
a single platform abstraction. This enables the true emergence of concepts such
as cyber-physical and cyber-biological systems, immersive computing, and
augmented reality, and will have a tremendous impact in domains such as advanced
healthcare, improved energy efficiency, environment-friendly living, mobility
management, enhanced security, and many others. As is shown in Fig. 1, the swarm
coexists and closely interacts with the two other essential components of the
global information-technology platform that have emerged over the past decades:
the cloud, offering unbounded computational, communication and storage capacity;
and the mobiles, which are the preferred means of access to and creation of
information today. One may argue that the swarm may ultimately subsume the
mobile access layer, as immersive computing and augmented reality environments
with their much richer set of information exchange models may totally eliminate
the need for us to carry a personal communication/computation device.

\subsection{Swarm and CPS}
\label{sec:swarm-cps}

Cyber-physical systems (CPS) are physical and engineered systems whose
operations are monitored, coordinated, controlled and integrated by a computing
and communication core~\cite{rajkumar2010cyber}. For CPS, ``cyber'' and
``physical'' parts are tightly coupled in a feedback relation, continuously
affecting one another. This property of CPSs renders the separate understanding
of the physical and the cyber insufficient, and requires CPS research to focus
on the intersection of the two rather than the union by developing formalisms
that require a complete understanding of the cyber-physical interaction [83,
78]. Advancements in this domain have recently enabled disruptive applications,
most notably in industrial automation, manufactur- ing, transportation, wearable
technologies, and energy systems. Multiple research directions have emerged to
realize systems sharing the theme of connecting billions of devices to humans
and their environment, realizing large-scale ``smart'' systems. One such
direction that has been receiving significant attention from researchers and
investors is the Internet-of-Things (IoT).

IoT systems, by definition, are a realization of CPSs, where the concept of
inten- sively networked components is highly emphasized. IoT envisions
leveraging Internet technology to connect and unprecedented number of devices,
yielding a “swarm” of heterogeneous sensors and actuators that can interact with
the physical environment and can be used to enable dynamic decision making in
many different domains [79, 80]. Of course, the resulting device swarms are
envisioned to be “smart,” continuously learning from behavioral patterns of
humans and other devices, then autonomously adapting to changes at run
time. This ability builds upon the implicit assumption that systems will be able
to make real-time decisions on streaming data. While the meaningful
interpretation of this specification is exceedingly context-dependent, for the
purposes of the IoT, the implied requirement is the ability of a system to
contin- ually process streams of data in order to extract actionable
information.

\subsection{Swarm Platforms}
\label{sec:swarm-platforms}


At the same time, we have seen a dizzying array of embedded platforms, from
powerful computing units to low-power microcontrollers (see
\autoref{tab:embedded}). None of these platforms \emph{must} connect with the
cloud; in fact the smallest devices require \emph{gateway} devices to even
communicate with cloud applications. Below are three categories of embedded
platforms:

\begin{enumerate}
\item \textbf{Smartphones:}~Many companies (like Fitbit~\cite{fitbit} or
  Automatic~\cite{automatic}) already use smartphones as gateways to connect
  low-power devices to the network.  Researchers have explored how smartphones
  can be used for IoT including reusing discarded
  smartphones~\cite{challen2014mote}, writing new operating systems~\cite{janos}
  and developing novel applications~\cite{hong2014smartphone}.

\item \textbf{Mini PC:}~Ranging from the powerful Mac Mini and Intel Next Unit
  of Computing (NUC) to inexpensive Raspberry Pi, BeagleBone Black, these
  devices typically run various versions of Linux to simplify application
  deployment.  Many companies~\cite{ninja, smartthings, wink} adopt these mini
  PCs as their gateway devices.

\item \textbf{Microcontroller platforms:}~Examples include
  Arduino~\cite{arduino}, mbed~\cite{mbed}, and Spark~\cite{spark}. This is an
  emerging category, providing new open platforms on crowdfunding
  websites~\cite{kickstarter}, good ecosystems featuring great support and
  libraries (e.g., Adafruit Online Tutorials~\cite{adafruit}), and novel
  applications/products~\cite{iotlist}.

\end{enumerate}

\begin{table}
  \centering
  \begin{tabular}{c c c c}
    \toprule
    Device & CPU Speed & Memory & Price \\
    \midrule
    Intel NUC & 1.3 GHz & 16 GB & \texttildelow\$300 \\
    Typical Phones & 2 GHz & 2 GB & \texttildelow\$300 \\
    Discarded Phones\tablefootnote{This data is from \cite{challen2014mote}, where the
    original authors noted ``Customer buyback price quoted by Sprint for a smartphone in good condition.''} & 1 GHz & 512 MB & \texttildelow\$22 \\
    BeagleBone Black & 1 GHz & 512 MB & \$55 \\
    Raspberry Pi & 900 MHz & 512 MB & \$35 \\
    Arduino Uno & 16 MHz & 2 KB & \texttildelow\$22 \\
    %% http://arduino.cc/en/Products.Compare
    mbed NXP LPC1768 & 96 MHz & 32 KB & \$10 \\
    \bottomrule
  \end{tabular}
  \vspace*{-0.075in}
  \caption{The world of IoT includes a wide spectrum of computing platforms
    (price as of 2015).}
  \vspace*{-0.1in}
  \label{tab:embedded}
\end{table}

%% CSV Data:
%% Intel NUC:
%% http://www.intel.com/content/dam/www/public/us/en/documents/product-briefs/nuc-kit-d54250wyk-product-brief.pdf

% Platform, CPU, Memory, Price
% Intel NUC, 1.3 GHz*4, 16 GB,
% Apple A8, 1.4 GHz*2, 1 G, $600
% Nexus 6, 2.7 GHz*4, 3G, $600
% Raspberry Pi, 900MHz*4, 1G, $35
% BeagleBone, 720MHz, 256MB,
% Arduino, 16MHz, 8KB, $75
% mbed NXP LPC11U24, 48MHz, 8KB, $10
% mbed NXP LPC1768, 96MHz, 32KB, $10


%%% Local Variables:
%%% mode: latex
%%% TeX-master: "../background"
%%% End:
