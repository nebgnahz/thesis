\section{Massive Adoption of Cloud Platforms}
\label{sec:cloud}

The cloud has become the \textit{de facto} central piece around which
modern web applications are constructed.

Cloud Computing is the delivery of computing services over the Internet. It
refers to both the applications delivered as services and the hardware and
systems software in the datacenters that provide those
services~\cite{armbrust2010view}. It has the potential to transform a large part
of the IT industry due to a number of advantages over traditional IT approaches:


\para{The cloud offers an illusion of infinite resources on demand}. This
eliminates the need for cloud users to plan far ahead for provisioning. The
elimination of an up-front commitment by Cloud users, thereby allowing companies
to start small and increase hardware resources only when there is an increase in
their needs.

\para{The cloud eliminates up-front cost and simplified resource
  management}. Computing as a utility allows companies to start small and
increase hardware resources only when there is an increase in their needs. This
is also useful for specialized computing resources, such as GPUs, TPUs and FPGA.

\para{Pay-as-you-go model benefits both users and providers}.  Users are
rewarded to release the resources that are no longer needed. As a result, the
cloud provider can reuse the resources, improving the utilization.

\para{The cloud facilitates data availability and sharing.} End users can access
the service ``anytime, anywhere'', share data and collaborate more easily, and
keep their data stored safely in the infrastructure

As a result, we have seen a huge recent trend of migrating computations and
services to the cloud.  Taking Amazon S3 as an example, over two trillion
objects were reported stored in the system back as of April
2013~\cite{barr2013amazon}.

Riding on this popularity of the cloud, many of today's IoT solutions arise by
connecting embedded platforms to the cloud as a universal computation and
storage backend. This approach has been taken by both industrial efforts (such
as Carriots~\cite{carriots}, GroveStreams~\cite{grovestreams}, SAMI~\cite{sami},
Xively~\cite{xively}) and academic research~\cite{gupta2014bolt,
  zachariah1001internet}. For example, Bolt~\cite{gupta2014bolt} provides data
management for the Lab of Things (LoT)~\cite{brush2013lab} and uses Amazon S3 or
Azure for data storage. Such direct connections often require an application
gateway~\cite{zachariah1001internet} to support low-power wireless
communications such as Z-Wave or Bluetooth Low Energy (BLE).  Even when devices
can utilize more standard communication protocols such as WiFi, the gateways
still exist as downloadable applications that run on cellphones or computers.
Companies tend to provide their own gateways (such as Ninja Sphere~\cite{ninja},
SmartThings Hub~\cite{smartthings}, Wink Hub~\cite{wink}); and researchers adopt
a similar approach (e.g., HomeHub for the LoT~\cite{brush2013lab}).

Admittedly, over the last few years, cloud computing has shaped the software
industry and made the development and deployment of web services easier than
ever. Public cloud providers such as Amazon, Microsoft, Google, Rackspace offer
pay-as-you-go services for the general public.  Such a service model has reduced
capital expenses, enabled elasticity for dynamic load adaption, and simplified
resource management~\cite{armbrust2010view}.

Tim O’Reilly believes that ``the future belongs to services that respond in real
time to information provided either by their users or by nonhuman
sensors.''~\cite{siegele2008let}. Such services will be attracted to the cloud
not only because they must be highly available, but also because these services
generally rely on large data sets that are most conveniently hosted in large
datacenters. This is especially the case for services that combine two or more
data sources or other services, e.g., mashups. While not all mobile devices
enjoy connectivity to the cloud 100\% of the time, the challenge of disconnected
operation has been addressed successfully in specific application domains, so we
do not see this as a significant obstacle to the appeal of mobile applications.

The fact that custom gateways are an integral part of IoT applications leads
directly to ``stovepipe'' solutions or balkanization. Data and services from one
company cannot be shared or utilized by devices from another company: connection
protocols, data formats, and security mechanisms (when present) are proprietary
and often undocumented.

%%% Local Variables:
%%% mode: latex
%%% TeX-master: "../background"
%%% End:
