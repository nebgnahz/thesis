\section{Massive Adoption of Cloud Platforms}
\label{sec:cloud}

The illusion of infinite computing resources available on demand, thereby
eliminating the need for Cloud Com- puting users to plan far ahead for
provisioning.  The elimination of an up-front commitment by Cloud users, thereby
allowing companies to start small and increase hardware resources only when
there is an increase in their needs.

The advantages of SaaS to both end users and service providers are well
understood. Service providers enjoy greatly simplified software installation and
maintenance and centralized control over versioning; end users can access the
service “anytime, anywhere”, share data and collaborate more easily, and keep
their data stored safely in the infrastructure. Cloud Computing does not change
these arguments, but it does give more application providers the choice of
deploying their product as SaaS without provisioning a datacenter: just as the
emergence of semiconductor foundries gave chip companies the opportunity to
design and sell chips without owning a fab, Cloud Computing allows deploying
SaaS—and scaling on demand—without building or provisioning a
datacenter. Analogously to how SaaS allows the user to offload some problems to
the SaaS provider, the SaaS provider can now offload some of his problems to the
Cloud Computing provider. From now on, we will focus on issues related to the
potential SaaS Provider (Cloud User) and to the Cloud Providers, which have
received less attention.

Over the last few years, cloud computing has shaped the software industry and
made the development and deployment of web services easier than ever.  Public
cloud providers such as Amazon, Microsoft, Google, Rackspace offer pay-as-you-go
services for the general public.  Such a service model has reduced capital
expenses, enabled elasticity for dynamic load adaption, and simplified resource
management~\cite{armbrust2010view}.

We have seen a huge recent trend of migrating computations and services to the
cloud.  Taking Amazon S3 as an example, over two trillion objects were reported
stored in the system back as of April 2013~\cite{barr2013amazon}.  Riding on
this popularity, application developers in the IoT space have blithely adopted
the cloud as a universal computation and storage backend.  This approach has
been taken by both industrial efforts (such as Carriots~\cite{carriots},
GroveStreams~\cite{grovestreams}, SAMI~\cite{sami}, Xively~\cite{xively}) and
academic research~\cite{gupta2014bolt, zachariah1001internet}.

Specialized computing resources, such as GPUs, TPUs and FPGA.

Cloud Pricing?

Such services will be attracted to the cloud not only because they must be
highly available, but also because these services generally rely on large data
sets that are most conveniently hosted in large datacenters. This is especially
the case for services that combine two or more data sources or other services,
e.g., mashups. While not all mobile devices enjoy connectivity to the cloud
100\% of the time, the challenge of disconnected operation has been addressed
successfully in specific application domains, 2 so we do not see this as a
significant obstacle to the appeal of mobile applications.

Tim O’Reilly believes that ``the future belongs to services that respond in real
time to information provided either by their users or by nonhuman
sensors.''~\cite{siegele2008let}. Such services will be attracted to the cloud
not only because they must be highly available, but also because these services
generally rely on large data sets that are most conveniently hosted in large
datacenters. This is especially the case for services that combine two or more
data sources or other services, e.g., mashups. While not all mobile devices
enjoy connectivity to the cloud 100\% of the time, the challenge of disconnected
operation has been addressed successfully in specific application domains, so we
do not see this as a significant obstacle to the appeal of mobile applications.

Many of today's IoT solutions arise by connecting embedded platforms to the
cloud.  For example, Bolt~\cite{gupta2014bolt} provides data management for the
Lab of Things (LoT)~\cite{brush2013lab} and uses Amazon S3 or Azure for data
storage\footnote{This approach is for ``efficiently sharing data across
  homes''~\cite{gupta2014bolt}.}.  Such direct connections often require an
application gateway~\cite{zachariah1001internet} to support low-power wireless
communications such as Z-Wave or Bluetooth Low Energy (BLE).  Even when devices
can utilize more standard communication protocols such as WiFi, the gateways
still exist as downloadable applications that run on cellphones or computers.
Companies tend to provide their own gateways (such as Ninja Sphere~\cite{ninja},
SmartThings Hub~\cite{smartthings}, Wink Hub~\cite{wink}); and researchers adopt
a similar approach (e.g., HomeHub for the LoT~\cite{brush2013lab}).

The fact that custom gateways are an integral part of IoT applications leads
directly to ``stovepipe'' solutions or balkanization. Data and services from one
company cannot be shared or utilized by devices from another company: connection
protocols, data formats, and security mechanisms (when present) are proprietary
and often undocumented.

%%% Local Variables:
%%% mode: latex
%%% TeX-master: "../background"
%%% End:
