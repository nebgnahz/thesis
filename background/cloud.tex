\section{Massive Adoption of Cloud Platforms}
\label{sec:cloud}

Cloud Computing~\cite{armbrust2010view} delivers computing services over the
Internet, offering computing as a utility. The term ``cloud'' often refers to
the hardware and systems in the datacenters that provide those computing
services. A cloud can be made public such that customers use it in a
pay-as-you-go manner, such as Amazon Web Services, Google AppEngine, and
Microsoft Azure. In this way, the cloud offers a number of advantages over
traditional IT approaches.

\begin{enumerate}
\item \textbf{Cost.} The cloud eliminates up-front capital expenses and resource
  management. Cloud users can focus on their application logic without hiring IT
  experts to manage on-site datacenters. It saves cloud users many
  time-consuming IT chores, including provisioning, physical configuration,
  hardware upgrades, software patching, etc.

\item \textbf{Elasticity.} The cloud offers an illusion of infinite resources on
  demand. It allows cloud users to start small without planning far ahead. Cloud
  users can increase their resources when the demand grows and decrease to
  reduce costs. They can also request specialized hardware resources, such as
  FPGA, GPU, and TPU~\cite{abadi2016tensorflow}.

\item \textbf{Utilization.} The pay-as-you-go model incentives users to release
  the resources that are no longer needed. As a result, these resources can be
  reused by cloud providers. Taking advantage of statistical multiplexing, the
  cloud can greatly improve resource utilization.

\item \textbf{Availability.} The cloud facilitates data availability and
  sharing. All end users can access the services ``anytime, anywhere'', making
  collaboration seamless. In addition, the cloud keeps the data safe and durable
  (with backups and disaster recovery).

\end{enumerate}

Because of these benefits, cloud computing has shaped the software industry and
made the development and deployment of web services easier than ever. It has
become the \textit{de facto} central piece around which modern web applications
are constructed.

Riding on this popularity of the cloud, many of today's solutions for the swarm
and IoT arise by connecting embedded platforms to the cloud as a universal
computation and storage backend. Many companies have taken this approach, such
as Carriots~\cite{carriots}, GroveStreams~\cite{grovestreams}, SAMI~\cite{sami},
Xively~\cite{xively}. They offer easy-to-use APIs, data processing,
visualization, and sample code for various hardware platforms. On the other
hand, academic research also explores what the cloud can
offer~\cite{gupta2014bolt, zachariah1001internet}.  For example,
Bolt~\cite{gupta2014bolt} provides data management for the Lab of Things
(LoT)~\cite{brush2013lab} and uses Amazon S3 or Azure for data storage.

%%% Local Variables:
%%% mode: latex
%%% TeX-master: "../background"
%%% End:
