\section{Massive Adoption of Cloud Platforms}
\label{sec:cloud}

Cloud Computing~\cite{armbrust2010view} delivers computing services over the
Internet, offering computing as a utility. The term ``cloud'' often refers to
the hardware and systems in the datacenters that provide those computing
services. A cloud can be made public such that customers use it in a
pay-as-you-go manner, such as Amazon Web Services, Google AppEngine, and
Microsoft Azure. In this way, the cloud offers a number of advantages over
traditional IT approaches.

\para{The cloud eliminates up-front cost and simplified resource
  management}. Cloud users can start small and only increase their resource
usages when the demand increases. This is also useful for specialized computing
resources, such as GPUs, TPUs and FPGA.

\para{The cloud offers an illusion of infinite resources on demand}.

This eliminates the need for cloud users to plan far ahead for provisioning. The
elimination of an up-front commitment by Cloud users, thereby allowing companies
to start small and increase hardware resources only when there is an increase in
their needs.

\para{Pay-as-you-go model benefits both users and providers}.  Users are
rewarded to release the resources that are no longer needed. As a result, the
cloud provider can reuse the resources, improving the utilization.

\para{The cloud facilitates data availability and sharing.} End users can access
the service ``anytime, anywhere'', share data and collaborate more easily, and
keep their data stored safely in the infrastructure

Because of the aforementioned benefits, cloud computing has shaped the software
industry and made the development and deployment of web services easier than
ever. It has become the \textit{de facto} central piece around which modern web
applications are constructed. The trend of migrating computation, storage and
services to the cloud is evident. Taking Amazon S3 as an example, over two
trillion objects were reported stored in the system back as of April
2013~\cite{barr2013amazon}.

Riding on this popularity of the cloud, many of today's IoT solutions arise by
connecting embedded platforms to the cloud as a universal computation and
storage backend. This approach has been taken by both industrial efforts (such
as Carriots~\cite{carriots}, GroveStreams~\cite{grovestreams}, SAMI~\cite{sami},
Xively~\cite{xively}) and academic research~\cite{gupta2014bolt,
  zachariah1001internet}. For example, Bolt~\cite{gupta2014bolt} provides data
management for the Lab of Things (LoT)~\cite{brush2013lab} and uses Amazon S3 or
Azure for data storage. Such direct connections often require an application
gateway~\cite{zachariah1001internet} to support low-power wireless
communications such as Z-Wave or Bluetooth Low Energy (BLE).

The IoT industry has benefited hugely from the economic model of the cloud.
With little investment in infrastructure, even novice users can start collecting
sensor data and streaming it back to the cloud~\cite{armbrust2010view}.  Several
IoT cloud platforms~\cite{carriots, grovestreams, xively, sami} have gone
further by offering easy-to-use APIs, data processing, visualization, and sample
code for various hardware platforms.

%%% Local Variables:
%%% mode: latex
%%% TeX-master: "../background"
%%% End:
